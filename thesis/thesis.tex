\documentclass[12pt, a4paper, twoside, openright, slovak]{book}

\usepackage[slovak]{babel}
\usepackage[utf8]{inputenc}
\usepackage[T1]{fontenc}
\usepackage[
	top=2.5cm,
	bottom=2.5cm,
	right=3.5cm,
	left=2.5cm
]{geometry}
%\usepackage{graphicx}
%\usepackage{hyperref}
%\usepackage{csquotes}
%\usepackage{listings}
%\usepackage{xcolor}
%\usepackage{enumitem}
%\usepackage{multirow}
%\usepackage{subcaption}
%\usepackage{fancyvrb}
\usepackage{titlesec}
\usepackage{setspace}
\usepackage{fancyhdr}
\usepackage{pdfpages}
\usepackage[fleqn]{amsmath}
\usepackage{tocloft}
\usepackage{csquotes}
\usepackage{expl3}
\usepackage[style=iso-numeric, backend=biber]{biblatex}
\addbibresource{literature.bib}

% Číslo kapitoly na rovnakom riadku ako názov
\titleformat{\chapter}{\normalfont\huge\bf}{\thechapter}{1em}{}

\raggedbottom
\newcommand{\emptypage}{\newpage\thispagestyle{empty}\mbox{}\newpage}
\newcommand{\signaturespace}[2]{
  % #1 = width of the dotted line
  % #2 = legend
  \begingroup
  \renewcommand{\arraystretch}{0}
  \begin{tabular}[t]{cc}
  \hspace*{0pt}
  \cleaders\hbox{\kern.6pt.\kern.6pt}\hskip#1\relax
  \hspace*{0pt}
  \\[0.5cm]
  #2
  \end{tabular}
  \endgroup
}

\pagestyle{fancy}
\fancyhf{}  % clear all header and footers
\fancyhead[L]{\nouppercase{\leftmark}}
\fancyfoot[LE, RO]{\thepage}
\fancypagestyle{plain}{
  \fancyhf{}%
  \renewcommand{\headrulewidth}{0pt}%
  \fancyhf[lef,rof]{\thepage}%
}


\setstretch{1.5}
\newcommand{\University}[0] {Slovenská technická univerzita v Bratislave}
\newcommand{\UniversityEN}[0] {Slovak University of Technology Bratislava}
\newcommand{\Faculty}[0] {Fakulta informatiky a informačných technológií}
\newcommand{\FacultyEN}[0] {Faculty of Informatics and Information Technologies}
\newcommand{\Thesis}[0] {Bakalárska práca}
\newcommand{\ThesisEN}[0] {Bachelor's Thesis}
\newcommand{\Title}[0] {Spracovanie dát generovaných senzorovou IoT sieťou}
\newcommand{\TitleEN}[0] {Data Processing for Sensor IoT Network}
\newcommand{\Author}[0] {Miroslav Hájek}
\newcommand{\Supervisor}[0] {Ing. Marcel Baláž, PhD.}
\newcommand{\PedagogicalSupervisor}[0] {Ing. Jakub Findura}
\newcommand{\RegNo}[0] {FIIT-1234-98765}
\newcommand{\Date}[0] {Máj 2022}
\newcommand{\DateEN}[0] {2022, May}
\newcommand{\StudyProgramme}[0] {Informatika}
\newcommand{\StudyProgrammeEN}[0] {Informatics}
\newcommand{\StudyField}[0] {Informatika}
\newcommand{\Institute}[0] {Ústav počítačového inžinierstva a aplikovanej informatiky}
\newcommand{\SignPlace}[0] {V Bratislave, }
\newcommand{\SignDate}[0] {1.5.2022}


\begin{document}

% Obal
\thispagestyle{empty}
{\centering
	{\large \University}\par
	{\large \Faculty}\par
	\vspace{\medskipamount}
	\RegNo
	\vfill
	\textbf{\large \Author}\par
	\vspace{1.5\bigskipamount}
	\textbf{\Large \Title}\par
	\vspace{1.5\bigskipamount}
	{\large \Thesis}\par
	\vfill
}
\begin{flushleft}
Vedúci práce:\quad \Supervisor{\Large \par}
\vspace{\medskipamount}
\Date
\end{flushleft}

\emptypage

% Titulný list
\thispagestyle{empty}
{\centering
	{\large \University}\par
	{\large \Faculty}\par
	\vspace{\medskipamount}
	\RegNo
	\vfill
	\textbf{\large \Author}\par
	\vspace{1.5\bigskipamount}
	\textbf{\Large \Title}\par
	\vspace{1.5\bigskipamount}
	{\large \Thesis}\par
	\vfill
}
\begin{flushleft}
{\setlength{\mathindent}{0.1cm}
\begin{align*}
& \text{Študijný program:} && \text{\StudyProgramme} \\
& \text{Študijný odbor:} && \text{\StudyField} \\
& \text{Miesto vypracovania:} && \text{\Institute} \\
& \text{Vedúci práce:} && \text{\Supervisor} \\
& \text{Pedagogický vedúci:} && \text{\PedagogicalSupervisor}
\end{align*}}
\vspace{2\bigskipamount}
\Date
\end{flushleft}

\emptypage

% Zadanie
\newpage
\thispagestyle{empty}
\includepdf[pages=1, scale=0.85]{zadanie}
\newpage

\emptypage
\pagenumbering{roman}

% Poďakovanie
\thispagestyle{empty}
\vspace*{\fill}
\section*{Poďakovanie}
Chcel by som sa poďakovať vedúcemu práce Ing.~Marcelovi Balážovi,~PhD. za ústretovosť, mnohé cenné pripomienky a podnety k vylepšeniam, usmernenia pri vytýčení zamerania a povzbudenie ku tvorivému preskúmaniu problematiky. 

Za poskytnutie  senzorovej jednotky a za postrehy ku formálnej stránke vďačím Ing.~Lukášovi Doubravskému. 

Tiež ďakujem svojmu kolegovi Ing.~Michalovi Juranyimu, ktorý ma za roky spolupráce mnohému priučil o vývoji softvéru.
Veľmi si cením morálnu podporu popri štúdiu od rodičov a od najbližšieho okruhu spolužiakov -- kamarátov.
\vspace{3cm}

%Anotácia
\thispagestyle{empty}
\section*{Anotácia}
\University \\
\uppercase{\Faculty}
\vspace{-8pt}
{\setlength{\mathindent}{0cm}
\begin{align*}
&\text{Študijný program:} && \text{\StudyProgramme} \\
&\text{Autor:} && \text{\Author} \\
&\text{\Thesis:} && \text{\Title} \\
&\text{Vedúci bakalárskej práce:} && \text{\Supervisor} \\
&\text{Pedagogický vedúci:} && \text{\PedagogicalSupervisor} \\
&\text{\Date}
\end{align*}}
V bakalárskej práci sa zameriavame na spôsoby spracovania signálov z vibrácií pri preprave, zachytených
senzormi akcelerácie. Zámerom je extrakcia čŕt záujmu z prúdu vzoriek do udalostí, čím sa redukujú vymieňané
dáta v senzorovej sieti. Analytická časť sa venuje fyzikálnemu opisu a číslicovému meraniu zrýchlenia s MEMS akcelerometrom,
z ktorého je možné získať ostatné kinematické veličiny. 

V časovej doméne nahliadame na sledovaný dej ako stochastický
proces vyjadriteľný metrikami deskriptívnej štatistiky. Významné okamihy sa prejavujú náhlou zmenou oproti predošlému správaniu
alebo nadprahovou úrovňou, čo sa zachytáva algoritmami na detekciu špičiek, ktoré predstavujú formu binárneho klasifikátora.
Vibrácie obsahujú rôzne frekvenčné zložky separovateľné Fourierovou a kosínusovou transformáciou algoritmom rýchlej Fourierovej
transformácie vo viacerých obmenách za aplikovania oknových funkcií. 

Spomenuté sú tiež filtre s konečnou impulznou odozvou
zohrávajúce úlohu pri úprave zdrojového signálu pred ďalšími stupňami spracovania. Prihliadame na obmedzenia vyplývajúce
s nasadenia riešenia na zariadenia Internetu vecí v kontexte Edge computing architektúry.

\emptypage

%Anotácia EN
\thispagestyle{empty}
\section*{Annotation}
\UniversityEN \\
\uppercase{\FacultyEN}
\vspace{-8pt}
{\setlength{\mathindent}{0cm}
\begin{align*}
&\text{Degree course:} && \text{\StudyProgrammeEN} \\
&\text{Author:} && \text{\Author} \\
&\text{\ThesisEN:} && \text{\TitleEN} \\
&\text{Supervisor:} && \text{\SupervisorEN} \\
&\text{Departmental advisor:} && \text{\PedagogicalSupervisorEN} \\
&\text{\DateEN}
\end{align*}}
In the bachelor thesis we focus on signal processing of vibrations during transport, captured
using acceleration sensors. The intention is to extract features of interest from the stream of samples into events, 
thereby reducing data exchange in the sensor network. The analytical part deals with the physical description of acceleration
and its digital measurement with the MEMS accelerometer, from which it is possible to obtain other kinematic quantities.

In the time domain, we look at the observed phenomenon as a stochastic process expressed by various descriptive statistics. 
Significant moments are manifested by a sudden change from the previous behavior or levels above the threshold
noticed by peak detection algorithms, which are a form of binary classifier. The vibrations contain different frequency 
components separable by Fourier and cosine transform by the Fast Fourier transform algorithm in several variants with the
application of window functions.

Finite impulse response filters are also mentioned as playing a role in adjusting the source signal before the next stages of 
the pipeline. We take into account the limitations of deployment on the Internet of Things devices in the context of the Edge 
computing architecture.

\emptypage 

% Obsah
\renewcommand{\contentsname}{Obsah}
\thispagestyle{empty}
\tableofcontents{}
\emptypage

% Kapitoly
\pagenumbering{arabic}

\chapter{Úvod}
Inteligentné senzorové systémy zariadení Internetu vecí (IoT) zaznamenávajú obrovskú kvantitu údajov z prostredia, kde
pôsobia. Prúdy vzoriek meraných veličín majú sami o sebe nízku informačnú hodnotu a zbytočne zaťažujú
prenosové pásmo komunikačných kanálov a kapacitu úložísk. Monitorovanie širokého rozsahu kladie požiadavky
na čo najmenšie výrobné náklady senzorových jednotiek a dlhodobú výdrž pri napájaní z batérií za minimálnej údržby.
Existuje preto potreba získané dáta spracovať do istej miery už v blízkosti ich zdroja, aby došlo k efektívnemu
využitiu dostupných prostriedkov.

Zameriame sa na sledovanie vibrácií a dolovanie čŕt záujmu z nich. Význam a dôležitosť sledovania vibrácií spočíva v ich v
výskyte u každého mechanického zariadenia pohybom jednotlivých súčiastok a trením v ložiskách. Ich nadmerná prítomnosť
býva spôsobená opotrebením dielov stroja a dôsledkom technických defektov. Ďalšou oblasťou hojnej prítomnosti vibrácií
je preprava osôb alebo tovaru, kde ich zapríčiňujú nerovnosti povrchu vozovky alebo koľaje v bode styku s kolesami, či aparát
ovplyvňujúci pohyb vozidla, čiže spaľovací alebo elektrický motor a brzdový systém.

Detekciou nežiaducich vibrácií v preprave sa dokáže zabezpečiť bezpečnosť pasažierov včasnou výmenou súčiastky,
ktorá by ovplyvnila prevádzkyschopnosť v kritických momentoch, a predísť nenávratnému poškodeniu krehkých materiálov,
znehodnoteniu reaktívnych substancií, či ich aktivácii v prípade výbušnín a pyrotechniky. Vibrácie sú súčasťou
nebezpečných prírodných úkazov a ich správna identifikácia má za následok varovania pre evakuáciu obyvateľstva
v oblasti postihnutej zemetrasením, či erupciou sopky, vedúcimi k ohrozenia zdravia osôb a poškodenia majetku.

V analýze problematiky sa zapodievame fyzikálnym modelom opisom vibrácií, od čoho sa odvíja metodika ich snímania
akcelerometrami typu MEMS a prevod do číslicovej podoby analógovo-digitálnym prevodníkom. Na spracovanie priebehu
signálu zrýchlenia sa pozrieme z troch hlavných hľadísk.

Integračné metódy umožňujú nadobudnúť odhad o relatívnej rýchlosti a polohe z akcelerácie. Na postupnosti pozorovaní
je možné nahliadať tiež v časovej doméne zužitkovaním základných agregačných a korelačných štatistík na odhalenie
náhlych zmien. Významne vyčlenené úrovne sú detegované
algoritmami na detekciu špičiek za rozličnej úspešnosti. Metódami transformácie do frekvenčnej oblasti sa
objavujú periodicky prítomné zložky, kde je opäť žiaduce upozorniť na momentálne prevládajúci spektrálny obsah.
Modely operujúce s vibračnými dátami by mali nasadené na IoT zariadenie kladúce svoje špecifické nároky a
obmedzenia.

Navrhneme kroky postupov na generovanie významných udalostí z nameranej akcelerácie. Po vyhodnotení úspešnosti
modelov v vlastných dátových sadách bude zámerom implementácia konfigurovateľnej senzorovej IoT jednotky.

\chapter{Analýza}

\section{Monitorovanie vibrácií a šoku}
Vibrácie sú periodickým kmitaním hmoty okolo rovnovážnej polohy vznikajúce excitáciou látky, ktorej je dodaná potenciálna energia, a zo zákona zachovania energie je následne premieňaná na kinetickú energiu. V realite dochádza pôsobením trenia k útlmu voľného oscilačného pohybu s časom  a pohybová energia sa uvoľňuje v podobe tepelnej alebo akustickej emisie do okolitého prostredia. Častejšie ako presné harmonické kmity sú pozorované náhodné vibrácie, ktorých vývoj nevieme dopredu predvídať. Naproti tomu šok, alebo aj prechodový jav, je náhle uvoľnenie kinetickej energie krátkeho trvania oproti prirodzenej oscilácii systému. 

Význam a dôležitosť sledovania vibrácií spočíva v ich výskyte u každého mechanického zariadenia a je zapríčinená pohybom jednotlivých súčiastok a trením v ložiskách. Ich nadmerná prítomnosť býva spôsobená opotrebením dielov stroja alebo nevyvážením rotačných častí, zakliesňovaním ozubených kolies, ako dôsledkoch iných technických defektov. V prevažnej väčšine prípadov ide o nežiaduci jav nakoľko zakladá zníženiu účinnosti so zvýšením hlučnosti ako vedľajšiemu produktu. 

Ďalšou oblasťou hojnej prítomnosti vibrácií je preprava osôb alebo tovaru cestnými a železničnými dopravnými prostriedkami, kde sú zapríčinené nerovnosťami povrchu vozovky alebo koľaje v bode styku s kolesami vozidla. Na zvýšenie ovládateľnosti vozidla a komfortu pasažierov sú kabíny odpružené od kolies tlmičmi. Lietadlá sú zasa pod vplyvom trenia vzduchu s trupom a krídlami konštrukcie, ktoré je ďalej zosilnené vzdušnými prúdmi a turbulenciami. 

Druhým významným faktorom podieľajúci sa na tvorbe vibrácii je aparát, ktorý uvádza vozidlo do pohybu, čiže hnací najčastejšie spaľovací, dieselový alebo elektrický motor, a mechanizmus, ktorý ho ho zastavuje, čím je brzdový systém. Jedná sa najmä o vplyv pohybu piestov, alebo rotora u elektrických vozidiel, a prenosu otáčavého pohybu motora cez oje hriadeľa na nápravy.

Detekciou nežiaducich vibrácií v preprave sa dokáže zabezpečiť aj bezpečnosť pasažierov včasnou výmenou súčiastky, ktorá by ovplyvnila prevádzkyschopnosť v kritických momentoch. Ich eliminácia dokáže predísť nenávratnému poškodeniu krehkých materiálov alebo znehodnoteniu reaktívnych substancií, či dokonca ich aktivácii v prípade výbušnín a pyrotechniky.

V neposlednom rade sú vibrácie súčasťou potenciálne nebezpečných prírodných úkazov a ich správna identifikácia má za následok varovania pre preventívnu evakuáciu obyvateľstva v oblasti, ktoré bude zasiahnutá zemetrasením, či erupciou sopky vedúcimi k ohrozenia zdravia osôb a poškodenia majetku.
 
\subsection{Meranie fyzikálnej veličiny akcelerácie}
Pohyb mechanického systému vystaveného vonkajším silám sa nazýva odozva, ktorej správanie opisuje zjednodušený model s jedným stupňom voľnosti kmitajúceho telesa s pružinou a tlmičom \cite{vibrations-shock}. Pri pôsobení vonkajšej sily $F$ na hmotu upevnenú na pružine vznikajú nútené vibrácie, ktoré ju vychylujú z rovnovážnej polohy. Uvedená sila je charakterizovaná druhým Newtonovým zákonom v tvare $F = ma$, kde $m$ je hmotnosť telesa a $a$ predstavuje zrýchlenie. 

V protismere pôsobí sila vyvolaná pružinou opísateľná cez vzťah $F_p = -kx$, kde $k$ je tuhosť pružiny ovplynená priemerom samotnej pružiny, priemerom drôtu z ktorého pozostáva, počtu a stúpania závitov, a $x$ je vychýlenie z rovnovážneho stavu. Fyzickým obmedzením telesa, ktorým je viazaný na pevnú podložku dochádza k takmer zaručenému návratu do rovnovážnej polohy, vynímajúc deformáciu, a to nám umožňuje merať intenzitu vibrácií cez zrýchlenie ťažidla. Pri použití trojosového akcelerometra, kedy sú evidované všetky tri priestorové súradnice časovo-premennej akcelerácie dostávame rovnicu vo vektorovom tvare: 
\begin{ceqn}\begin{align}
   \vec{a}(t) = \frac{\vec{F}(t)}{m}
\end{align}\end{ceqn}

Magnitúda akcelerácie s troma súradnicami je potom daná $L_2$ normou vektora:
\begin{ceqn}\begin{align}
   |a| = \sqrt{a_x^2 + a_y^2 + a_z^2}
\end{align}\end{ceqn}

\subsubsection{MEMS kapacitný akcelerometer}
Bežné inerciálne senzory na meranie zrýchlenia priamočiareho, ale aj rotačného pohybu (gyroskop), sa vyrábajú technológiou \emph{MEMS – mikromechanický systém}, kedy je celé zariadenie vrátane všetkých mechanických súčastí umiestnené na kremík procesom mikrovýroby vo viacerých vrstvách. Sila spôsobujúca zrýchlenie je potom meraná vychýlením vstavanej odpruženej hmoty vzhľadom na pevné elektródy, ktoré môžu byť usporiadané jednostranne alebo ako diferenčný pár \cite{mdof-mems-accelerometers}.

Pri diferenčnom páre spôsobí pohyb doštičky ťažidla medzi elektródami zmenu kapacít a ich rozdielom je možné zistiť aplikovanú silu a cez uvedený vzťah zrýchlenie. Na zvýšenie celkovej kapacity sa používa viacero párov elektród zapojených paralelne. Pred prevodom na číslicový signál musí napäťová úroveň zo senzora prejsť úpravou zahŕňajúcou nábojovocitlivý predzosilňovač, osovú demoduláciu a anti-aliasingové filtrovanie. Viacosové akcelerometre vyžadujú viaceré opísané štruktúry orientované kolmo na seba podľa počtu vyžadovaných stupňov voľnosti, pričom v skutočných senzoroch vždy existuje aspoň minimálna závislosť medzi osami rádovo najviac v jednotkách percent. Teplota ovplyvňuje citlivosť MEMS akcelerometrov len nepatrne, v stotinách percenta na stupeň Celzia.

Akcelerometre sa odlišujú v niekoľkých dôležitých vlastnostiach, ktoré zvyknú byť nastaviteľné vo výrobcom stanovenom rozsahu prípustných hodnôt s príslušnými toleranciami. \emph{Citlivosť} stanovuje najmenšiu rozlíšiteľnú zmenu v odčítanom napätí ku zmene externého pohybu, resp. zrýchlenia. Uvádza sa v jednotkách $mV/g$ (milivolt na tiažové zrýchlenie) pri analógovom výstupe, alebo $mg/\mathrm{LSB}$ (mili-g na najmenej významový bit). pri senzoroch so vstavaným analógovo-digitálnym prevodníkom. Jednotka $mg/LSB$ Vyjadruje o koľko sa zmení zrýchlenie keď zvýšime alebo ponížime binárne číslo na výstupe o jedna. Niekedy sa namiesto citlivosti uvádza mierka ako prevrátená hodnota citlivosti v $LSB/g$.

Tiažové zrýchlenie $g$ sa mierne líši podľa zemepisnej šírke, ale zaužívaný prepočet na jednotky SI je určený: $1 g = 9.80665 m/s^2$ \footnote{\url{https://physics.nist.gov/cgi-bin/cuu/Value?gn|search_for=acceleration}}. \emph{Dynamický rozsah} je uvádzaný v $g$ a hovorí o najmenšej a najväčšej rozlíšiteľnej hodnote zrýchlenia nad úrovňou ktorej už dochádza k skresleniu signálu orezaním špičiek.

S nevyhnutnými drobnými nepresnosťami výroby mikromechaniky je tzv. \emph{zero-g napätie} popisujúce odchýlku skutočného od ideálneho výstupu, keď na sústavu nepôsobí žiadne zrýchlenie. Za ideálnych okolností bez pohybu na vodorovnom povrchu namerajú osi $x$ a $y$ zrýchlenie $0g$, zatiaľčo na $z$ pôsobí $1g$. Očakávaním je nulová hodnota výstupného napätia a tým aj výstupného registra.

Nie často uvádzanou vlastnosťou býva \emph{frekvenčná odozva} senzora, ktorá určuje o koľko sa v rámci tolerancie odlišuje skutočná citlivosť od referenčnej pre zodpovedajúcu frekvenciu vibrácii.

% Čo je ODR?
\subsubsection{Analógovo-digitálny prevodník}
Spojitá napäťová úroveň transformuje analógovo-digitálny (A/D) prevodník pre spracovanie digitálnym systémom do množiny diskrétnych hodnôt. Vstupný signál najprv prechádza fázou vzorkovania, kedy sa vzorky zaznamenávajú v pravidelných intervaloch. Počet vzoriek odčítaných za sekundu je vyjadrený vzorkovacou frekvenciou $f_s$ v $Hz$. Časový rozdiel medzi vzorkami, nazývaný perióda vzorkovania, je prevrátenou hodnotou vzorkovacej frekvencie $T_s = \frac{1}{f_s}$. Pre presnú rekonštrukciu pásmovo obmedzeného signálu v hraniciach $[-f_{max}; f_{max}]$ je nevyhnuté podľa Nyquist-Shannonovej vety o vzorkovaní, aby vzorkovacia frekvencia bola najmenej dvojnásobkom maximálnej frekvencie snímaného signálu.
\begin{ceqn}\begin{align}
   f_s \geq 2 \cdot f_{max} 
\end{align}\end{ceqn} 

Každej vzorke je následne v procese kvantovania priradená diskrétna hodnota s konečným počtom $n$ bitov, ktorá je najbližšia možná ku skutočnej hladine analógového vstupu. Dochádza pritom k istému zaokrúhľovaniu z dôvodu nepresnosti vyjadrenia spojitého penza amplitúd diskrétnym číslom. Tento jav označujeme ako kvantizačný šum, ktorý je najviac polovicou z maximálnej rozlíšiteľnej zmeny signálu a trpia nim všetky existujúce A/D prevodníky. Rovnako tak sa u všetkých prevodníkov prejavuje aspoň nepatrná miera nelinearity výstupného kódu, chýbajúce kódy alebo ich nemonotónnosť.

Prevodníky integrované priamo s inerciálnymi jednotkami sa vyhotovujú v rozlíšeniach 12, 16 alebo 20 bitov. Umožňujú tak pripojiť akcelerometer rovno na sérové zbernice \emph{SPI} alebo \emph{I2C}. Všeobecne platí, že pri $n$ bitoch je k dispozícii $2^n$ rozličných čísel. Kódovaním v dvojkovom doplnku pre zachytenie záporných hodnôt sa uvažuje s intervalom $[-2^\frac{n}{2}; 2^\frac{n}{2} - 1]$. Napríklad pri 12-bitovom A/D prevodníku s referenčným napätím $3.3V$ je teoreticky najmenšia rozlíšiteľná zmena na najmenej významový bit $3.3V / 2^{12} = 0.81 mV$. Ak je rozhranie senzora priamo vybavené analógovým výstupom nič nebráni v použití presnejšieho prevodníka, napriek tomu najmenší merateľný dielik je zdola stále ohraničený citlivosťou akcelerometra. 

Číslicovú hodnotu v dvojkovom doplnku získanú konverziou $\hat{x}$ je z dôvodu širšej zrozumiteľnosti žiaduce prepočítať na štandardné fyzikálne jednotky pre zrýchlenie, $a$ v $m/s^2$). $R$ prestavuje nastavený dynamický rozsah v jednotkách $g$ a $n$ je počet bitov A/D prevodníka.
\begin{ceqn}\begin{align}
   a = \hat{x} \cdot \frac{R \cdot g}{2^\frac{n}{2}}
\end{align}\end{ceqn}

Ako bolo spomenuté ohľadom vlastností MEMS akcelerometrov, presnejší prevod dosiahneme zužitkovaním deklarovanej citlivosti senzora pri danom dynamickom rozsahu $S_R$ udávaného v $mg/LSB$.
\begin{ceqn}\begin{align}
   a = \hat{x} \cdot \frac{S_R \cdot g}{1000}
\end{align}\end{ceqn}

A/D prevodníky existujú v rôznych prevedeniach podľa žiaducej rýchlosti prevodu, presnosti a zložitosti vyhotovenia. Integračný prevodník sa spolieha na meranie času potrebného na vybitie kondenzátora v zapojení s obvodom operačného zosilňovača. Z pomedzi typov prevodníkov je zďaleka najpomalší. Aproximačný prevodník pozostáva z obvodu vzorkuj a podrž (sample and hold) komparátora, D/A prevodníka a výstupného registra. Diskretizácia signálu prebieha binárnym vyhľadávaním od najvýznamnejšieho bitu, rozhodujúceho o $\frac{1}{2}$ celkového rozsahu, počiatočne nastaveného na jedničku. Po D/A prevode slova sú napätia porovnané komparátorom. Ak je vstupná úroveň nižšia, bit sa nastaví na nulu a pokračuje sa postupne nižšími bitmi s dopadom na $\frac{1}{4}$, $\frac{1}{8}$ plného rozsahu atď. Okrem toho sa hojne používa Sigma-Delta A/D prevodník. Vyžaduje 1-bitový D/A prevodník zapojený v spätnoväzobnej slučke,
kedy sa nadvzorkuje vstupný signál v modulátore, poslaný je do digitálneho filtra, kde je vyhladený a reťaz zakončuje decimátor.

Na ilustráciu uvádzame parametre najrozšírenejších typov akcelerometrov\cite{mems-accel-mechanical-vibrations}. MPU6050 umožňuje použiť rozsahy merania v rozpätiach: $\pm2g$, $\pm4g$, $\pm8g$, $\pm16g$, pričom všeobecne sa každý rozsah vyznačuje svojou citlivosťou. Zvolením menšieho rozsahu spôsobíme väčšiu citlivosť, ktoré sú pre MPU6050 približne $0.06\,g/LSB$ pri $\pm2g$ až po $0.49\,g/LSB$ pri $\pm16g$. Vyznačuje sa taktiež 16-bitovým A/D prevodníkom. LSM9DS1 disponuje rovnakými rozsahmi a podobnými citlivosťami, s výnimkou rozsahu $\pm 16g$ s citlivosťou $0.73\, g/LSB$. ADXL-345 sa vyznačuje opäť pri totožných rozsahoch merania citlivosťou $0.004\,g/LSB$ s rozlíšením 10 bitov alebo až 13-bitov pri najväčšom rozsahu. Vyrábajú sa tiež akcelerometre s väčšími dynamickými rozsahmi napríklad ADXL357 so škálami $\pm 10\,g$, $\pm 20\,g$ a $\pm 40\,g$ s citlivosťou $0,02\,g/LSB$ po $0,08\,g/LSB$ a rozlíšením 20 bitov. Naproti tomu príbuzný akcelerometer ADXL357 nedisponuje žiadnym prevodníkom a má deklarovanú citlivosť 20 - 80 $mV/g$.

% Adaptívne vzorkovanie - preskočenie vzorky, keď je predpoklad, že na základe existujúcich vzoriek vieme dostatočne dobre odhadnúť budúce čítania. \cite{adaptive-sampling}. Efekt zberu až po prekročení prahu intenzity akcelerácie na signál.

\subsection{Odvodzovanie rýchlosti a polohy zo zrýchlenia}
fyzikálne vzťahy pre polohu, rýchlosť a zrýchlenie, numerická derivácia a integrácia (obdĺžníkové, lichobežníkové, Simposonovo pravidlo) \cite{integration-acceleration-envelopes}

\section{Metódy analýzy signálu v časovej doméne}
časový rad, okná
\cite{time-series-analysis} \cite{practical-time-series} \cite{generalized-esd} \cite{twitter-esd}, 
online algoritmus \cite{online-anomaly-detection}, 
požiadavky na efektívne algoritmy
	

\subsection{Číselné charakteristiky štatistického rozdelenia}
bodové odhady momentov. výberový priemer = stredná hodnota, výberový rozptyl (Welfordov online algoritmus), šikmosť, špicatosť, kvantily, normálne rozdelenie pravdepodobnosti

%\subsection{Prúdové algoritmy}
%stream algorithm models, Data Stream Model - cash register, turnstile, time series, window model
%Odhad momentov, Rátanie frekvencií, Zistenie či sa dopyt už vyskytol
%Count–min sketch (\url{https://florian.github.io/count-min-sketch/}) - probabilistická dátová štruktúra

\subsection{Algoritmy na rozpoznávanie špičiek}
lokálne minimá a maximá extrémy - cez prvú a druhú deriváciu,
topografická prominencia a izolácia - relatívna výška vrchola / extrému
vyhladzovanie: mean filter, derivative filter (diskrétny derivačný operátor) - sobel filter na , Savitzky–Golay filter
Vzájomná korelácia - jadra vrchola a signálu
detect peaks in a signal and to measure their positions, heights, widths, and/or areas

\cite{spectrometry-peak-detection}
\cite{survey-peaks-valleys}
\cite{peek-mountaineer-method}
\cite{ecg-r-peak-detection}
\cite{ampd-algorithm}

\subsubsection{Jednoduchý detektor vrcholov}
Preskočí špičky s absolútnou magnitúdou menšou ako `height`. Vo vnútornom cykle zisťuje či je bod vyššie ako všetkých `k` susedov doprava a doľava s toleranciou `epsilon`.

\begin{equation}
\forall c \in \{y_{i-k}, ... y_{i+k}\} - \{y_i\}; \; y_i - c \geq -\epsilon
\end{equation}

\begin{lstlisting}[language=Python]
def find_peaks(y: list, k: int, epsilon: float=0, height: float=None) -> list:
    spikes = []
    
    for a in range(k, y.size - k): 
        if height is not None and abs(y[a]) < height:
            continue
        locmax = True
        for b in range(-k, k):
            if b != 0 and y[a + b] - y[a] > epsilon:
                locmax = False
        if locmax:
            spikes.append(a)

    return spikes
\end{lstlisting}
\url{https://terpconnect.umd.edu/~toh/spectrum/PeakFindingandMeasurement.htm}

\subsubsection{Algoritmus Negative Zero-Crossing}
Nájdi vo n-krát vyhladenom signále body, že platí: $f'(x) = 0$, kde $f''(x) < 0$ a $|f''(x)| > \theta$. Pre odstránenie hraničných efektov použiť mód 'valid' s prísušným posunom hraníc polí.

Diskrétna verzia:
$$f'(x) = 0 \; \mathrm{(Dotyčnica)} \implies |y_{i+k} - y_{i-k}| \leq \epsilon \;\mathrm{(Sečnica)}$$
$$f''(x) < 0 \implies (y_{i+k} - y_i) - (y_i - y_{i-k}) < 0$$
$$|f''(x)| > \theta \implies |(y_{i+k} - y_i) - (y_i - y_{i-k})| > \theta$$

Parametre: 
\begin{itemize}
\item $\epsilon$: tolerancia pre nulovú deriváciu
\item $\theta$: strmosť druhej derivácie, čiže špicatosť vrchola
\item $k$: polovica dĺžka sečnice pre výpočet prvej derivácie
\item $n$: veľkosť konvolučného jadra
\item $smooth$: počet vyhladzovaní
\end{itemize}

\subsubsection{Algoritmus horského turistu}
Obsahuje pamäťový efekt pre dočasné zmeny a berie ich do úvahy ak lokálne záchvevy prekročia tolerovanú úroveň. Ak sa zmení 'slope' oproti predošlému kroku DeltaY, tak hneď neoznačí za zmenu medzi dolinou a vrcholom, ale až po prekročení nastavených prahov. Skutočnosť z reality: jama != údolie


\subsection{Online detekcia anomálií a odchýliek pozorovaní}
Outlier je pozorovanie, ktoré sa odchyluje, tak významne od ostatných pozorovaní, že vzbudzuje
podozrenie, že bolo vytvorené odlišným mechanizmom. Normálne dáta, Šum, Anomálie (slabé a silné odchýlky)
Výstupy algoritmov na detekciu výchyliek: 
\begin{itemize}
\itemsep0pt
\item Outlier skóre - miera vychýlenosti bodu
\item Binárne štítky - binárne štítky, označenie, či je bod anomália alebo nie.
\end{itemize}
Algoritmy na anomálie vytvárajú model normálnych vzorov v dátach a skóre vychýlenosti je dané deviáciou od týchto vzorov.
\cite{outlier-analysis} 
Hampel filter

\cite{survey-univariate-time-series} 
Množstvo prenášaných a spracovaných dát prevyšuje ľudskú schopnosť manuálneho prieskumu. Anomália je pozorovanie alebo postupnosť pozorovaní, ktoré sa významne odchyluje od distribúcie zvyšku dát. 
\begin{itemize}
	\item Bodové anomálie - $x_t$ je bodová anomália ak sa jeho hodnota významne odlišuje od všetkých 		
		bodov v intervale $ [x_{t-k}; x_{t+k}] $
	\item Kolektívne (disonanancie) - jednotlivé body nepredstavujú anomálne správanie, až ak vezmeme dlhšiu postupnosť môže byť označená za anomáliu.
	\item Kontextové výchylky / anomálie - body sú normálne v určitom kontexte, ale v inom anomálne.
\end{itemize}		

\cite{review-outlier-datection} \cite{anomaly-detection-algorithms}
\begin{equation}
|x - \hat{x}| \geq \theta 
\end{equation}

Bežiaci filter  
\cite{anomaly-detection-models}

\subsubsection{Metriky pre binárny klasifikátor}
Skóre anomálnosti, 	binárny klasifikátor, 
Falošne pozívny - proces je normálny, ale registrujeme neočakávané správanie, 
Falošne negatívny - proces je abnormálny, ale správanie prechádza bez povšimnutia
Matica zámen, Precision, Recall, 
ROC - kvalita binárneho klasifikátora v závislosi od prahu, AUC
\cite{wsn-outlier-detection-survey}

Z-score a Z-test: 
\begin{equation}
z = \frac{|x - \mu|}{\sigma} \approx \frac{|x - \bar{x}|}{S}
\end{equation}
Bežiaci priemerovací filter \cite{anomaly-detection-models}

Zisťovanie výchyliek testovaním štatistických hypotéz: 
generalized extreme Studentized deviate test \cite{generalized-esd} 

Median Absolute Deviate (robustná štatistika na určenie vychýlenosti od normálu)
\begin{equation}
MAD = \mathrm{median}(|X_i - \bar{X}|)
\end{equation}


\section{Frekvenčná a časovo-frekvenčná analýza signálu}
vlastnosti frekvenčného spektra, decibele, spektrogram, odstup od šumu (SNR), power spectrum density (dbFS), spektrálny analyzátor 

\subsection{Fourierová transformácia}
Diskrétna fourierová transformácia mapuje signál dĺžky $N$ do množiny $N$ diskrétnych frekvenčných komponentov. \cite{signal-processing}
\begin{equation}
X = \mathbf{W}x; W_{nk} = e^{-i\frac{2\pi}{N}nk} = W_N^{nk}
\end{equation}
Inverzná transformácia
\begin{equation}
x = \frac{1}{N}\mathbf{W}^H X
\end{equation}

Integrálne transformácie: Fourierová transformácia (CFT, DFT), Kosínusová transfomácia (MDCT), Vlnková transformácia (CWT) \cite{dct} \cite{casove-frekvencia-analyza-signalu}

\begin{equation}
T(n) = \int{f(t) K(t,x) \mathrm{dt}}
\end{equation}

\begin{equation}
\mathcal{F}: X(\omega) = \int_{-\infty}^{+\infty}{x(t) \cdot e^{-i\omega t} \mathrm{dt}}
\end{equation}

\subsection{Algoritmus FFT pre DFT a DCT}
Opis DIT radix-2 FFT algoritmus komplexných, reálny, pre MDCT \cite{fft-blackbox}

\begin{equation}
X(m) = \sum_{n = 0}^{N-1}{x(n) \cdot e^{-i2\pi n m / N}}
\end{equation}

Frekvenčné rozlíšenie
\begin{equation}
\Delta f = \frac{f_s}{N}
\end{equation}

\subsection{Oknové funkcie}
Gaborová transformácia, Prehľad okien a ich transformácií (sinc), Efekt oknových funkcií na spectral leakage, výhodné percentá prekryvu FT 	\cite{understanding-dsp} \cite{spectral-density-estimation}
Priemerovanie a prekryv - Amplitude Flatness (AF), Power Flatness (PF), Overlap Correlation (OC)

\subsection{Filtre s konečnou impulznou odozvou}
Roziel medzi FIR a IIR, Dolná pripusť, pásmová priepusť, horná pripusť,
 Konvolúcia a konvolučné jadro, konvolučná veta, účel: identifikácia prítomnosti známej frekvencie v signále akcelerácie
\begin{equation}
y(n) = \sum_{k=0}^{D_y}{x(k) \cdot h(n-k)} = x(n) * h(n)
\end{equation}

Prenosová funkcia
\begin{equation}
H(\Omega) = \frac{Y(\Omega)}{X(\Omega)}
\end{equation}	

\subsubsection{Detektor obálok}
\footnote{\url{https://www.mathworks.com/help/dsp/ug/envelope-detection.html}}
\footnote{\url{https://www.dsprelated.com/showarticle/938.php}}

\section{Senzorová sieť}
Nízko-energetické zariadenia komunikujúce cez odľahčené sieťové protokoly so snahou spracovania v reálnom čase a ponechaním najdôležitejších informácii dolovaním z veľkého množstva zdrojových dát. Cloud / Fog comuting. Sink a Edge nodes

Vlastnosti senzorovej siete
\begin{itemize}
\itemsep0em 
\item Autokonfigurácia senzora - reakcia na zmeny v sieti a prostredia pôsobenia
\item Škálovateľnosť - veľké množstvo senzorov so spoločným účelom a schopnosťou vzájomnej kooperácie a interoperability.
\item Odolnosť voči chybám - v prípade pridania alebo odobratia uzla budú spojenia bez prerušenia.
\item Energeticky efektívna komunikácia uzlov - s upravenými protokolmi štandardného sieťového zásobníka
\begin{itemize}
\itemsep0em 
\item Event-driven - stály zber dát a reakcia na náhle zmeny. posielajú údaje až po prekročený kritického prahu
\item Query-driven - zbierajú údaje iba po prijatí dopytu od používateľa
\item Time-driven - pravidelne odosielajú údaje sinku. vzorkovaciu frekvenciu volí sink
\end{itemize}
\end{itemize}
\cite{wsn-overview}

Spracovanie toku informácií (IFP - Information flow processing) - nástroj na včasné spracovanie dát ako tečie z periférií do centra systému. Snahou je ukladanie agregovaných štatistík, napr. detektor požiaru za použitia čidiel teploty a dymu nepotrebuje ukladať jednotlivé merania, lebo sú samo o sebe nepodstatné. Keď nastane varovná situácia, je potrebné aby tá obsahoval všetky údaje na lokalizáciu ohniska.

CEP - Complex event processing - spracúva toky udalosti zo zdrojov reálneho sveta na základe aplikovania aktívnych pravidiel stanovených správcami systému a poupraví toky do komplexnejšieho výstupu. Pravidlá sú v tvare Event-Condition-Action (ECA).
\begin{itemize}
\itemsep0em
\item Udalosť - definuje zdroje ako generátory udalostí
\item Podmienka - uvažuje ktorá časť udalosti bude braná do úvahy pri spracovaní, napr. môže ísť o prekročenie prahu
\item Akcia - aká sada úloh má byť vykonaná pri detekcii udalosti
\end{itemize}

Behové pravidlá sú spracúvané vo viacerých fázach
\begin{itemize}
\item Signalizácia - detekcia udalosti
\item Spustenie - asociácia udalosti so sadou pravidiel
\item Vyhodnotenie - vyhodnotenie podmienky
\item Plánovanie - stanovenie poradia vykonania
\item Vykonanie - vykonanie pravidlá
\end{itemize}
\cite{processing-information-flows}

\subsection{Senzorová jednotka}

Súčasti senzorovej jednotky:
\begin{itemize}
\item Zberná jednotka
\item Výpočtová jednotka
\item Komunikačná jednotka
\item Napájacia jednotka
\end{itemize}

Obmedzenia na senzorové uzly 
\begin{itemize}
\item Spotreba energie - energetická autonómia uzlov vo WSN umožňuje nasadzovanie zariadení do odľahlých miest pre využitie v inteligentných mestách alebo na účely ochrany prírody. životnosť senzorovej jednotky je ohraničená kapacitou batérie.
\item Dosah komunikácie - Senzory disponujú obmedzenou energiou na vysielanie a dosah je negatívne ovplyvnený silou signálu na anténe. Z toho vyplývajú aj nižšie prenosové rýchlosti.
\item Výpočtový výkon a úložisko - Nízka taktovacia frekvencia procesora v megaherzoch a veľkosti pracovných pamätí v stovkách kilobajoch alebo megabajtoch.
\end{itemize}
\cite{big-data-collection-wsn}


Za ideálnych okolností by sa mal online algoritmus učiť kontinuálne bez ukladania predošlých bodov a detekcií.
V rozhodnutiach algoritmu sú zahrnuté informácie o všetkých predošlých bodoch do terajšieho rozhodnutia. Mal by mať schopnosť sa adaptovať dynamickému prostrediu, v ktorom pôsobí. Bez nutnosti manuálnych úprav parametrov modelu. Zároveň je žiaduce minimalizovať falošné pozitíva a negatíva pri detekcii udalostí.

Kontinuálne spracovanie častokrát vyžaduje, aby boli dáta rozdelenie do rovnako dlhých blokov zvaných okná. Okno $\mathcal{W}_{\sigma, \pi}$ je funkciou morfujúca maticu vstupných dát $X$ do vektora $w = (W_1, ... , W_q)$  \cite{online-anomaly-detection}

	
\emptypage 


\chapter{Návrh riešenia}
V súlade so stanovenými kritériami navrhneme postupnosť krokov úpravy nameraného pohybu dopravného prostriedku. 
Zhotovenú konfigurovateľnú sústavu uplatníme na zariadení senzorovej jednotky za účelom ohlasovania udalostí 
o vybraných pravidelných rysoch signálu. Prihliadať sa bude viac na splnenie úloh v reálnom čase pri
dosiahnuteľnom výkone, v dostupnom pamäťovom priestore, ako na energetickú úsporu. Výmena údajov sa má 
odohrávať zrozumiteľným, širko podporovaným formátom, za obmedzenia nadbytočného sieťového prenosu 
v najvyššej možnej miere.


\section{Špecifikácia požiadaviek}
Zariadenie internetu vecí určené na zber a analýzu vibrácií bude realizovať nasledujúce funkčné požiadavky: 
\begin{itemize}[noitemsep,topsep=0pt]
\item Zber trojosovej akcelerácie s nastaviteľnou vzorkovacou frekvenciou a dynamickým rozsahom akcelerometra, v hraniciach danými 
obmedzeniami hardvéru, najmenej však intenzity vyskytujúcej sa pri preprave konvenčnými pozemnými motorovými vozidlami.
\item Spracovanie osí akcelerácie nezávisle s obmedzením výberu aktívnych osí.
\item Vzdialene realizovateľná zmena parametrov jednotlivých stupňov sústavy na úpravu akceleračného signálu v posuvných oknách.
\item Ukladanie nameranej akcelerácie na pamäťovú kartu s ohľadom na najvyššiu dosiahnuteľnú rýchlosť zápisu.
\item Identifikácia významných frekvencií zo vzoriek akcelerometra so zachytením ich trvania a amplitúdy podľa aktuálnej 
konfigurácie detekčných algoritmov.
\item Sumarizácia hodnôt akcelerácie po posuvných oknách do popisných štatistík.
\item Odosielanie zachytených udalostí cez spoľahlivé sieťové spojenie za dosiahnutia redukcie množstva dát oproti priamo odčítaným vzorkám. 
\item Poskytnutie možnosti odosielania výstupov z podstatných medzikrokov spracovania za účelom potenciálneho ponechania 
analýzy pre ďalšie úrovne senzorovej siete alebo skupinovej koordinácie údajov z viacerých uzlov. 
\end{itemize}
\bigskip

Z povahy okolností nasadenia firmvéru na relatívne zdrojovo oklieštené Edge zariadenie vyplývajú 
vymenované nie-funkčné požiadavky, prevažne na účinnosť a prenositeľnosť:
\begin{itemize}[noitemsep,topsep=0pt]
\item Firmvér sa zmestí do programovej pamäte s rezervou pre budúce rozširovanie funkcionality.
\item Ľubovoľný scenár spracovania musí prebehnúť v reálnom čase.
\item Odozva detekcie udalostí bude najkratšia možná.
\item Výmena údajov cez sieťové protokol v štandardizovanom formáte hierarchickej štruktúry za najmenšej uskutočniteľnej réžie.
\item Platformová závislosť sa obmedzí na nevyhnuté súčasti systému ako sú hardvérové ovládače a akcelerácia náročných výpočtov.
\end{itemize}

\section{Hardvér senzorovej jednotky}
Navrhované zariadenie bude postavené na mikrokontroléri ESP32 od Espressif, pretože pri nízkej obstarávacej cene
ponúka možnosť konektivity na 2,4 GHz s Wifi 802.11 b/g/n a Bluetooth 4.2. V porovnaní s podobnými zariadeniami disponuje 
nebývalým výpočtovým výkonom a kapacitou pamätí.

Konkrétne zvolíme dosku FireBeetle osadenú modulom \emph{ESP32-WROOM-32D} s typickým napájacím napätím 3,3 V a dvoj-jadrovým 
32-bitovým procesorom Xtensa s taktovacou frekvenciou od 80 do 240 MHz. Modul obsahuje až 520 kB zdieľanej SRAM 
pre inštrukcie a dáta. 

Použitý model akcelerometera je súčasťou MEMS inerciálnej meracej jednotky \emph{LSM9DS1} zabudovanej na doske
STEVAL-MKI159V1 slúžiacej na adaptáciu pre púzdro DIL24. Akcelerometer bude s mikrokontrolérom komunikovať cez
poloduplexnú SPI zbernicu s horným obmedzením frekvencie hodín na 8 MHz. Navyše budú zapojené aj vývody prerušení INT1 a INT2
pre upozornenie prekročenia určených prahových úrovní. Blokový diagram zapojenia je na schéme na obr. \ref{schematics:block}.

\begin{figure}[h]
\centering
\begin{subfigure}[b]{0.65\textwidth}
    \centering
    \includegraphics[width=\textwidth]{figures/design/block-circuit-diagram.png}
    \caption{Blokový diagram modulov}
       \label{schematics:block}
\end{subfigure}
\hfill
\begin{subfigure}[b]{0.3\textwidth}
    \centering
    \includegraphics[width=\textwidth]{figures/design/fet.png}
    \caption{Ovládanie napájania OpenLog cez FET}
    \label{schematics:fet}
\end{subfigure}
\label{schematics}
\caption{Schéma zapojenia hardvéru}
\end{figure}
 
Pamäťová Micro SD karta so súborovým systémom FAT32 bude pripojená v module \emph{OpenLog} od Sparkfun, ktorý zaznamená znakový
vstup prijímaný UART zbernicou do  textových súborov podľa pravidiel zo súboru \verb|config.txt| alebo povelmi odoslaných 
po zapnutí. Ukladať vzorky na externé médium nie je vždy žiaduce, preto bude napájanie spínané cez pin mikrokontroléra. Vyšší 
prúdový odber než dodá výstup a požadované napätie rovné s napájacím napätím dosky vyžaduje umiestnenie tranzistora riadeného 
poľom N-kanál \emph{BTS117} na premostenie riadiaceho signálu podľa obr. \ref{schematics:fet}.


\section{Architektúra systému}
Celková skladba komponentov systému pozostáva zo súčastí pôsobiacich na mikrokontroléri ESP32
interagujúcimi s akcelerometrom a zapisovačom cez lokálne sériové zbernice. Výstupy uspôsobenia a zoskupenia vektorov akcelerácie sú 
odosielané do počítačovej siete prostredníctvom prístupového bodu WiFi aplikačným protokolom MQTT (Message Queue Transport Telemetry)
na server poskytujúci službu MQTT broker správ Eclipse Mosquitto. Odtiaľ sú odoberané témy prešírené klientom metódou publish -- subcribe.
Diagram na obr. \ref{uml:component} zachytáva náhľad na poskladanie komponentov systému. 

\begin{figure}[h]
	\centering
	\includegraphics[width=\textwidth]{figures/design/components.png}
	\caption{Diagram komponentov}
	\label{uml:component}
\end{figure}

Odčítavanie úrovní z akcelerometra zabezpečuje časovač vzorkovania, ktorý si v pravidelných intervaloch pýta aktuálne hodnoty rozhraním 
periférneho adaptéra pre prístup ku SPI zbernici. Po vypršaní vzorkovacej periódy je zavolaná obsluha prerušenia, ktorá odblokuje vlákno 
úlohy na synchrónne načítanie okamžitého vektora zrýchlenia konvertovaného z číslicovej úrovne prevodníka na SI jednotku $m/s^2$. 
Postup vzorkovania ilustruje sekvenčný diagram na obr. \ref{uml:sequence}.

Zachytené hodnoty sú preposlané do nezávislých thread-safe front analyzátora signálu pre individuálne priestorové osi akcelerácie. 
V prípade nutnosti zachytávať ucelený priebeh veličiny  sa vektor umiestni do frontu ku vláknu loggera. Postup vzorkovania 
ilustruje sekvenčný diagram na obr. \ref{uml:sequence}. Pipeline spracovania signálu rozdeľuje vzorky do prelínajúcich sa posuvných okien, 
počíta z nich štatistiky a vyhľadáva udalosti vo frekvenčnom spektre. Nespracované hodnoty sú podľa potreby ukladané na pamäťovú kartu. 

Zmenu a uchovanie parametrov pipelinov pre jednotlivé bloky spracovania zaobstaráva správa konfigurácie. Modifikované nastavenia sú medzi 
spusteniami zachované vo vyhradenej partícií nevolatilnej flash pamäte na záznamy dvojíc v asociatívnej štruktúre. Predvolené správanie 
načítané pre nedostupnosť stavu z flash úložiska určujú konštanty v programovej pamäti.

Binárny serializačný formát Message Pack zaobaluje vzorky, udalosti a konfigurácia posielané na rozličné MQTT témy. Vychádza z 
formátu JSON (JavaScript Object Notation), ale narozdiel od neho sa sústredí na efektívne kódovanie dátových typov. Namiesto prevodu 
číselných údajov do znakového kódovania, napr. Unicode, ponecháva ich pôvodnú binárnu reprezentáciu s špecifikovanou bytovou značkou 
určujúcou typ. Hodnoty vyjadriteľné menším počtom bajtov reprezentuje dokonca v kratšom tvare než podmieňuje ich celkový rozsah.
Rovnako v zoznamoch a slovníkov sa  zaobchádza bez oddelovacích znakov, ktoré nahrádza informáciou o počte údajov. Dĺžka 
položky predchádzajúca zloženými atribútom uľahčuje následné parsovanie.

\begin{figure}[h]
	\centering
	\includegraphics[width=\textwidth]{figures/design/tasks.png}
	\caption{Sekvenčný diagram vzorkovania signálu}
	\label{uml:sequence}
\end{figure}

\section{Funkčné bloky dátovej pipeline}

\begin{figure}[h]
	\centering
	\includegraphics[width=\textwidth]{figures/design/pipeline.png}
	\caption{Diagram aktivít navrhovaného systému na spracovanie dát zo senzora}
\end{figure}


\begin{figure}[h]
	\centering
	\includegraphics[width=\textwidth]{figures/design/configuration.png}
	\caption{Diagram aktivít konfigurácie}
\end{figure}

Podľa pravidlami aktivovanými modulmi bude súbežne vykonateľných 5 ciest spracovania. 
Frekvenčná analýza bude pozostávať z krokov oknovania signálu  zvoliteľnou oknovou funkciou nastaviteľnej veľkosti,
následne sa signál transformuje do frekvenčnej domény,
kde dôjde k vyhladeniu spektra kĺzavým priemerom alebo Welchovou metódou a nakoniec sa podľa prítomnosti špičiek
vo frekvenčnom vedierku vytvorí udalosť na začiatku a na konci súvislého časového pôsobenia. Časová analýza
bude detegovať náhle zmeny v priebehu signálu na základe štatistík polohy, rozptýlenosti a tvaru
distribúcie vzhľadom na predošlé okná. Okrem toho sa môžu po podvzorkovaní odosielať alebo ukladať ,,surové dáta''
hodnôt veličiny v čase alebo frekvenčných vedierok. Numerická kvadratúra korigovaná obálkami umožní extrahovať zo
snímaného zrýchlenia odhad o rýchlosti a polohe.

Diagram aktivít na obr. \ref{fig:design}
vizualizuje navrhovaný beh činností na zariadení.

\clearpage


\section{Prúdový algoritmus detekcie zmien frekvencií}
Nevidí celý prúd naraz ani ho nemôže celý uložiť. Welchove priemerovanie vyžaduje veľa pamäte pre dlhšie okno.
Ale musíme dosiahnuť odšumenie detegovanie špičiek a zároveň posielať cez sieť menej ako nespracované
frekvenčné vedierka. Detegovanie anomálií, resp. automatické upozornenie na prevládajúce zložky.

\begin{figure}[h]
\centering
\begin{subfigure}[b]{0.8\textwidth}
    \centering
    \includegraphics[width=\textwidth]{figures/design/event-detection-min-duration.png}
    \caption{Minimálne trvanie udalosti $t_{\min} = 2$}
\end{subfigure}
\begin{subfigure}[b]{0.8\textwidth}
    \centering
    \includegraphics[width=\textwidth]{figures/design/event-detection-time-proximity.png}
    \caption{Maximálna vzdialenosť špičiek $t_{\Delta} = 1$ v čase}
\end{subfigure}
\caption{Parametre algoritmu na detekciu udalostí. $t$ je poradové číslo okna, $f$ je stav detegovanie špičky
vo frekvenčnom vedierku, $t_d$ je trvanie udalosti (duration), $t_p$ je počet okien do minulosti naposledy videnej špičky (past)}
\end{figure}

\begin{algorithm}[h]
\caption{Detektor zmeny frekvenčnej zložky}
\begin{algorithmic}[1]
\Require{$event$, $bin$, $t$, $t_{min}$, $t_{\Delta}$}
\If {V predošlom okne $t - 1$ bola emitovaná udalosť Koniec}
	\State Vynuluj udalosť: $event$: $duration \gets amplitude \gets 0$, $lastSeen \gets -1$
\EndIf

\If {$\mathrm{IsPeak}(bin)$}  
	\State $link \gets \max\{1, event.lastSeen + 1\}$
	\If {$event.duration < t_{min} \leq event.duration + link$}
		\State $event.start \gets t - event.duration - link + 1$
		\State \textbf{Emituj udalosť Štart} výskytu frekvencie podľa $event$
	\EndIf
	
	\State \textbf{Inkrementuj} $event.duration$ \textbf{o} $link$
	\State \textbf{Inkrementuj} $event.amplitude$ \textbf{o} $(bin - event.amplitude)\;/\;event.duration$
	\State $event.lastSeen \gets 0$

\ElsIf {$event.lastSeen \geq 0$}
	\State \textbf{Inkrementuj} $event.lastSeen$ \textbf{o}  $1$

	\If {$events.lastSeen > t_{\Delta}$}
        	\If {$event.duration \geq t_{min}$}
        		\State \textbf{Emituj udalosť Koniec} výskytu frekvencie podľa $event$
        	\EndIf
        \Else
        	\State Vynuluj udalosť: $event$: $duration \gets amplitude \gets 0$, $lastSeen \gets -1$
        \EndIf
\EndIf
\end{algorithmic}
\label{algo:event-detector}
\end{algorithm}

\section{Zber vibrácií z premávky}
V časti motora alebo nad nápravou z mestskej hromadnej dopravy. 500Hz, rozlíšní 2g z v režime zapisovania na SD kartu.
Predošlé prieskumné merania na akcelerometri smarfónu ukázali rozsah do +/-$8 m/s^2$do 1g najčastejšie do $2 - 3 m/s^2$

Pre LSB formát je tesne pod teoretickou hranicou rýchlosti UART (115200 baud). 1 znak = 8b + štart + stop = 10bitov.
Rozsah hodnôt signed int16 (5-ciferné číslo a znamienko na jednu súradnicu). Riadok mal teda aj s medzerami a novým riadkom max.
21 znakov na 3D vzorku. 21 znakov je 210 bitov. Teoretická max. vzorkovacia frekvencia je 548 Hz.

Realistické vzhľadom k prevážanému obsahu

\section{Generátor frekvenčného spektra}
Opis políčok konfigurácie a postup generovania fade-in/out sinsoid a náhodné vkladanie do signálu

\begin{figure}[h]
   \centering
    \includegraphics[width=0.8\textwidth]{figures/verification/fade-in-sinusoid.png}
   \caption{Základný tón formujúci syntetický signál}
\end{figure} 



\chapter{Zhodnotenie}
Hlavné výsledky práce, prípadne porovnanie s inými prístupmi, možné smery ďalšieho rozvíjania.
Tu sa musí presne špecifikovať, čo je pôvodné a čo riešiteľ prebral.
\emptypage


\pagenumbering{gobble}
\nocite{*}
\printbibliography[title={Literatúra}]
%\emptypage

% no page numbers for appendicies
\addtocontents{toc}{\protect\setcounter{tocdepth}{0}}
\addtocontents{toc}{\cftpagenumbersoff{chapter}}
\appendix
\titleformat{\chapter}{\normalfont\huge\bf}{Príloha \thechapter:}{1em}{}


% Príloha
\setcounter{figure}{0}
%\setcounter{listing}{0}
\chapter{Technická dokumentácia}
\pagenumbering{arabic}
\renewcommand*{\thepage}{A-\arabic{page}}

Prílohy dopĺňajú hlavnú časť práce. Obsahujú napríklad podrobné informácie k jednotlivým
etapám riešenia projektu. Typicky sa tu uvádza aj podstatná časť technickej dokumentácie.
Pozor, prílohy nesmú obsahovať také informácie, ktoré sú pre pochopenie práce kľúčové. Tie
musí obsahovať hlavná časť práce, ktorá musí byť úplná, celistvá.

Súčasťou príloh nie je len textový obsah, ale aj ďalšie artefakty, ktoré sú výsledkom projektu,
napr. počítačový kód, dátové vzorky, vedecký článok či plagát. Zvláštnu pozornosť venujte tým
artefaktom, ktoré sú potrebné pre replikovateľnosť postupov opisovaných v práci (napr. aby
mohol oponent pri vyhodnocovaní práce zopakovať uvádzané postupy a prísť k rovnakým
záverom). 

Digitálne artefakty sa prikladajú na elektronickom médiu. K akémukoľvek
digitálnemu obsahu treba uviesť v dokumente priebežnej či záverečnej správy
bakalárskej/diplomovej práce primeraný textový opis, preto nezabudnite digitálne médium
zdokumentovať. Prinajmenšom medzi prílohy zaraďte kapitolu "Obsah elektronického média".
Na prílohy sa nezabudnite z hlavnej časti práce primerane odkazovať.

Obsah technickej dokumentácie závisí od povahy riešeného problému. Uvádza sa technická dokumentácia k systému (počítačový, softvérový), ktorý bol vytvorený v rámci riešenia projektu (ak sa toto v zadaní požadovalo). Samotný obsah a rozsah závisí aj od účelu vytvoreného systému (produkt, experimentovanie a pod.)
V prípade softvérového systému technická dokumentácia spravidla obsahuje časti v náväznosti na etapy tvorby softvérového systému:
\begin{itemize}
	\item dokumentáciu k etape špecifikácie požiadaviek
    \item dokumentáciu k etape návrhu projektu
    \item dokumentáciu k implementácii
    \item v prípade, že súčasťou riešenia sú programy, dokumentáciu k implementácii tvoria zdrojové texty programov
    \item v prípade, že súčasťou riešenia je návrh zariadenia, dokumentáciu k implementácii tvorí technická dokumentácia (schémy zapojenia, návrh dosiek plošných spojov, schémy rozmiestnenia súčiastok, zoznam použitých súčiastok, opis konektorov atď.)
    \item dokumentáciu k overeniu riešenia
    \item dokumentáciu k používaniu a údržbe (návody na použitie a údržbu projektu)
\end{itemize}


% Harmonogram práce
\thispagestyle{empty}
\chapter{Harmonogram práce}
\pagenumbering{arabic}
\renewcommand*{\thepage}{B-\arabic{page}}
\section{Zimný semester}
\section{Letný semester}

% Digitálne médium
\thispagestyle{empty}

\chapter{Obsah digitálneho média}
\pagenumbering{arabic}
\renewcommand*{\thepage}{C-\arabic{page}}
\par Evidenčné číslo práce v informačnom systéme: \RegNo
\par Obsah digitálnej časti práce (archív ZIP):
\par Názov odovzdaného archívu: ...zip

\end{document}