\documentclass[12pt, a4paper, twoside, openright, slovak]{book}

\usepackage[slovak]{babel}
\usepackage[utf8]{inputenc}
\usepackage[T1]{fontenc}
\usepackage[top=2.5cm,
	bottom=2.5cm,
	right=3.2cm,
	left=3.2cm
]{geometry}

\usepackage{subcaption}
\usepackage{hyperref}
\usepackage{enumitem}
\usepackage{tabularx}
\usepackage{afterpage}
\usepackage{multirow}
\usepackage{amsfonts}
\usepackage{amssymb}
\usepackage{listings}
\usepackage{titlesec}
\usepackage{setspace}
\usepackage{fancyhdr}
\usepackage{fancyvrb}
\usepackage[fleqn]{amsmath}
\usepackage{pdfpages}
\usepackage{nccmath}
\usepackage{tocloft}
\usepackage{csquotes}
\usepackage{diagbox}
\usepackage{ifthen}
\usepackage{algorithm}
\usepackage{algpseudocode}


\usepackage{nomencl}
\usepackage{makeidx}
\usepackage{expl3}
\usepackage{etoolbox}
\preto\tabular{\shorthandoff{-}}

\usepackage[style=iso-numeric, backend=biber]{biblatex}
\addbibresource{literature.bib}

% Zoznam skratiek
\makenomenclature
\renewcommand{\nomname}{Zoznam skratiek a pojmov}

% Rovnice
\newcommand{\listequationsname}{Zoznam rovníc}
\newlistof{myequations}{equ}{\listequationsname}
\newcommand{\myequations}[1]{
\addcontentsline{equ}{myequations}{\protect\numberline{\theequation}\quad #1}}

% Algoritmy
\makeatletter
\renewcommand*{\ALG@name}{Algoritmus}
\renewcommand{\listalgorithmname}{Zoznam algoritmov}
\algrenewcommand\algorithmicrequire{\textbf{Vstupy:}}
\algrenewcommand\algorithmicensure{\textbf{Výstup:}}
\makeatother

% Číslo kapitoly na rovnakom riadku ako názov
\titleformat{\chapter}{\normalfont\huge\bf}{\thechapter}{1em}{}

\raggedbottom
\newcommand{\emptypage}{\newpage\thispagestyle{empty}\mbox{}\newpage}
\newcommand{\signaturespace}[2]{
  \begingroup
  \renewcommand{\arraystretch}{0}
  \begin{tabular}[t]{cc}
  \hspace*{0pt}
  \cleaders\hbox{\kern.6pt.\kern.6pt}\hskip#1\relax
  \hspace*{0pt}
  \\[0.5cm]
  #2
  \end{tabular}
  \endgroup
}

\pagestyle{fancy}
\fancyhf{}  % clear all header and footers
\fancyhead[LE]{\leftmark}
\fancyhead[RO]{\rightmark}
\fancyfoot[LE, RO]{\thepage}

\fancypagestyle{plain}{
  \fancyhf{}
  \renewcommand{\headrulewidth}{0pt}
  \fancyhf[lef,rof]{\thepage}
}

\setlength{\headheight}{16pt}

\renewcommand{\ttdefault}{pcr}
\lstdefinestyle{cstyle}{
    language=C,
	basicstyle=\linespread{1.1}\ttfamily\footnotesize,
    numbers=left,
    numberstyle=\tiny,
    frame=single,
    tabsize=4,
    captionpos=b,
    breaklines=true,
    texcl=true,
	numbersep=8pt,
	framexleftmargin=15pt,
	xleftmargin=5ex,
    xrightmargin=3.4pt,
	morekeywords = {uint8_t,uint16_t,int16_t,uint32_t,int32_t,bool}
}
\lstdefinestyle{docs}{
    language=C,
	basicstyle=\linespread{1.1}\ttfamily\small\bfseries,
    tabsize=4,
    breaklines=true,
    belowskip=0pt
}
\lstdefinestyle{messages}{
	basicstyle=\linespread{1.1}\ttfamily\small,
    tabsize=4,
    breaklines=true,
    belowskip=0pt
}

\lstdefinestyle{experiments}{
	basicstyle=\linespread{1.1}\ttfamily\small,
    tabsize=4,
    breaklines=true,
    belowskip=0pt,
    numbers=left, 
    numberstyle=\tiny,
    framexleftmargin=10pt,
	xleftmargin=5ex,
}
\lstdefinestyle{implementation}{
	basicstyle=\linespread{1.1}\ttfamily\small,
    tabsize=4,
    breaklines=true,
    frame=single
}
\renewcommand{\lstlistingname}{Zdrojový kód}

\setstretch{1.5}
\newcommand{\University}[0] {Slovenská technická univerzita v Bratislave}
\newcommand{\UniversityEN}[0] {Slovak University of Technology Bratislava}
\newcommand{\Faculty}[0] {Fakulta informatiky a informačných technológií}
\newcommand{\FacultyEN}[0] {Faculty of Informatics and Information Technologies}
\newcommand{\Thesis}[0] {Bakalárska práca}
\newcommand{\ThesisEN}[0] {Bachelor's Thesis}
\newcommand{\Title}[0] {Spracovanie dát generovaných senzorovou IoT sieťou}
\newcommand{\TitleEN}[0] {Data Processing for Sensor IoT Network}
\newcommand{\Author}[0] {Miroslav Hájek}
\newcommand{\Supervisor}[0] {Ing. Marcel Baláž, PhD.}
\newcommand{\PedagogicalSupervisor}[0] {Ing. Jakub Findura}
\newcommand{\SupervisorEN}[0] {Dr. Marcel Baláž}
\newcommand{\PedagogicalSupervisorEN}[0] {Jakub Findura}
\newcommand{\RegNo}[0] {FIIT-5212-102927}
\newcommand{\Date}[0] {Máj 2022}
\newcommand{\DateEN}[0] {2022, May}
\newcommand{\StudyProgramme}[0] {Informatika}
\newcommand{\StudyProgrammeEN}[0] {Informatics}
\newcommand{\StudyField}[0] {Informatika}
\newcommand{\Institute}[0] {Ústav počítačového inžinierstva a aplikovanej informatiky}
\newcommand{\SignPlace}[0] {V Bratislave, }
\newcommand{\SignDate}[0] {16.5.2022}


\begin{document}
\nomenclature{\textbf{DOF}}{Degree of Freedom (stupeň voľnosti mechanického systému)}
\nomenclature{\textbf{IoT}}{Internet of Things (internet vecí)}
\nomenclature{\textbf{MEMS}}{Micro-Electro-Mechanical Systems (mikromechanický systém)}
\nomenclature{\textbf{LSB}}{Least significant bit (najmenej významový bit)}
\nomenclature{\textbf{ODR}}{Output data rate (výstupný dátový tok)}
\nomenclature{\textbf{A/D}}{Analógový na digitálny}
\nomenclature{\textbf{$f_s$}}{Vzorkovacia frekvencia}
\nomenclature{\textbf{$g$}}{Tiažové zrýchlenie (1 $g$ = 9,80665 $m/s^2$)}
\nomenclature{\textbf{SPI}}{Serial Peripheral Interface (synchrónne sériové periférne rozhranie)}
\nomenclature{\textbf{UART}}{Universal asynchronous receiver-transmitter (zbernica asynchrónneho sériového prenosu)}
\nomenclature{\textbf{I$^\mathrm{2}$C}}{Inter-Integrated Circuit (dvojvodičová synchrónna sériová zbernica)}
\nomenclature{\textbf{FET}}{Field-effect transistor (tranzistor riadený poľom)}
\nomenclature{\textbf{MQTT}}{Message Queuing Telemetry Transport (aplikačný protokol na prenos telemetrických údajov cez fronty správ)}
\nomenclature{\textbf{FFT}}{Fast Fourier Transform (algoritmus rýchlej Fourierovej transformácie)}
\nomenclature{\textbf{DFT}}{Discrete Fourier Transform (diskrétna Fourierová transformácia)}
\nomenclature{\textbf{DCT}}{Discrete Cosine Transform (diskrétna kosínusová transformácia)}
\nomenclature{\textbf{SDK}}{Software development kit (nástroje na vývoj softvéru pre špecifickú platformu)}
\nomenclature{\textbf{TCP}}{Transmission Control Protocol (transportný sieťový protokol riadenia prenosu)}
\nomenclature{\textbf{ISM pásma}}{Voľné pásma pre rádiové vysielanie v priemyselnom, vedeckom a zdravotníckom sektore}
\nomenclature{\textbf{Senzitivita}}{Sensitivity, True Positive Rate (TPR), Recall. (pravdepodobnosť pozitívneho testu byť skutočne pozitívnym)}
\nomenclature{\textbf{Špecifickosť}}{Specificity, True Negative Rate (TNR). (pravdepodobnosť negatívneho testu byť skutočne negatívnym)}
\nomenclature{\textbf{Precíznosť}}{Precision (tesnosť zhody medzi výsledkami meraní navzájom)}
\nomenclature{\textbf{Presnosť}}{Accuracy (blízkosť nameraných hodnôt ku pravdivej hodnote)}
\nomenclature{\textbf{Chybovosť}}{False positive rate (pravdepodobnosť výskytu falošného poplachu)}
\nomenclature{\textbf{MTU}}{Maximum transmission unit (najdlhší poslaný paket bez fragmentácie)}
\nomenclature{\textbf{Pipeline}}{Súbor prvkov spracovania údajov zapojených do série, kde výstup jedného prvku je vstupom ďalšieho prvku}
\nomenclature{\textbf{MQTT téma}}{MQTT topic (logické zoskupenie publikovaných správ so spoločným zameraním)}
\nomenclature{\textbf{MCU}}{Microcontroller unit (mikrokontrolér na jednom integrovanom obvode)}

% Obal -----------------------------------------------------------------------
\thispagestyle{empty}
{\centering
	\linespread{1.5}
	{\large \MakeUppercase{\University}}\par
	{\large \MakeUppercase{\Faculty}}\par
	\vspace{\bigskipamount}
	{\large \begin{flushleft}\MakeUppercase{\RegNo}\end{flushleft}}
	\vfill
	{\Large \MakeUppercase{\Title}}\par
	\vspace{1.5\bigskipamount}
	{\large \MakeUppercase{\Thesis}}\par
	\vfill
	{\large \MakeUppercase{2022 \hfill \Author}}
}
\emptypage

%  Hlavná časť -----------------------------------------------------------------------
\newgeometry{top=2.5cm, bottom=2.5cm, right=2.5cm, left=3.5cm}

% Titulný list
\pagenumbering{roman}
\thispagestyle{empty}
{\centering
	{\large \University}\par
	{\large \Faculty}\par
	\vspace{\medskipamount}
	\RegNo
	\vfill
	\textbf{\large \Author}\par
	\vspace{1.5\bigskipamount}
	\textbf{\Large \Title}\par
	\vspace{1.5\bigskipamount}
	{\large \Thesis}\par
	\vfill
}
\emptypage

% Zadanie
\thispagestyle{empty}
\includepdf[pages=1, scale=0.9]{zadanie}
\emptypage

\input{chapters/preliminaries/declaration}
\emptypage
% Poďakovanie
\thispagestyle{empty}
\vspace*{\fill}
\section*{Poďakovanie}
Chcel by som sa poďakovať vedúcemu práce Ing.~Marcelovi Balážovi,~PhD. za ústretovosť, mnohé cenné pripomienky a podnety k vylepšeniam, usmernenia pri vytýčení zamerania a povzbudenie ku tvorivému preskúmaniu problematiky. 

Za poskytnutie  senzorovej jednotky a za postrehy ku formálnej stránke vďačím Ing.~Lukášovi Doubravskému. 

Tiež ďakujem svojmu kolegovi Ing.~Michalovi Juranyimu, ktorý ma za roky spolupráce mnohému priučil o vývoji softvéru.
Veľmi si cením morálnu podporu popri štúdiu od rodičov a od najbližšieho okruhu spolužiakov -- kamarátov.
\vspace{3cm}
\emptypage
%Anotácia
\thispagestyle{empty}
\section*{Anotácia}
\University \\
\uppercase{\Faculty}
\vspace{-8pt}
{\setlength{\mathindent}{0cm}
\begin{align*}
&\text{Študijný program:} && \text{\StudyProgramme} \\
&\text{Autor:} && \text{\Author} \\
&\text{\Thesis:} && \text{\Title} \\
&\text{Vedúci bakalárskej práce:} && \text{\Supervisor} \\
&\text{Pedagogický vedúci:} && \text{\PedagogicalSupervisor} \\
&\text{\Date}
\end{align*}}
V bakalárskej práci sa zameriavame na spôsoby spracovania signálov z vibrácií pri preprave, zachytených
senzormi akcelerácie. Zámerom je extrakcia čŕt záujmu z prúdu vzoriek do udalostí, čím sa redukujú vymieňané
dáta v senzorovej sieti. Analytická časť sa venuje fyzikálnemu opisu a číslicovému meraniu zrýchlenia s MEMS akcelerometrom,
z ktorého je možné získať ostatné kinematické veličiny. 

V časovej doméne nahliadame na sledovaný dej ako stochastický
proces vyjadriteľný metrikami deskriptívnej štatistiky. Významné okamihy sa prejavujú náhlou zmenou oproti predošlému správaniu
alebo nadprahovou úrovňou, čo sa zachytáva algoritmami na detekciu špičiek, ktoré predstavujú formu binárneho klasifikátora.
Vibrácie obsahujú rôzne frekvenčné zložky separovateľné Fourierovou a kosínusovou transformáciou algoritmom rýchlej Fourierovej
transformácie vo viacerých obmenách za aplikovania oknových funkcií. 

Spomenuté sú tiež filtre s konečnou impulznou odozvou
zohrávajúce úlohu pri úprave zdrojového signálu pred ďalšími stupňami spracovania. Prihliadame na obmedzenia vyplývajúce
s nasadenia riešenia na zariadenia Internetu vecí v kontexte Edge computing architektúry.

\emptypage

%Anotácia EN
\thispagestyle{empty}
\section*{Annotation}
\UniversityEN \\
\uppercase{\FacultyEN}
\vspace{-8pt}
{\setlength{\mathindent}{0cm}
\begin{align*}
&\text{Degree course:} && \text{\StudyProgrammeEN} \\
&\text{Author:} && \text{\Author} \\
&\text{\ThesisEN:} && \text{\TitleEN} \\
&\text{Supervisor:} && \text{\SupervisorEN} \\
&\text{Departmental advisor:} && \text{\PedagogicalSupervisorEN} \\
&\text{\DateEN}
\end{align*}}
In the bachelor thesis we focus on signal processing of vibrations during transport, captured
using acceleration sensors. The intention is to extract features of interest from the stream of samples into events, 
thereby reducing data exchange in the sensor network. The analytical part deals with the physical description of acceleration
and its digital measurement with the MEMS accelerometer, from which it is possible to obtain other kinematic quantities.

In the time domain, we look at the observed phenomenon as a stochastic process expressed by various descriptive statistics. 
Significant moments are manifested by a sudden change from the previous behavior or levels above the threshold
noticed by peak detection algorithms, which are a form of binary classifier. The vibrations contain different frequency 
components separable by Fourier and cosine transform by the Fast Fourier transform algorithm in several variants with the
application of window functions.

Finite impulse response filters are also mentioned as playing a role in adjusting the source signal before the next stages of 
the pipeline. We take into account the limitations of deployment on the Internet of Things devices in the context of the Edge 
computing architecture.

\emptypage 

% Obsah
\renewcommand{\contentsname}{Obsah}
\pagestyle{empty}
\tableofcontents{}
\listoffigures
\listofmyequations
{\let\clearpage\relax \printnomenclature}
\emptypage

\pagestyle{fancy}
% Kapitoly
\pagenumbering{arabic}

\chapter{Úvod}
Inteligentné senzorové systémy zariadení Internetu vecí (IoT) zaznamenávajú obrovskú kvantitu údajov z prostredia, kde
pôsobia. Prúdy vzoriek meraných veličín majú sami o sebe nízku informačnú hodnotu a zbytočne zaťažujú
prenosové pásmo komunikačných kanálov a kapacitu úložísk. Monitorovanie širokého rozsahu kladie požiadavky
na čo najmenšie výrobné náklady senzorových jednotiek a dlhodobú výdrž pri napájaní z batérií za minimálnej údržby.
Existuje preto potreba získané dáta spracovať do istej miery už v blízkosti ich zdroja, aby došlo k efektívnemu
využitiu dostupných prostriedkov.

Zameriame sa na sledovanie vibrácií a dolovanie čŕt záujmu z nich. Význam a dôležitosť sledovania vibrácií spočíva v ich v
výskyte u každého mechanického zariadenia pohybom jednotlivých súčiastok a trením v ložiskách. Ich nadmerná prítomnosť
býva spôsobená opotrebením dielov stroja a dôsledkom technických defektov. Ďalšou oblasťou hojnej prítomnosti vibrácií
je preprava osôb alebo tovaru, kde ich zapríčiňujú nerovnosti povrchu vozovky alebo koľaje v bode styku s kolesami, či aparát
ovplyvňujúci pohyb vozidla, čiže spaľovací alebo elektrický motor a brzdový systém.

Detekciou nežiaducich vibrácií v preprave sa dokáže zabezpečiť bezpečnosť pasažierov včasnou výmenou súčiastky,
ktorá by ovplyvnila prevádzkyschopnosť v kritických momentoch, a predísť nenávratnému poškodeniu krehkých materiálov,
znehodnoteniu reaktívnych substancií, či ich aktivácii v prípade výbušnín a pyrotechniky. Vibrácie sú súčasťou
nebezpečných prírodných úkazov a ich správna identifikácia má za následok varovania pre evakuáciu obyvateľstva
v oblasti postihnutej zemetrasením, či erupciou sopky, vedúcimi k ohrozenia zdravia osôb a poškodenia majetku.

V analýze problematiky sa zapodievame fyzikálnym modelom opisom vibrácií, od čoho sa odvíja metodika ich snímania
akcelerometrami typu MEMS a prevod do číslicovej podoby analógovo-digitálnym prevodníkom. Na spracovanie priebehu
signálu zrýchlenia sa pozrieme z troch hlavných hľadísk.

Integračné metódy umožňujú nadobudnúť odhad o relatívnej rýchlosti a polohe z akcelerácie. Na postupnosti pozorovaní
je možné nahliadať tiež v časovej doméne zužitkovaním základných agregačných a korelačných štatistík na odhalenie
náhlych zmien. Významne vyčlenené úrovne sú detegované
algoritmami na detekciu špičiek za rozličnej úspešnosti. Metódami transformácie do frekvenčnej oblasti sa
objavujú periodicky prítomné zložky, kde je opäť žiaduce upozorniť na momentálne prevládajúci spektrálny obsah.
Modely operujúce s vibračnými dátami by mali nasadené na IoT zariadenie kladúce svoje špecifické nároky a
obmedzenia.

Navrhneme kroky postupov na generovanie významných udalostí z nameranej akcelerácie. Po vyhodnotení úspešnosti
modelov v vlastných dátových sadách bude zámerom implementácia konfigurovateľnej senzorovej IoT jednotky.

\chapter{Analýza}

\section{Monitorovanie vibrácií a šoku}
Vibrácie sú periodickým kmitaním hmoty okolo rovnovážnej polohy vznikajúce excitáciou látky, ktorej je dodaná potenciálna energia, a zo zákona zachovania energie je následne premieňaná na kinetickú energiu. V realite dochádza pôsobením trenia k útlmu voľného oscilačného pohybu s časom  a pohybová energia sa uvoľňuje v podobe tepelnej alebo akustickej emisie do okolitého prostredia. Častejšie ako presné harmonické kmity sú pozorované náhodné vibrácie, ktorých vývoj nevieme dopredu predvídať. Naproti tomu šok, alebo aj prechodový jav, je náhle uvoľnenie kinetickej energie krátkeho trvania oproti prirodzenej oscilácii systému. 

Význam a dôležitosť sledovania vibrácií spočíva v ich výskyte u každého mechanického zariadenia a je zapríčinená pohybom jednotlivých súčiastok a trením v ložiskách. Ich nadmerná prítomnosť býva spôsobená opotrebením dielov stroja alebo nevyvážením rotačných častí, zakliesňovaním ozubených kolies, ako dôsledkoch iných technických defektov. V prevažnej väčšine prípadov ide o nežiaduci jav nakoľko zakladá zníženiu účinnosti so zvýšením hlučnosti ako vedľajšiemu produktu. 

Ďalšou oblasťou hojnej prítomnosti vibrácií je preprava osôb alebo tovaru cestnými a železničnými dopravnými prostriedkami, kde sú zapríčinené nerovnosťami povrchu vozovky alebo koľaje v bode styku s kolesami vozidla. Na zvýšenie ovládateľnosti vozidla a komfortu pasažierov sú kabíny odpružené od kolies tlmičmi. Lietadlá sú zasa pod vplyvom trenia vzduchu s trupom a krídlami konštrukcie, ktoré je ďalej zosilnené vzdušnými prúdmi a turbulenciami. 

Druhým významným faktorom podieľajúci sa na tvorbe vibrácii je aparát, ktorý uvádza vozidlo do pohybu, čiže hnací najčastejšie spaľovací, dieselový alebo elektrický motor, a mechanizmus, ktorý ho ho zastavuje, čím je brzdový systém. Jedná sa najmä o vplyv pohybu piestov, alebo rotora u elektrických vozidiel, a prenosu otáčavého pohybu motora cez oje hriadeľa na nápravy.

Detekciou nežiaducich vibrácií v preprave sa dokáže zabezpečiť aj bezpečnosť pasažierov včasnou výmenou súčiastky, ktorá by ovplyvnila prevádzkyschopnosť v kritických momentoch. Ich eliminácia dokáže predísť nenávratnému poškodeniu krehkých materiálov alebo znehodnoteniu reaktívnych substancií, či dokonca ich aktivácii v prípade výbušnín a pyrotechniky.

V neposlednom rade sú vibrácie súčasťou potenciálne nebezpečných prírodných úkazov a ich správna identifikácia má za následok varovania pre preventívnu evakuáciu obyvateľstva v oblasti, ktoré bude zasiahnutá zemetrasením, či erupciou sopky vedúcimi k ohrozenia zdravia osôb a poškodenia majetku.
 
\subsection{Meranie fyzikálnej veličiny akcelerácie}
Pohyb mechanického systému vystaveného vonkajším silám sa nazýva odozva, ktorej správanie opisuje zjednodušený model s jedným stupňom voľnosti kmitajúceho telesa s pružinou a tlmičom \cite{vibrations-shock}. Pri pôsobení vonkajšej sily $F$ na hmotu upevnenú na pružine vznikajú nútené vibrácie, ktoré ju vychylujú z rovnovážnej polohy. Uvedená sila je charakterizovaná druhým Newtonovým zákonom v tvare $F = ma$, kde $m$ je hmotnosť telesa a $a$ predstavuje zrýchlenie. 

V protismere pôsobí sila vyvolaná pružinou opísateľná cez vzťah $F_p = -kx$, kde $k$ je tuhosť pružiny ovplynená priemerom samotnej pružiny, priemerom drôtu z ktorého pozostáva, počtu a stúpania závitov, a $x$ je vychýlenie z rovnovážneho stavu. Fyzickým obmedzením telesa, ktorým je viazaný na pevnú podložku dochádza k takmer zaručenému návratu do rovnovážnej polohy, vynímajúc deformáciu, a to nám umožňuje merať intenzitu vibrácií cez zrýchlenie ťažidla. Pri použití trojosového akcelerometra, kedy sú evidované všetky tri priestorové súradnice časovo-premennej akcelerácie dostávame rovnicu vo vektorovom tvare: 
\begin{ceqn}\begin{align}
   \vec{a}(t) = \frac{\vec{F}(t)}{m}
\end{align}\end{ceqn}

Magnitúda akcelerácie s troma súradnicami je potom daná $L_2$ normou vektora:
\begin{ceqn}\begin{align}
   |a| = \sqrt{a_x^2 + a_y^2 + a_z^2}
\end{align}\end{ceqn}

\subsubsection{MEMS kapacitný akcelerometer}
Bežné inerciálne senzory na meranie zrýchlenia priamočiareho, ale aj rotačného pohybu (gyroskop), sa vyrábajú technológiou \emph{MEMS – mikromechanický systém}, kedy je celé zariadenie vrátane všetkých mechanických súčastí umiestnené na kremík procesom mikrovýroby vo viacerých vrstvách. Sila spôsobujúca zrýchlenie je potom meraná vychýlením vstavanej odpruženej hmoty vzhľadom na pevné elektródy, ktoré môžu byť usporiadané jednostranne alebo ako diferenčný pár \cite{mdof-mems-accelerometers}.

Pri diferenčnom páre spôsobí pohyb doštičky ťažidla medzi elektródami zmenu kapacít a ich rozdielom je možné zistiť aplikovanú silu a cez uvedený vzťah zrýchlenie. Na zvýšenie celkovej kapacity sa používa viacero párov elektród zapojených paralelne. Pred prevodom na číslicový signál musí napäťová úroveň zo senzora prejsť úpravou zahŕňajúcou nábojovocitlivý predzosilňovač, osovú demoduláciu a anti-aliasingové filtrovanie. Viacosové akcelerometre vyžadujú viaceré opísané štruktúry orientované kolmo na seba podľa počtu vyžadovaných stupňov voľnosti, pričom v skutočných senzoroch vždy existuje aspoň minimálna závislosť medzi osami rádovo najviac v jednotkách percent. Teplota ovplyvňuje citlivosť MEMS akcelerometrov len nepatrne, v stotinách percenta na stupeň Celzia.

Akcelerometre sa odlišujú v niekoľkých dôležitých vlastnostiach, ktoré zvyknú byť nastaviteľné vo výrobcom stanovenom rozsahu prípustných hodnôt s príslušnými toleranciami. \emph{Citlivosť} stanovuje najmenšiu rozlíšiteľnú zmenu v odčítanom napätí ku zmene externého pohybu, resp. zrýchlenia. Uvádza sa v jednotkách $mV/g$ (milivolt na tiažové zrýchlenie) pri analógovom výstupe, alebo $mg/\mathrm{LSB}$ (mili-g na najmenej významový bit). pri senzoroch so vstavaným analógovo-digitálnym prevodníkom. Jednotka $mg/LSB$ Vyjadruje o koľko sa zmení zrýchlenie keď zvýšime alebo ponížime binárne číslo na výstupe o jedna. Niekedy sa namiesto citlivosti uvádza mierka ako prevrátená hodnota citlivosti v $LSB/g$.

Tiažové zrýchlenie $g$ sa mierne líši podľa zemepisnej šírke, ale zaužívaný prepočet na jednotky SI je určený: $1 g = 9.80665 m/s^2$ \footnote{\url{https://physics.nist.gov/cgi-bin/cuu/Value?gn|search_for=acceleration}}. \emph{Dynamický rozsah} je uvádzaný v $g$ a hovorí o najmenšej a najväčšej rozlíšiteľnej hodnote zrýchlenia nad úrovňou ktorej už dochádza k skresleniu signálu orezaním špičiek.

S nevyhnutnými drobnými nepresnosťami výroby mikromechaniky je tzv. \emph{zero-g napätie} popisujúce odchýlku skutočného od ideálneho výstupu, keď na sústavu nepôsobí žiadne zrýchlenie. Za ideálnych okolností bez pohybu na vodorovnom povrchu namerajú osi $x$ a $y$ zrýchlenie $0g$, zatiaľčo na $z$ pôsobí $1g$. Očakávaním je nulová hodnota výstupného napätia a tým aj výstupného registra.

Nie často uvádzanou vlastnosťou býva \emph{frekvenčná odozva} senzora, ktorá určuje o koľko sa v rámci tolerancie odlišuje skutočná citlivosť od referenčnej pre zodpovedajúcu frekvenciu vibrácii.

% Čo je ODR?
\subsubsection{Analógovo-digitálny prevodník}
Spojitá napäťová úroveň transformuje analógovo-digitálny (A/D) prevodník pre spracovanie digitálnym systémom do množiny diskrétnych hodnôt. Vstupný signál najprv prechádza fázou vzorkovania, kedy sa vzorky zaznamenávajú v pravidelných intervaloch. Počet vzoriek odčítaných za sekundu je vyjadrený vzorkovacou frekvenciou $f_s$ v $Hz$. Časový rozdiel medzi vzorkami, nazývaný perióda vzorkovania, je prevrátenou hodnotou vzorkovacej frekvencie $T_s = \frac{1}{f_s}$. Pre presnú rekonštrukciu pásmovo obmedzeného signálu v hraniciach $[-f_{max}; f_{max}]$ je nevyhnuté podľa Nyquist-Shannonovej vety o vzorkovaní, aby vzorkovacia frekvencia bola najmenej dvojnásobkom maximálnej frekvencie snímaného signálu.
\begin{ceqn}\begin{align}
   f_s \geq 2 \cdot f_{max} 
\end{align}\end{ceqn} 

Každej vzorke je následne v procese kvantovania priradená diskrétna hodnota s konečným počtom $n$ bitov, ktorá je najbližšia možná ku skutočnej hladine analógového vstupu. Dochádza pritom k istému zaokrúhľovaniu z dôvodu nepresnosti vyjadrenia spojitého penza amplitúd diskrétnym číslom. Tento jav označujeme ako kvantizačný šum, ktorý je najviac polovicou z maximálnej rozlíšiteľnej zmeny signálu a trpia nim všetky existujúce A/D prevodníky. Rovnako tak sa u všetkých prevodníkov prejavuje aspoň nepatrná miera nelinearity výstupného kódu, chýbajúce kódy alebo ich nemonotónnosť.

Prevodníky integrované priamo s inerciálnymi jednotkami sa vyhotovujú v rozlíšeniach 12, 16 alebo 20 bitov. Umožňujú tak pripojiť akcelerometer rovno na sérové zbernice \emph{SPI} alebo \emph{I2C}. Všeobecne platí, že pri $n$ bitoch je k dispozícii $2^n$ rozličných čísel. Kódovaním v dvojkovom doplnku pre zachytenie záporných hodnôt sa uvažuje s intervalom $[-2^\frac{n}{2}; 2^\frac{n}{2} - 1]$. Napríklad pri 12-bitovom A/D prevodníku s referenčným napätím $3.3V$ je teoreticky najmenšia rozlíšiteľná zmena na najmenej významový bit $3.3V / 2^{12} = 0.81 mV$. Ak je rozhranie senzora priamo vybavené analógovým výstupom nič nebráni v použití presnejšieho prevodníka, napriek tomu najmenší merateľný dielik je zdola stále ohraničený citlivosťou akcelerometra. 

Číslicovú hodnotu v dvojkovom doplnku získanú konverziou $\hat{x}$ je z dôvodu širšej zrozumiteľnosti žiaduce prepočítať na štandardné fyzikálne jednotky pre zrýchlenie, $a$ v $m/s^2$). $R$ prestavuje nastavený dynamický rozsah v jednotkách $g$ a $n$ je počet bitov A/D prevodníka.
\begin{ceqn}\begin{align}
   a = \hat{x} \cdot \frac{R \cdot g}{2^\frac{n}{2}}
\end{align}\end{ceqn}

Ako bolo spomenuté ohľadom vlastností MEMS akcelerometrov, presnejší prevod dosiahneme zužitkovaním deklarovanej citlivosti senzora pri danom dynamickom rozsahu $S_R$ udávaného v $mg/LSB$.
\begin{ceqn}\begin{align}
   a = \hat{x} \cdot \frac{S_R \cdot g}{1000}
\end{align}\end{ceqn}

A/D prevodníky existujú v rôznych prevedeniach podľa žiaducej rýchlosti prevodu, presnosti a zložitosti vyhotovenia. Integračný prevodník sa spolieha na meranie času potrebného na vybitie kondenzátora v zapojení s obvodom operačného zosilňovača. Z pomedzi typov prevodníkov je zďaleka najpomalší. Aproximačný prevodník pozostáva z obvodu vzorkuj a podrž (sample and hold) komparátora, D/A prevodníka a výstupného registra. Diskretizácia signálu prebieha binárnym vyhľadávaním od najvýznamnejšieho bitu, rozhodujúceho o $\frac{1}{2}$ celkového rozsahu, počiatočne nastaveného na jedničku. Po D/A prevode slova sú napätia porovnané komparátorom. Ak je vstupná úroveň nižšia, bit sa nastaví na nulu a pokračuje sa postupne nižšími bitmi s dopadom na $\frac{1}{4}$, $\frac{1}{8}$ plného rozsahu atď. Okrem toho sa hojne používa Sigma-Delta A/D prevodník. Vyžaduje 1-bitový D/A prevodník zapojený v spätnoväzobnej slučke,
kedy sa nadvzorkuje vstupný signál v modulátore, poslaný je do digitálneho filtra, kde je vyhladený a reťaz zakončuje decimátor.

Na ilustráciu uvádzame parametre najrozšírenejších typov akcelerometrov\cite{mems-accel-mechanical-vibrations}. MPU6050 umožňuje použiť rozsahy merania v rozpätiach: $\pm2g$, $\pm4g$, $\pm8g$, $\pm16g$, pričom všeobecne sa každý rozsah vyznačuje svojou citlivosťou. Zvolením menšieho rozsahu spôsobíme väčšiu citlivosť, ktoré sú pre MPU6050 približne $0.06\,g/LSB$ pri $\pm2g$ až po $0.49\,g/LSB$ pri $\pm16g$. Vyznačuje sa taktiež 16-bitovým A/D prevodníkom. LSM9DS1 disponuje rovnakými rozsahmi a podobnými citlivosťami, s výnimkou rozsahu $\pm 16g$ s citlivosťou $0.73\, g/LSB$. ADXL-345 sa vyznačuje opäť pri totožných rozsahoch merania citlivosťou $0.004\,g/LSB$ s rozlíšením 10 bitov alebo až 13-bitov pri najväčšom rozsahu. Vyrábajú sa tiež akcelerometre s väčšími dynamickými rozsahmi napríklad ADXL357 so škálami $\pm 10\,g$, $\pm 20\,g$ a $\pm 40\,g$ s citlivosťou $0,02\,g/LSB$ po $0,08\,g/LSB$ a rozlíšením 20 bitov. Naproti tomu príbuzný akcelerometer ADXL357 nedisponuje žiadnym prevodníkom a má deklarovanú citlivosť 20 - 80 $mV/g$.

% Adaptívne vzorkovanie - preskočenie vzorky, keď je predpoklad, že na základe existujúcich vzoriek vieme dostatočne dobre odhadnúť budúce čítania. \cite{adaptive-sampling}. Efekt zberu až po prekročení prahu intenzity akcelerácie na signál.

\subsection{Odvodzovanie rýchlosti a polohy zo zrýchlenia}
fyzikálne vzťahy pre polohu, rýchlosť a zrýchlenie, numerická derivácia a integrácia (obdĺžníkové, lichobežníkové, Simposonovo pravidlo) \cite{integration-acceleration-envelopes}

\section{Metódy analýzy signálu v časovej doméne}
časový rad, okná
\cite{time-series-analysis} \cite{practical-time-series} \cite{generalized-esd} \cite{twitter-esd}, 
online algoritmus \cite{online-anomaly-detection}, 
požiadavky na efektívne algoritmy
	

\subsection{Číselné charakteristiky štatistického rozdelenia}
bodové odhady momentov. výberový priemer = stredná hodnota, výberový rozptyl (Welfordov online algoritmus), šikmosť, špicatosť, kvantily, normálne rozdelenie pravdepodobnosti

%\subsection{Prúdové algoritmy}
%stream algorithm models, Data Stream Model - cash register, turnstile, time series, window model
%Odhad momentov, Rátanie frekvencií, Zistenie či sa dopyt už vyskytol
%Count–min sketch (\url{https://florian.github.io/count-min-sketch/}) - probabilistická dátová štruktúra

\subsection{Algoritmy na rozpoznávanie špičiek}
lokálne minimá a maximá extrémy - cez prvú a druhú deriváciu,
topografická prominencia a izolácia - relatívna výška vrchola / extrému
vyhladzovanie: mean filter, derivative filter (diskrétny derivačný operátor) - sobel filter na , Savitzky–Golay filter
Vzájomná korelácia - jadra vrchola a signálu
detect peaks in a signal and to measure their positions, heights, widths, and/or areas

\cite{spectrometry-peak-detection}
\cite{survey-peaks-valleys}
\cite{peek-mountaineer-method}
\cite{ecg-r-peak-detection}
\cite{ampd-algorithm}

\subsubsection{Jednoduchý detektor vrcholov}
Preskočí špičky s absolútnou magnitúdou menšou ako `height`. Vo vnútornom cykle zisťuje či je bod vyššie ako všetkých `k` susedov doprava a doľava s toleranciou `epsilon`.

\begin{equation}
\forall c \in \{y_{i-k}, ... y_{i+k}\} - \{y_i\}; \; y_i - c \geq -\epsilon
\end{equation}

\begin{lstlisting}[language=Python]
def find_peaks(y: list, k: int, epsilon: float=0, height: float=None) -> list:
    spikes = []
    
    for a in range(k, y.size - k): 
        if height is not None and abs(y[a]) < height:
            continue
        locmax = True
        for b in range(-k, k):
            if b != 0 and y[a + b] - y[a] > epsilon:
                locmax = False
        if locmax:
            spikes.append(a)

    return spikes
\end{lstlisting}
\url{https://terpconnect.umd.edu/~toh/spectrum/PeakFindingandMeasurement.htm}

\subsubsection{Algoritmus Negative Zero-Crossing}
Nájdi vo n-krát vyhladenom signále body, že platí: $f'(x) = 0$, kde $f''(x) < 0$ a $|f''(x)| > \theta$. Pre odstránenie hraničných efektov použiť mód 'valid' s prísušným posunom hraníc polí.

Diskrétna verzia:
$$f'(x) = 0 \; \mathrm{(Dotyčnica)} \implies |y_{i+k} - y_{i-k}| \leq \epsilon \;\mathrm{(Sečnica)}$$
$$f''(x) < 0 \implies (y_{i+k} - y_i) - (y_i - y_{i-k}) < 0$$
$$|f''(x)| > \theta \implies |(y_{i+k} - y_i) - (y_i - y_{i-k})| > \theta$$

Parametre: 
\begin{itemize}
\item $\epsilon$: tolerancia pre nulovú deriváciu
\item $\theta$: strmosť druhej derivácie, čiže špicatosť vrchola
\item $k$: polovica dĺžka sečnice pre výpočet prvej derivácie
\item $n$: veľkosť konvolučného jadra
\item $smooth$: počet vyhladzovaní
\end{itemize}

\subsubsection{Algoritmus horského turistu}
Obsahuje pamäťový efekt pre dočasné zmeny a berie ich do úvahy ak lokálne záchvevy prekročia tolerovanú úroveň. Ak sa zmení 'slope' oproti predošlému kroku DeltaY, tak hneď neoznačí za zmenu medzi dolinou a vrcholom, ale až po prekročení nastavených prahov. Skutočnosť z reality: jama != údolie


\subsection{Online detekcia anomálií a odchýliek pozorovaní}
Outlier je pozorovanie, ktoré sa odchyluje, tak významne od ostatných pozorovaní, že vzbudzuje
podozrenie, že bolo vytvorené odlišným mechanizmom. Normálne dáta, Šum, Anomálie (slabé a silné odchýlky)
Výstupy algoritmov na detekciu výchyliek: 
\begin{itemize}
\itemsep0pt
\item Outlier skóre - miera vychýlenosti bodu
\item Binárne štítky - binárne štítky, označenie, či je bod anomália alebo nie.
\end{itemize}
Algoritmy na anomálie vytvárajú model normálnych vzorov v dátach a skóre vychýlenosti je dané deviáciou od týchto vzorov.
\cite{outlier-analysis} 
Hampel filter

\cite{survey-univariate-time-series} 
Množstvo prenášaných a spracovaných dát prevyšuje ľudskú schopnosť manuálneho prieskumu. Anomália je pozorovanie alebo postupnosť pozorovaní, ktoré sa významne odchyluje od distribúcie zvyšku dát. 
\begin{itemize}
	\item Bodové anomálie - $x_t$ je bodová anomália ak sa jeho hodnota významne odlišuje od všetkých 		
		bodov v intervale $ [x_{t-k}; x_{t+k}] $
	\item Kolektívne (disonanancie) - jednotlivé body nepredstavujú anomálne správanie, až ak vezmeme dlhšiu postupnosť môže byť označená za anomáliu.
	\item Kontextové výchylky / anomálie - body sú normálne v určitom kontexte, ale v inom anomálne.
\end{itemize}		

\cite{review-outlier-datection} \cite{anomaly-detection-algorithms}
\begin{equation}
|x - \hat{x}| \geq \theta 
\end{equation}

Bežiaci filter  
\cite{anomaly-detection-models}

\subsubsection{Metriky pre binárny klasifikátor}
Skóre anomálnosti, 	binárny klasifikátor, 
Falošne pozívny - proces je normálny, ale registrujeme neočakávané správanie, 
Falošne negatívny - proces je abnormálny, ale správanie prechádza bez povšimnutia
Matica zámen, Precision, Recall, 
ROC - kvalita binárneho klasifikátora v závislosi od prahu, AUC
\cite{wsn-outlier-detection-survey}

Z-score a Z-test: 
\begin{equation}
z = \frac{|x - \mu|}{\sigma} \approx \frac{|x - \bar{x}|}{S}
\end{equation}
Bežiaci priemerovací filter \cite{anomaly-detection-models}

Zisťovanie výchyliek testovaním štatistických hypotéz: 
generalized extreme Studentized deviate test \cite{generalized-esd} 

Median Absolute Deviate (robustná štatistika na určenie vychýlenosti od normálu)
\begin{equation}
MAD = \mathrm{median}(|X_i - \bar{X}|)
\end{equation}


\section{Frekvenčná a časovo-frekvenčná analýza signálu}
vlastnosti frekvenčného spektra, decibele, spektrogram, odstup od šumu (SNR), power spectrum density (dbFS), spektrálny analyzátor 

\subsection{Fourierová transformácia}
Diskrétna fourierová transformácia mapuje signál dĺžky $N$ do množiny $N$ diskrétnych frekvenčných komponentov. \cite{signal-processing}
\begin{equation}
X = \mathbf{W}x; W_{nk} = e^{-i\frac{2\pi}{N}nk} = W_N^{nk}
\end{equation}
Inverzná transformácia
\begin{equation}
x = \frac{1}{N}\mathbf{W}^H X
\end{equation}

Integrálne transformácie: Fourierová transformácia (CFT, DFT), Kosínusová transfomácia (MDCT), Vlnková transformácia (CWT) \cite{dct} \cite{casove-frekvencia-analyza-signalu}

\begin{equation}
T(n) = \int{f(t) K(t,x) \mathrm{dt}}
\end{equation}

\begin{equation}
\mathcal{F}: X(\omega) = \int_{-\infty}^{+\infty}{x(t) \cdot e^{-i\omega t} \mathrm{dt}}
\end{equation}

\subsection{Algoritmus FFT pre DFT a DCT}
Opis DIT radix-2 FFT algoritmus komplexných, reálny, pre MDCT \cite{fft-blackbox}

\begin{equation}
X(m) = \sum_{n = 0}^{N-1}{x(n) \cdot e^{-i2\pi n m / N}}
\end{equation}

Frekvenčné rozlíšenie
\begin{equation}
\Delta f = \frac{f_s}{N}
\end{equation}

\subsection{Oknové funkcie}
Gaborová transformácia, Prehľad okien a ich transformácií (sinc), Efekt oknových funkcií na spectral leakage, výhodné percentá prekryvu FT 	\cite{understanding-dsp} \cite{spectral-density-estimation}
Priemerovanie a prekryv - Amplitude Flatness (AF), Power Flatness (PF), Overlap Correlation (OC)

\subsection{Filtre s konečnou impulznou odozvou}
Roziel medzi FIR a IIR, Dolná pripusť, pásmová priepusť, horná pripusť,
 Konvolúcia a konvolučné jadro, konvolučná veta, účel: identifikácia prítomnosti známej frekvencie v signále akcelerácie
\begin{equation}
y(n) = \sum_{k=0}^{D_y}{x(k) \cdot h(n-k)} = x(n) * h(n)
\end{equation}

Prenosová funkcia
\begin{equation}
H(\Omega) = \frac{Y(\Omega)}{X(\Omega)}
\end{equation}	

\subsubsection{Detektor obálok}
\footnote{\url{https://www.mathworks.com/help/dsp/ug/envelope-detection.html}}
\footnote{\url{https://www.dsprelated.com/showarticle/938.php}}

\section{Senzorová sieť}
Nízko-energetické zariadenia komunikujúce cez odľahčené sieťové protokoly so snahou spracovania v reálnom čase a ponechaním najdôležitejších informácii dolovaním z veľkého množstva zdrojových dát. Cloud / Fog comuting. Sink a Edge nodes

Vlastnosti senzorovej siete
\begin{itemize}
\itemsep0em 
\item Autokonfigurácia senzora - reakcia na zmeny v sieti a prostredia pôsobenia
\item Škálovateľnosť - veľké množstvo senzorov so spoločným účelom a schopnosťou vzájomnej kooperácie a interoperability.
\item Odolnosť voči chybám - v prípade pridania alebo odobratia uzla budú spojenia bez prerušenia.
\item Energeticky efektívna komunikácia uzlov - s upravenými protokolmi štandardného sieťového zásobníka
\begin{itemize}
\itemsep0em 
\item Event-driven - stály zber dát a reakcia na náhle zmeny. posielajú údaje až po prekročený kritického prahu
\item Query-driven - zbierajú údaje iba po prijatí dopytu od používateľa
\item Time-driven - pravidelne odosielajú údaje sinku. vzorkovaciu frekvenciu volí sink
\end{itemize}
\end{itemize}
\cite{wsn-overview}

Spracovanie toku informácií (IFP - Information flow processing) - nástroj na včasné spracovanie dát ako tečie z periférií do centra systému. Snahou je ukladanie agregovaných štatistík, napr. detektor požiaru za použitia čidiel teploty a dymu nepotrebuje ukladať jednotlivé merania, lebo sú samo o sebe nepodstatné. Keď nastane varovná situácia, je potrebné aby tá obsahoval všetky údaje na lokalizáciu ohniska.

CEP - Complex event processing - spracúva toky udalosti zo zdrojov reálneho sveta na základe aplikovania aktívnych pravidiel stanovených správcami systému a poupraví toky do komplexnejšieho výstupu. Pravidlá sú v tvare Event-Condition-Action (ECA).
\begin{itemize}
\itemsep0em
\item Udalosť - definuje zdroje ako generátory udalostí
\item Podmienka - uvažuje ktorá časť udalosti bude braná do úvahy pri spracovaní, napr. môže ísť o prekročenie prahu
\item Akcia - aká sada úloh má byť vykonaná pri detekcii udalosti
\end{itemize}

Behové pravidlá sú spracúvané vo viacerých fázach
\begin{itemize}
\item Signalizácia - detekcia udalosti
\item Spustenie - asociácia udalosti so sadou pravidiel
\item Vyhodnotenie - vyhodnotenie podmienky
\item Plánovanie - stanovenie poradia vykonania
\item Vykonanie - vykonanie pravidlá
\end{itemize}
\cite{processing-information-flows}

\subsection{Senzorová jednotka}

Súčasti senzorovej jednotky:
\begin{itemize}
\item Zberná jednotka
\item Výpočtová jednotka
\item Komunikačná jednotka
\item Napájacia jednotka
\end{itemize}

Obmedzenia na senzorové uzly 
\begin{itemize}
\item Spotreba energie - energetická autonómia uzlov vo WSN umožňuje nasadzovanie zariadení do odľahlých miest pre využitie v inteligentných mestách alebo na účely ochrany prírody. životnosť senzorovej jednotky je ohraničená kapacitou batérie.
\item Dosah komunikácie - Senzory disponujú obmedzenou energiou na vysielanie a dosah je negatívne ovplyvnený silou signálu na anténe. Z toho vyplývajú aj nižšie prenosové rýchlosti.
\item Výpočtový výkon a úložisko - Nízka taktovacia frekvencia procesora v megaherzoch a veľkosti pracovných pamätí v stovkách kilobajoch alebo megabajtoch.
\end{itemize}
\cite{big-data-collection-wsn}


Za ideálnych okolností by sa mal online algoritmus učiť kontinuálne bez ukladania predošlých bodov a detekcií.
V rozhodnutiach algoritmu sú zahrnuté informácie o všetkých predošlých bodoch do terajšieho rozhodnutia. Mal by mať schopnosť sa adaptovať dynamickému prostrediu, v ktorom pôsobí. Bez nutnosti manuálnych úprav parametrov modelu. Zároveň je žiaduce minimalizovať falošné pozitíva a negatíva pri detekcii udalostí.

Kontinuálne spracovanie častokrát vyžaduje, aby boli dáta rozdelenie do rovnako dlhých blokov zvaných okná. Okno $\mathcal{W}_{\sigma, \pi}$ je funkciou morfujúca maticu vstupných dát $X$ do vektora $w = (W_1, ... , W_q)$  \cite{online-anomaly-detection}

	
\emptypage 


\chapter{Návrh riešenia}
V súlade so stanovenými kritériami navrhneme postupnosť krokov úpravy nameraného pohybu dopravného prostriedku. 
Zhotovenú konfigurovateľnú sústavu uplatníme na zariadení senzorovej jednotky za účelom ohlasovania udalostí 
o vybraných pravidelných rysoch signálu. Prihliadať sa bude viac na splnenie úloh v reálnom čase pri
dosiahnuteľnom výkone, v dostupnom pamäťovom priestore, ako na energetickú úsporu. Výmena údajov sa má 
odohrávať zrozumiteľným, širko podporovaným formátom, za obmedzenia nadbytočného sieťového prenosu 
v najvyššej možnej miere.


\section{Špecifikácia požiadaviek}
Zariadenie internetu vecí určené na zber a analýzu vibrácií bude realizovať nasledujúce funkčné požiadavky: 
\begin{itemize}[noitemsep,topsep=0pt]
\item Zber trojosovej akcelerácie s nastaviteľnou vzorkovacou frekvenciou a dynamickým rozsahom akcelerometra, v hraniciach danými 
obmedzeniami hardvéru, najmenej však intenzity vyskytujúcej sa pri preprave konvenčnými pozemnými motorovými vozidlami.
\item Spracovanie osí akcelerácie nezávisle s obmedzením výberu aktívnych osí.
\item Vzdialene realizovateľná zmena parametrov jednotlivých stupňov sústavy na úpravu akceleračného signálu v posuvných oknách.
\item Ukladanie nameranej akcelerácie na pamäťovú kartu s ohľadom na najvyššiu dosiahnuteľnú rýchlosť zápisu.
\item Identifikácia významných frekvencií zo vzoriek akcelerometra so zachytením ich trvania a amplitúdy podľa aktuálnej 
konfigurácie detekčných algoritmov.
\item Sumarizácia hodnôt akcelerácie po posuvných oknách do popisných štatistík.
\item Odosielanie zachytených udalostí cez spoľahlivé sieťové spojenie za dosiahnutia redukcie množstva dát oproti priamo odčítaným vzorkám. 
\item Poskytnutie možnosti odosielania výstupov z podstatných medzikrokov spracovania za účelom potenciálneho ponechania 
analýzy pre ďalšie úrovne senzorovej siete alebo skupinovej koordinácie údajov z viacerých uzlov. 
\end{itemize}
\bigskip

Z povahy okolností nasadenia firmvéru na relatívne zdrojovo oklieštené Edge zariadenie vyplývajú 
vymenované nie-funkčné požiadavky, prevažne na účinnosť a prenositeľnosť:
\begin{itemize}[noitemsep,topsep=0pt]
\item Firmvér sa zmestí do programovej pamäte s rezervou pre budúce rozširovanie funkcionality.
\item Ľubovoľný scenár spracovania musí prebehnúť v reálnom čase.
\item Odozva detekcie udalostí bude najkratšia možná.
\item Výmena údajov cez sieťové protokol v štandardizovanom formáte hierarchickej štruktúry za najmenšej uskutočniteľnej réžie.
\item Platformová závislosť sa obmedzí na nevyhnuté súčasti systému ako sú hardvérové ovládače a akcelerácia náročných výpočtov.
\end{itemize}

\section{Hardvér senzorovej jednotky}
Navrhované zariadenie bude postavené na mikrokontroléri ESP32 od Espressif, pretože pri nízkej obstarávacej cene
ponúka možnosť konektivity na 2,4 GHz s Wifi 802.11 b/g/n a Bluetooth 4.2. V porovnaní s podobnými zariadeniami disponuje 
nebývalým výpočtovým výkonom a kapacitou pamätí.

Konkrétne zvolíme dosku FireBeetle osadenú modulom \emph{ESP32-WROOM-32D} s typickým napájacím napätím 3,3 V a dvoj-jadrovým 
32-bitovým procesorom Xtensa s taktovacou frekvenciou od 80 do 240 MHz. Modul obsahuje až 520 kB zdieľanej SRAM 
pre inštrukcie a dáta. 

Použitý model akcelerometera je súčasťou MEMS inerciálnej meracej jednotky \emph{LSM9DS1} zabudovanej na doske
STEVAL-MKI159V1 slúžiacej na adaptáciu pre púzdro DIL24. Akcelerometer bude s mikrokontrolérom komunikovať cez
poloduplexnú SPI zbernicu s horným obmedzením frekvencie hodín na 8 MHz. Navyše budú zapojené aj vývody prerušení INT1 a INT2
pre upozornenie prekročenia určených prahových úrovní. Blokový diagram zapojenia je na schéme na obr. \ref{schematics:block}.

\begin{figure}[h]
\centering
\begin{subfigure}[b]{0.65\textwidth}
    \centering
    \includegraphics[width=\textwidth]{figures/design/block-circuit-diagram.png}
    \caption{Blokový diagram modulov}
       \label{schematics:block}
\end{subfigure}
\hfill
\begin{subfigure}[b]{0.3\textwidth}
    \centering
    \includegraphics[width=\textwidth]{figures/design/fet.png}
    \caption{Ovládanie napájania OpenLog cez FET}
    \label{schematics:fet}
\end{subfigure}
\label{schematics}
\caption{Schéma zapojenia hardvéru}
\end{figure}
 
Pamäťová Micro SD karta so súborovým systémom FAT32 bude pripojená v module \emph{OpenLog} od Sparkfun, ktorý zaznamená znakový
vstup prijímaný UART zbernicou do  textových súborov podľa pravidiel zo súboru \verb|config.txt| alebo povelmi odoslaných 
po zapnutí. Ukladať vzorky na externé médium nie je vždy žiaduce, preto bude napájanie spínané cez pin mikrokontroléra. Vyšší 
prúdový odber než dodá výstup a požadované napätie rovné s napájacím napätím dosky vyžaduje umiestnenie tranzistora riadeného 
poľom N-kanál \emph{BTS117} na premostenie riadiaceho signálu podľa obr. \ref{schematics:fet}.


\section{Architektúra systému}
Celková skladba komponentov systému pozostáva zo súčastí pôsobiacich na mikrokontroléri ESP32
interagujúcimi s akcelerometrom a zapisovačom cez lokálne sériové zbernice. Výstupy uspôsobenia a zoskupenia vektorov akcelerácie sú 
odosielané do počítačovej siete prostredníctvom prístupového bodu WiFi aplikačným protokolom MQTT (Message Queue Transport Telemetry)
na server poskytujúci službu MQTT broker správ Eclipse Mosquitto. Odtiaľ sú odoberané témy prešírené klientom metódou publish -- subcribe.
Diagram na obr. \ref{uml:component} zachytáva náhľad na poskladanie komponentov systému. 

\begin{figure}[h]
	\centering
	\includegraphics[width=\textwidth]{figures/design/components.png}
	\caption{Diagram komponentov}
	\label{uml:component}
\end{figure}

Odčítavanie úrovní z akcelerometra zabezpečuje časovač vzorkovania, ktorý si v pravidelných intervaloch pýta aktuálne hodnoty rozhraním 
periférneho adaptéra pre prístup ku SPI zbernici. Po vypršaní vzorkovacej periódy je zavolaná obsluha prerušenia, ktorá odblokuje vlákno 
úlohy na synchrónne načítanie okamžitého vektora zrýchlenia konvertovaného z číslicovej úrovne prevodníka na SI jednotku $m/s^2$. 
Postup vzorkovania ilustruje sekvenčný diagram na obr. \ref{uml:sequence}.

Zachytené hodnoty sú preposlané do nezávislých thread-safe front analyzátora signálu pre individuálne priestorové osi akcelerácie. 
V prípade nutnosti zachytávať ucelený priebeh veličiny  sa vektor umiestni do frontu ku vláknu loggera. Postup vzorkovania 
ilustruje sekvenčný diagram na obr. \ref{uml:sequence}. Pipeline spracovania signálu rozdeľuje vzorky do prelínajúcich sa posuvných okien, 
počíta z nich štatistiky a vyhľadáva udalosti vo frekvenčnom spektre. Nespracované hodnoty sú podľa potreby ukladané na pamäťovú kartu. 

Zmenu a uchovanie parametrov pipelinov pre jednotlivé bloky spracovania zaobstaráva správa konfigurácie. Modifikované nastavenia sú medzi 
spusteniami zachované vo vyhradenej partícií nevolatilnej flash pamäte na záznamy dvojíc v asociatívnej štruktúre. Predvolené správanie 
načítané pre nedostupnosť stavu z flash úložiska určujú konštanty v programovej pamäti.

Binárny serializačný formát Message Pack zaobaluje vzorky, udalosti a konfigurácia posielané na rozličné MQTT témy. Vychádza z 
formátu JSON (JavaScript Object Notation), ale narozdiel od neho sa sústredí na efektívne kódovanie dátových typov. Namiesto prevodu 
číselných údajov do znakového kódovania, napr. Unicode, ponecháva ich pôvodnú binárnu reprezentáciu s špecifikovanou bytovou značkou 
určujúcou typ. Hodnoty vyjadriteľné menším počtom bajtov reprezentuje dokonca v kratšom tvare než podmieňuje ich celkový rozsah.
Rovnako v zoznamoch a slovníkov sa  zaobchádza bez oddelovacích znakov, ktoré nahrádza informáciou o počte údajov. Dĺžka 
položky predchádzajúca zloženými atribútom uľahčuje následné parsovanie.

\begin{figure}[h]
	\centering
	\includegraphics[width=\textwidth]{figures/design/tasks.png}
	\caption{Sekvenčný diagram vzorkovania signálu}
	\label{uml:sequence}
\end{figure}

\section{Funkčné bloky dátovej pipeline}

\begin{figure}[h]
	\centering
	\includegraphics[width=\textwidth]{figures/design/pipeline.png}
	\caption{Diagram aktivít navrhovaného systému na spracovanie dát zo senzora}
\end{figure}


\begin{figure}[h]
	\centering
	\includegraphics[width=\textwidth]{figures/design/configuration.png}
	\caption{Diagram aktivít konfigurácie}
\end{figure}

Podľa pravidlami aktivovanými modulmi bude súbežne vykonateľných 5 ciest spracovania. 
Frekvenčná analýza bude pozostávať z krokov oknovania signálu  zvoliteľnou oknovou funkciou nastaviteľnej veľkosti,
následne sa signál transformuje do frekvenčnej domény,
kde dôjde k vyhladeniu spektra kĺzavým priemerom alebo Welchovou metódou a nakoniec sa podľa prítomnosti špičiek
vo frekvenčnom vedierku vytvorí udalosť na začiatku a na konci súvislého časového pôsobenia. Časová analýza
bude detegovať náhle zmeny v priebehu signálu na základe štatistík polohy, rozptýlenosti a tvaru
distribúcie vzhľadom na predošlé okná. Okrem toho sa môžu po podvzorkovaní odosielať alebo ukladať ,,surové dáta''
hodnôt veličiny v čase alebo frekvenčných vedierok. Numerická kvadratúra korigovaná obálkami umožní extrahovať zo
snímaného zrýchlenia odhad o rýchlosti a polohe.

Diagram aktivít na obr. \ref{fig:design}
vizualizuje navrhovaný beh činností na zariadení.

\clearpage


\section{Prúdový algoritmus detekcie zmien frekvencií}
Nevidí celý prúd naraz ani ho nemôže celý uložiť. Welchove priemerovanie vyžaduje veľa pamäte pre dlhšie okno.
Ale musíme dosiahnuť odšumenie detegovanie špičiek a zároveň posielať cez sieť menej ako nespracované
frekvenčné vedierka. Detegovanie anomálií, resp. automatické upozornenie na prevládajúce zložky.

\begin{figure}[h]
\centering
\begin{subfigure}[b]{0.8\textwidth}
    \centering
    \includegraphics[width=\textwidth]{figures/design/event-detection-min-duration.png}
    \caption{Minimálne trvanie udalosti $t_{\min} = 2$}
\end{subfigure}
\begin{subfigure}[b]{0.8\textwidth}
    \centering
    \includegraphics[width=\textwidth]{figures/design/event-detection-time-proximity.png}
    \caption{Maximálna vzdialenosť špičiek $t_{\Delta} = 1$ v čase}
\end{subfigure}
\caption{Parametre algoritmu na detekciu udalostí. $t$ je poradové číslo okna, $f$ je stav detegovanie špičky
vo frekvenčnom vedierku, $t_d$ je trvanie udalosti (duration), $t_p$ je počet okien do minulosti naposledy videnej špičky (past)}
\end{figure}

\begin{algorithm}[h]
\caption{Detektor zmeny frekvenčnej zložky}
\begin{algorithmic}[1]
\Require{$event$, $bin$, $t$, $t_{min}$, $t_{\Delta}$}
\If {V predošlom okne $t - 1$ bola emitovaná udalosť Koniec}
	\State Vynuluj udalosť: $event$: $duration \gets amplitude \gets 0$, $lastSeen \gets -1$
\EndIf

\If {$\mathrm{IsPeak}(bin)$}  
	\State $link \gets \max\{1, event.lastSeen + 1\}$
	\If {$event.duration < t_{min} \leq event.duration + link$}
		\State $event.start \gets t - event.duration - link + 1$
		\State \textbf{Emituj udalosť Štart} výskytu frekvencie podľa $event$
	\EndIf
	
	\State \textbf{Inkrementuj} $event.duration$ \textbf{o} $link$
	\State \textbf{Inkrementuj} $event.amplitude$ \textbf{o} $(bin - event.amplitude)\;/\;event.duration$
	\State $event.lastSeen \gets 0$

\ElsIf {$event.lastSeen \geq 0$}
	\State \textbf{Inkrementuj} $event.lastSeen$ \textbf{o}  $1$

	\If {$events.lastSeen > t_{\Delta}$}
        	\If {$event.duration \geq t_{min}$}
        		\State \textbf{Emituj udalosť Koniec} výskytu frekvencie podľa $event$
        	\EndIf
        \Else
        	\State Vynuluj udalosť: $event$: $duration \gets amplitude \gets 0$, $lastSeen \gets -1$
        \EndIf
\EndIf
\end{algorithmic}
\label{algo:event-detector}
\end{algorithm}

\section{Zber vibrácií z premávky}
V časti motora alebo nad nápravou z mestskej hromadnej dopravy. 500Hz, rozlíšní 2g z v režime zapisovania na SD kartu.
Predošlé prieskumné merania na akcelerometri smarfónu ukázali rozsah do +/-$8 m/s^2$do 1g najčastejšie do $2 - 3 m/s^2$

Pre LSB formát je tesne pod teoretickou hranicou rýchlosti UART (115200 baud). 1 znak = 8b + štart + stop = 10bitov.
Rozsah hodnôt signed int16 (5-ciferné číslo a znamienko na jednu súradnicu). Riadok mal teda aj s medzerami a novým riadkom max.
21 znakov na 3D vzorku. 21 znakov je 210 bitov. Teoretická max. vzorkovacia frekvencia je 548 Hz.

Realistické vzhľadom k prevážanému obsahu

\section{Generátor frekvenčného spektra}
Opis políčok konfigurácie a postup generovania fade-in/out sinsoid a náhodné vkladanie do signálu

\begin{figure}[h]
   \centering
    \includegraphics[width=0.8\textwidth]{figures/verification/fade-in-sinusoid.png}
   \caption{Základný tón formujúci syntetický signál}
\end{figure} 


\chapter{Implementácia}
- Programovacie jazyky a knižnice: 
	C, esp-idf v4.0, esp-dsp v1.2.0, MPack v1.1 (\url{https://ludocode.github.io/mpack/md_docs_expect.html}), FreeRTOS v9.0.0
- Python3.10, numpy, scipy (stats, signal, fft, interpolate), pandas, seaborn, matplotlib
- msgpack, json, cmd


\begin{figure}[h]
\centering
\begin{subfigure}[b]{0.65\textwidth}
    \centering
    \includegraphics[width=\textwidth]{figures/design/esp32.jpg}
\end{subfigure}
\hfill
\begin{subfigure}[b]{0.3\textwidth}
    \centering
    \includegraphics[width=\textwidth]{figures/design/esp32-front.jpg}
\end{subfigure}
\label{schematics}
\caption{Schéma zapojenia hardvéru}
\end{figure}

\section{Adaptér akcelerometra}
Načítavanie a prepočet osi zrýchlenia z akcelerometra % na m/s^2 vo funkcii imu\_acceleration (imu, *x)
Nastavenie vzorkovacej frekvencie a dynamického rozsahu

\begin{lstlisting}[style=cstyle]
uint8_t buffer[6];
spi_read_buffer(imu->dev, 0x80 | IMU_OUT_X_L_XL, 6, buffer);
    
uint8_t xlo = buffer[0];
int16_t xhi = buffer[1];
xhi = (xhi << 8) | xlo;
*x = xhi * imu->precision * G_CONSTANT;
\end{lstlisting}


\section{Frekvenčná transformácia}
Uprava DCT v knižnici
\begin{lstlisting}[style=cstyle,caption=Transformácia do frekvenčnej domény vo funkcii process\_spectrum]
for (uint16_t i = 0; i < n; i++) {
	spectrum[2*i+0] = buffer[i] * window[i];
    spectrum[2*i+1] = 0;
}
dsps_fft2r_fc32_ae32(spectrum, n);
dsps_bit_rev2r_fc32(spectrum, n);
dsps_cplx2reC_fc32(spectrum, n);
\end{lstlisting}
 
Magnituda spektra v dB
\begin{lstlisting} [style=cstyle]         
for (uint16_t i = 0; i < bins; i++)
	spectrum[i] = dsps_sqrtf_f32_ansi(
          square(spectrum[i*2]) + square(spectrum[i*2+1])
    );

float ref = maximum(spectrum, bins);
for (uint16_t i = 0; i < bins; i++)
	spectrum[i] = 20 * log10f(spectrum[i] / ref);
\end{lstlisting}


\section{Detekcia udalostí}
\begin{lstlisting}[style=cstyle]
typedef struct {
    SpectrumEventAction action;
    uint32_t start;
    uint32_t duration;
    int32_t last_seen;
    float amplitude;
} SpectrumEvent;
\end{lstlisting}

\section{Vzdialená konfigurácia}
- Nástroj klienta
- Opis štruktúr konfigurácie
- Uloženie v pamäti
\begin{lstlisting}[style=cstyle]
typedef struct {
    SamplingConfig sensor;
    SmoothingConfig tsmooth;
    StatisticsConfig stats;
    FFTTransformConfig transform;
    SmoothingConfig fsmooth;
    EventDetectionConfig peak;
    SaveFormatConfig logger;
} Configuration;
\end{lstlisting}
- MessagePack formát - serilizácia a deserializácia (výmena iba pri zmene)

\begin{lstlisting}[style=cstyle]
Configuration c;
memcpy(&c, &conf, sizeof(c));
change = config_parse(event->data, event->data_len, &c, &error);

if (error) {
    esp_mqtt_client_publish(client, MQTT_TOPIC_SYSLOG, "config malformed", 0, 1, 0);
} else if (change) {
	nvs_save_config(&c);
	esp_mqtt_client_publish(client, MQTT_TOPIC_SYSLOG, "config applied", 0, 1, 0);
	esp_wifi_stop();
	esp_wifi_deinit();
	esp_restart();
}
\end{lstlisting}

\begin{verbatim}
# /etc/mosquitto/mosquitto.conf
listener 1883 0.0.0.0
allow_anonymous true
\end{verbatim}

\emptypage


\chapter{Overenie riešenia} \label{chapter:verification}
Funkčnosť a efektivitu riešenia v súlade s kladenými požiadavkami overíme v rozličných scenároch. Zároveň experimentálne
demonštrujeme odvodenie hyperparametrov klasifikácie špičiek s mriežkovým vyhľadávaním (grid search)
a vyjadríme úspešnosť zaužívanými metrikami.

\section{Pamäťová efektivita}
Skompilovaný program senzorovej jednotky sa zmestí do pamäte inštrukcií so značnou rezervou. Kódový segment zaberá
64,42\% alebo 81,8 kB dostupného priestoru vynímajúc vyhradené časti na vektory prerušení a vyrovnávacie pamäte procesora.
Obsadením 43,4 kB v segmente .bss, prevažne na staticky alokované polia reťazcov odosielaných správ, a nárokovania si 14,7 kB
konštánt, zostáva 82,3\% DRAM na haldu. Spotrebu SRAM v bajtoch podľa segmentov objektového súboru podľa nástroja
\emph{GNU size} uvádza tabuľka \ref{tab:ram-segments}.
\begin{table}[h]
\def\arraystretch{1.25}
\begin{tabular}{|l|llll|lll|}
\hline
                     & \multicolumn{4}{c|}{\textbf{IRAM (192 kB)}}                                                                              & \multicolumn{3}{c|}{\textbf{DRAM (328 kB)}}                                           \\ \hline
\textbf{Sekcia}      & \multicolumn{1}{l|}{CPU cache} & \multicolumn{1}{l|}{.vectors} & \multicolumn{1}{l|}{.text} & voľné                      & \multicolumn{1}{l|}{.bss}  & \multicolumn{1}{l|}{.data} & voľné (heap)                \\ \hline
\textbf{Veľkosť} & \multicolumn{1}{r|}{65536}     & \multicolumn{1}{r|}{1027}     & \multicolumn{1}{r|}{83780} & \multicolumn{1}{r|}{46265} & \multicolumn{1}{r|}{44392} & \multicolumn{1}{r|}{15040} & \multicolumn{1}{r|}{276440} \\ \hline
\end{tabular}
\caption{Rozdelenie pamäte v bajtoch medzi sekcie}
\label{tab:ram-segments}
\end{table}

Vyťaženie dynamickej pamäte z haldy lineárne závisí od počtu súčasne vyhodnocovaných údajových bodov. Trend sa prejavuje v grafe
celkovej percentuálnej naplnenosti haldy \ref{graph:mem-usage}. Markantný stály podiel z voľného priestoru až okolo 55 kB sa poskytne na komunikáciu cez WiFi a na TCP/IP protokolový zásobník (graf \ref{graph:task-memory})

Zvyšok majú k dispozícii vlákna úloh na manipuláciu so vzorkami osí zrýchlenia (x, y, z), na zdieľané koeficienty FFT a
konvolučné masky a  relatívne nepatrne si obsadia úlohy vzorkovania (imu) a zápisu na pamäťovú kartu (logger).
FreeRTOS je nastavený na dynamickú alokáciu zásobníkov, preto už pri veľkosti okna 8 potrebuje spracovateľská úloha 10 kB. Najdlhšie
akceptovateľné posuvné okno a tým aj veľkosť frekvenčnej transformácie je 1024 bodov, ktoré vzhľadom na 97\% spotreby haldy za istých
okolností vykazuje nestabilitu systému. Odporúča sa pri zložitejšom procese úpravy signálu vystačiť si s 512 bodmi.

\begin{figure}[h]
	\centering
     \hfill
     \begin{subfigure}{0.48\textwidth}
        \centering
     	\includegraphics[width=\textwidth]{figures/verification/memory-usage-percentage.png}
     	 \caption{Celková spotreba pamäte}
     	 \label{graph:mem-usage}
     \end{subfigure}
     \hfill
      \begin{subfigure}{0.48\textwidth}
    	\centering
        \includegraphics[width=\textwidth]{figures/verification/memory-profile-bytes.png}
         \caption{Rozdelenie pamäte medzi úlohy}
        \label{graph:task-memory}
     \end{subfigure}
     \caption{Profilovanie dynamickej pamäte z haldy v DRAM}
\end{figure}


Veľkosť vyrovnávacej pamäte má priamy dopad na objem posielanej sieťovej premávky, ako vyčísľuje tabuľka \ref{tab:msg-size} v bajtoch.
Sekvencia $n$ meraní má za následok $m$ frekvenčných vedierok s násobiacim faktorom veľkosti dátového typu. Nesúrodé
štruktúry sú vyjadrené úhrnnou mierou informácie.
\begin{table}[h]
\def\arraystretch{1.25}
\centering
\begin{tabular}{|l|r|r|rr|cr|r|l}
\cline{1-8}
\multirow{2}{*}{\textbf{MQTT topic}} & \multicolumn{1}{c|}{\multirow{2}{*}{\textbf{\begin{tabular}[c]{@{}c@{}}Min. \\topic \end{tabular}}}} & \multicolumn{1}{c|}{\multirow{2}{*}{\textbf{\begin{tabular}[c]{@{}c@{}}Veľkosť\\ v RAM\end{tabular}}}} & \multicolumn{2}{l|}{\textbf{Hlavička (h)}}                      & \multicolumn{2}{l|}{\textbf{Prvok (p)}}                         & \multicolumn{1}{c|}{\multirow{2}{*}{\textbf{\begin{tabular}[c]{@{}c@{}}Max. celková\\ veľkosť\end{tabular}}}} & \textbf{} \\ \cline{4-7}
                                     & \multicolumn{1}{c|}{}                                                                                     & \multicolumn{1}{c|}{}                                                                                  & \multicolumn{1}{c|}{\textbf{Min.}} & \multicolumn{1}{c|}{\textbf{Max.}} & \multicolumn{1}{c|}{\textbf{Min.}} & \multicolumn{1}{c|}{\textbf{Max.}} & \multicolumn{1}{c|}{}                                                                                         &           \\ \cline{1-8}
config/response                      & 21                                                                                                        & 124                                                                                                    & \multicolumn{2}{c|}{-}                                                  & \multicolumn{1}{r|}{-}             & \multicolumn{1}{l|}{}              & 450                                                                                                           &           \\ \cline{1-8}
samples                              & 13                                                                                                        & $4\cdot n$                                                                                                    & \multicolumn{1}{r|}{1}             & 3                                  & \multicolumn{2}{c|}{5}                                                  & $h + p\cdot n$                                                                                                      &           \\ \cline{1-8}
spectrum/+                           & 16                                                                                                        & $4\cdot m$                                                                                                & \multicolumn{1}{r|}{14}            & 22                                 & \multicolumn{2}{c|}{5}                                                  & $h + p\cdot m$                                                                                                 &           \\ \cline{1-8}
stats/+                              & 13                                                                                                        & 52                                                                                                     & \multicolumn{1}{r|}{4}             & 8                                  & \multicolumn{1}{r|}{9}             & 12                                 & 127                                                                                                           &           \\ \cline{1-8}
events/+                             & 14                                                                                                        & $20\cdot m$                                                                                               & \multicolumn{1}{r|}{18}            & 26                                 & \multicolumn{1}{r|}{17}            & 27                                 & $h + p\cdot m$                                                                                                &           \\ \cline{1-8}
\end{tabular}
\caption{Veľkosti Message Pack správ podľa MQTT topic}
\label{tab:msg-size}
\end{table}

Používaný formát Message Pack máp sa zväčša skladá z hlavičky, čo je názov
pre dvojice spoločne opisujúce variabilný počet obsiahnutých prvkov. Rozpätie v objeme hlavičky a položiek vyplýva z
balenia menších hodnôt pod kratšiu binárnu reprezentáciu. Produkovaný obsah sa rozčleňuje na základe logických kategórií
do MQTT topics vkladané do hlavičky protokolu s dĺžkou vyjadrenou vrátane najkratšieho prefixu.

Aby sme detekciou udalostí dosiahli úsporu v množstve prenášaných údajov, je žiaduce dosiahnuť kratšiu správu ako
zaslaním frekvenčných vedierok bez úpravy. Maximálna celková veľkosť dátovej nálože na tému \emph{events} musí byť menšia
ako na tému \emph{spectrum}. Pri $m = 16$ to činí 2 udalosti na okno (12,5\% z celkového počtu vedierok) a pri $m = 512$
sa musí vyskytnúť menej ako 93 udalostí na okno (18,16\%). Datasety z autobusov vykazujú emisiu udalostí najviac do
približne 6\% z úhrnného počtu frekvencií a 0,5\% v priemere. Štatistiky sa oplatí vytvárať pri najmenšom $n = 32$.

\begin{table}[h]
\def\arraystretch{1.25}
\centering
\begin{tabular}{|l|r|r|r|r|r|}
\hline
\textbf{Protokol}         & \textbf{Ethernet II} & \textbf{IPv4} & \textbf{TCP} & \textbf{MQTT}  & \textbf{Spolu}  \\ \hline
\textbf{Veľkosť hlavičky} & 14                   & 20            & 20           & \textgreater 5 & \textgreater 55 \\ \hline
\end{tabular}
\caption{Réžia sieťového prenosu v bajtoch}
\label{tab:net-overhead}
\end{table}

Vysielané správy zapúzdrené v sieťových protokoloch nižších vrstiev OSI modelu pridávajú
hlavičky na správne doručenie adresátovi, čím zvyšujú celkovú réžiu. V lokálnej WiFi sieti, kde bola infraštruktúra prítomná,
sa protokol MQTT pôsobiaci nad TCP, prenášal cez IPv4 v Ethernet-ových rámcoch. Každý paket má preto veľkosť vždy najmenej
podľa tabuľky \ref{tab:net-overhead}. Prenášaný obsah môže prevyšovať MTU, čo zapríčiní fragmentáciu do viacerých TCP segmentov
a ďalší nárast nadbytku.

\section{Časová efektivita}
Naplnenie kritérií na rýchlosť odozvy odmeriame systémovým časovačom s mikrosekundovou presnosťou.
Vplyv jednotlivých algoritmov na trvanie procesu úpravy signálu spriemerovaním 10 behov
je patrný z tabuľky \ref{tab:algorithm-execution}. Aktívna bola len úloha pre vybranú os zrýchlenia. S ohľadom na odchýlky
najmä v dôsledku prerušení od vzorkovania a obsluhy plánovača operačného systému sa potrebný čas navyšuje priamo úmerne
s dĺžkou postupnosti bodov.
\begin{table}[h]
\def\arraystretch{1.25}
\centering
\begin{tabular}{|l|l|l|l|l|l|l|}
\hline
\textbf{Veľkosť okna}         & \textbf{32} & \textbf{64} & \textbf{128} & \textbf{256} & \textbf{512} & \textbf{1024} \\ \hline
\textbf{Štatistiky bez korelácii}& 3673        & 7471        & 14652        & 29574        & 59158        & 112871        \\ \hline
\textbf{DFT}                     & 80          & 162         & 306          & 611          & 1243         & 2620          \\ \hline
\textbf{DCT}                     & 91          & 165         & 310          & 612          & 1226         & 2532          \\ \hline
\textbf{Špičky: susedia}         & 45          & 102         & 216          & 451          & 913          & 1812          \\ \hline
\textbf{Špičky: nulou do záporu} & 6           & 10          & 17           & 33           & 62           & 121           \\ \hline
\textbf{Špičky: horský turista}  & 19          & 32          & 54           & 109          & 199          & 431           \\ \hline
\textbf{Udalosti}                & 7           & 10          & 17           & 31           & 58           & 114           \\ \hline
\end{tabular}
\caption{Čas vykonávania algoritmov od veľkosti posuvného okna v $\mu s$ pri taktovacej frekvencii 160 MHz a intervale
plánovania 100 Hz}
\label{tab:algorithm-execution}
\end{table}

Výpočet štatistík sa najvýraznejšie podieľa na predlžovaní obratu spracovania rádovo v desiatkach
milisekúnd, zatiaľ čo väčšina krokov prebehne aspoň 40-krát rýchlejšie. Nevyrovnanosť zapríčiňujú miery vychádzajúce z
mediánu, pretože sa opakovane aplikuje Quickselect. Okrem toho si povšimneme, že v rýchlosti vykonávania nie je
žiaden rozdiel medzi frekvenčnou transformáciou s FFT a DCT, z dôvodu spomenutých nedostatkov knižničnej implementácie DCT-II.

Najpomalším hľadaním špičiek je metóda najvyššieho spomedzi susedov, ktorá má najhoršiu asymptotickú časovú zložitosť
spomedzi preberaných spôsobov. Dosahuje do 4-krát dlhšie časy ako algoritmus horského turistu a do 14-krát oproti prechodu
nulou do záporu. Výber prístupu ku klasifikácii špičiek nezáleží len od rýchlosti, ale tiež od charakteru rozloženia
vrcholov líšiaceho sa medzi algoritmami.

Vyhladzovanie v časovej alebo frekvenčnej doméne sa vyznačuje meniteľnou dĺžkou konvolučnej masky a počtom opakovaných
prechodov. Čas na dokončenie rovnako stúpa lineárne podľa oboch vlastností.
\begin{table}[h]
\def\arraystretch{1.25}
\centering
\begin{tabular}{|l|r|r|r|}
\hline
\textbf{Veľkosť masky}  & \textbf{4} & \textbf{16} & \textbf{64} \\ \hline
\textbf{1x} & 108        & 262         & 891         \\ \hline
\textbf{4x} & 413        & 1041        & 3697        \\ \hline
\textbf{8x} & 819        & 2065        & 7209        \\ \hline
\end{tabular}
\caption{Čas v $\mu s$ na vyhladzovanie v závislosti od veľkosti konvolučného jadra pri N = 512 a počtu opakovaní}
\label{tab:kernel-execution}
\end{table}

Porovnávanie variant fáz spracovania separátne nezohľadňuje serializáciu a publikovanie správ zvolených tém, či
dopad plánovania a synchronizácie na konkurentné úlohy. Pozrieme sa na výkonnosť postupu spracovania dát v dvoch odlišných prípadoch
pri frekvencii procesora 160 MHz a spriemerovaním desiatich spustení.

Správanie zariadenia v pokoji, bez odvodzovania akýchkoľvek štatistík, netvoriace sieťovú premávku, približuje tabuľka
\ref{tab:pipeline-simple}. Časy po pridaní výpočtu dostupných štatistík vrátane korelácií a upozorňovanie na udalosti
cez bezdrôtovú linku s RSSI na hladine cca -40 dBm popisuje tabuľka \ref{tab:pipeline-complex}. Vyhodnocuje sa trvanie behu
v mikrosekundách vzhľadom na veľkosť posuvného okna podľa algoritmu na hľadanie špičiek (číslovanie: A1, A2, A3, podľa kapitoly analýzy)
pre aktivovaný počet rozmerov akcelerácie. U troch dimenzií sa zohľadní do priemeru úloha s najdlhším časom vykonávania.
Frekvenčná transformácia je použitá za každej situácie FFT.

\begin{table*}[h]
     \def\arraystretch{1.25}
	 \centering
     \captionsetup[subtable]{position=below}

     \begin{subtable}{0.48\linewidth}
         \centering
		\begin{tabular}{|l|l|r|r|r|}
		\hline
		\textbf{}                       & \textbf{N}  & \textbf{32} & \textbf{256} & \textbf{1024} \\ \hline
		\multirow{3}{*}{\textbf{1 os}}  & \textbf{A1} & 518         & 2616         & 6401          \\ \cline{2-5}
 		                                & \textbf{A2} & 435         & 1598         & 6209          \\ \cline{2-5}
                                        & \textbf{A3} & 467         & 1695         & 3864          \\ \hline
		\multirow{3}{*}{\textbf{3 osi}} & \textbf{A1} & 2503        & 3077         & 10177         \\ \cline{2-5}
                                        & \textbf{A2} & 556         & 3340         & 10334         \\ \cline{2-5}
                                        & \textbf{A3} & 591         & 1295         & 4670          \\ \hline
		\end{tabular}
		\caption{V pokoji bez posielania správ}
		\label{tab:pipeline-simple}
	\end{subtable}
    \hfill
    \begin{subtable}{0.48\linewidth}
         \centering
		\begin{tabular}{|l|l|r|r|r|}
		\hline
		\textbf{}                       & \textbf{N}  & \textbf{32} & \textbf{256} & \textbf{1024} \\ \hline
		\multirow{3}{*}{\textbf{1 os}}  & \textbf{A1} & 14750       & 34883        & 129190        \\ \cline{2-5}
                              			& \textbf{A2} & 8824        & 34351        & 139451        \\ \cline{2-5}
                                        & \textbf{A3} & 13795       & 34890        & 137346        \\ \hline
		\multirow{3}{*}{\textbf{3 osi}} & \textbf{A1} & 23851       & 101978       & 273696        \\ \cline{2-5}
                                        & \textbf{A2} & 22981       & 78972        & 272161        \\ \cline{2-5}
                                        & \textbf{A3} & 24232       & 100156       & 270110        \\ \hline
		\end{tabular}
		\caption{Štatistiky s koreláciami a udalosti}
		\label{tab:pipeline-complex}
	\end{subtable}

	 \captionsetup[table]{position=below}
     \caption{Čas na spracovanie okna vzoriek v $\mu s$.}
     \label{tab:pipeline}
\end{table*}


Hraničný čas $t$ pokiaľ si vystačíme s tzv. ,,double buffering'', čiže sa neoneskorujeme od prúdu prichádzajúcich vzoriek o viac ako jednu
dĺžku vyrovnávacej pamäte $N$ nastáva, keď platí: $t \leq N / f_s$. Na určenie teoreticky najvyššej vzorkovacej frekvencie
pri danej dĺžke $N$ vezmeme časový parameter z tabuľky \ref{tab:pipeline}.

Ľubovoľné nastavenie spracovania signálu zvládne za okolností podľa \ref{tab:pipeline-simple} $f_s =$ 61,8 kHz pre jednorozmernú
sekvenciu a $f_s =$ 12,8 kHz pre trojrozmernú. Posielanie štatistík a udalostí z \ref{tab:pipeline-complex} pri jednej osi
dovoľuje $f_s =$ 2,1 KHz, ale pri väčších $N$ sa $f_s$ pohybuje nad 7 kHz. Tri dimenzie v tejto náročnej konfigurácii stíhajú nanajvýš $f_s$
od 1,3 KHz ($N = 1024$) do 3,7 kHz ($N = 32$). Limit vzorkovacej frekvencie senzora na 952 Hz zaručuje, že bežné prevádzkové
situácie sa stíhajú uskutočniť pred naplnením následného posuvného okna.

\section{Úspešnosť detekcie špičiek}
Syntetický signál so známymi časovo-frekvenčné spektrom sa zložil zo sinusoíd s exponenciálnym nábehom a dobehom.
Očakávaná kvantita zlúčených tónov na základný úsek sa stanovila podľa frekvencie vzorkovania ovplyvňujúcej rozlišovaciu
schopnosť susediacich komponentov. Pri 238 Hz sa zmiešalo 8 rozdielnych frekvencií, pri 476 Hz je ich 16 a pri 952 Hz sa
ich rozmiestnilo 32. Pseudonáhodný generátor inicializovaný so semienkom 10 vytvoril trénovaciu množinu s trvaním 60 sekúnd
a vzápätí 20 sekundovú testovaciu množinu.

Prehľadávaním parametrov detekcie špičiek hrubou silou nad trénovacím signálom boli odhalené najlepšie hodnoty z preddefinovanej sady.
Na testovacom signále sa zistili relevantné metriky úspešnosti klasifikácie vrcholov makro-priemerovaním medzi posuvnými oknami
o veľkosti $n$. Signál nebol dodatočne filtrovaný. Obdržanie spektrálneho obsahu v decibeloch prebehlo cez FFT s Hannovou
oknovou funkciou a 50\%  prekryvom okien. Vyčíslené sú presnosti (tab. \ref{tab:accuracy}), senzitivita (tab. \ref{tab:sensitivity}) a chybovosť (tab. \ref{tab:error-rate}) detekčných stratégií v percentách.

\begin{table}[h]
\def\arraystretch{1.1}
\centering
\begin{tabular}{|c|ccc|ccc|ccc|}
\hline
                    & \multicolumn{3}{c|}{\textbf{Algoritmus 1}}                                           & \multicolumn{3}{c|}{\textbf{Algoritmus 2}}                                           & \multicolumn{3}{c|}{\textbf{Algoritmus 3}}                                           \\ \hline
\diagbox{$n$}{$f_s$} & \multicolumn{1}{c|}{\textbf{238}} & \multicolumn{1}{c|}{\textbf{476}} & \textbf{952} & \multicolumn{1}{c|}{\textbf{238}} & \multicolumn{1}{c|}{\textbf{476}} & \textbf{952} & \multicolumn{1}{c|}{\textbf{238}} & \multicolumn{1}{c|}{\textbf{476}} & \textbf{952} \\ \hline
\textbf{128}        & \multicolumn{1}{c|}{84.46}        & \multicolumn{1}{c|}{73.76}        & 59.49        & \multicolumn{1}{c|}{83.67}        & \multicolumn{1}{c|}{74.12}        & 54.96        & \multicolumn{1}{c|}{83.69}        & \multicolumn{1}{c|}{73.41}        & 51.87        \\ \hline
\textbf{256}        & \multicolumn{1}{c|}{91.19}        & \multicolumn{1}{c|}{85.41}        & 73.59        & \multicolumn{1}{c|}{90.63}        & \multicolumn{1}{c|}{84.39}        & 71.84        & \multicolumn{1}{c|}{91.30}        & \multicolumn{1}{c|}{84.78}        & 72.00        \\ \hline
\textbf{512}        & \multicolumn{1}{c|}{95.54}        & \multicolumn{1}{c|}{91.91}        & 85.19        & \multicolumn{1}{c|}{95.45}        & \multicolumn{1}{c|}{91.92}        & 84.05        & \multicolumn{1}{c|}{95.54}        & \multicolumn{1}{c|}{91.93}        & 85.05        \\ \hline
\end{tabular}
\caption{Percentuálna presnosť klasifikácie frekvenčných špičiek}
\label{tab:accuracy}
\end{table}

\begin{table}[h]
\def\arraystretch{1.1}
\centering
\begin{tabular}{|c|lll|lll|lll|}
\hline
                    & \multicolumn{3}{c|}{\textbf{Algoritmus 1}}                                                                & \multicolumn{3}{c|}{\textbf{Algoritmus 2}}                                                                & \multicolumn{3}{c|}{\textbf{Algoritmus 3}}                                                                \\ \hline
\diagbox{$n$}{$f_s$}  & \multicolumn{1}{c|}{\textbf{238}} & \multicolumn{1}{c|}{\textbf{476}} & \multicolumn{1}{c|}{\textbf{952}} & \multicolumn{1}{c|}{\textbf{238}} & \multicolumn{1}{c|}{\textbf{476}} & \multicolumn{1}{c|}{\textbf{952}} & \multicolumn{1}{c|}{\textbf{238}} & \multicolumn{1}{c|}{\textbf{476}} & \multicolumn{1}{c|}{\textbf{952}} \\ \hline
\textbf{128}        & \multicolumn{1}{l|}{23.77}        & \multicolumn{1}{l|}{35.52}        & 41.04                             & \multicolumn{1}{l|}{16.11}        & \multicolumn{1}{l|}{22.33}        & 21.94                             & \multicolumn{1}{l|}{9.65}         & \multicolumn{1}{l|}{14.58}        & 9.47                              \\ \hline
\textbf{256}        & \multicolumn{1}{l|}{10.62}        & \multicolumn{1}{l|}{27.42}        & 29.98                             & \multicolumn{1}{l|}{6.58}         & \multicolumn{1}{l|}{15.33}        & 28.03                             & \multicolumn{1}{l|}{13.26}        & \multicolumn{1}{l|}{11.88}        & 9.51                              \\ \hline
\textbf{512}        & \multicolumn{1}{l|}{12.08}        & \multicolumn{1}{l|}{14.52}        & 32.09                             & \multicolumn{1}{l|}{3.68}         & \multicolumn{1}{l|}{0.00}         & 17.13                             & \multicolumn{1}{l|}{2.35}         & \multicolumn{1}{l|}{1.80}         & 15.26                             \\ \hline
\end{tabular}
\caption{Percentuálna senzitivita (TPR) klasifikácie frekvenčných špičiek}
\label{tab:sensitivity}
\end{table}

\begin{table}[h]
\def\arraystretch{1.1}
\centering
\begin{tabular}{|c|lll|lll|lll|}
\hline
                    & \multicolumn{3}{c|}{\textbf{Algoritmus 1}}                                                                & \multicolumn{3}{c|}{\textbf{Algoritmus 2}}                                                                & \multicolumn{3}{c|}{\textbf{Algoritmus 3}}                                                                \\ \hline
\diagbox{$n$}{$f_s$} & \multicolumn{1}{c|}{\textbf{238}} & \multicolumn{1}{c|}{\textbf{476}} & \multicolumn{1}{c|}{\textbf{952}} & \multicolumn{1}{c|}{\textbf{238}} & \multicolumn{1}{c|}{\textbf{476}} & \multicolumn{1}{c|}{\textbf{952}} & \multicolumn{1}{c|}{\textbf{238}} & \multicolumn{1}{c|}{\textbf{476}} & \multicolumn{1}{c|}{\textbf{952}} \\ \hline
\textbf{128}        & \multicolumn{1}{l|}{3.67}         & \multicolumn{1}{l|}{11.31}        & 21.59                             & \multicolumn{1}{l|}{3.06}         & \multicolumn{1}{l|}{5.38}         & 11.38                             & \multicolumn{1}{l|}{1.75}         & \multicolumn{1}{l|}{3.36}         & 4.87                              \\ \hline
\textbf{256}        & \multicolumn{1}{l|}{1.08}         & \multicolumn{1}{l|}{4.10}         & 8.66                              & \multicolumn{1}{l|}{1.26}         & \multicolumn{1}{l|}{3.12}         & 10.33                             & \multicolumn{1}{l|}{1.19}         & \multicolumn{1}{l|}{1.97}         & 2.41                              \\ \hline
\textbf{512}        & \multicolumn{1}{l|}{5.27}         & \multicolumn{1}{l|}{1.30}         & 5.00                              & \multicolumn{1}{l|}{0.96}         & \multicolumn{1}{l|}{0.00}         & 3.48                              & \multicolumn{1}{l|}{0.96}         & \multicolumn{1}{l|}{0.15}         & 1.95                              \\ \hline
\end{tabular}
\caption{Percentuálna chybovosť (FPR) klasifikácie frekvenčných špičiek}
\label{tab:error-rate}
\end{table}
Platnosť získaných percentuálnych metrík v absolútnej škále a hyperparametrov sa vzťahuje výlučne na zvolenú techniku
syntézy signálového priebehu, a teda nevieme potvrdiť podobné výsledky pre reálnu prepravu. Napriek tomu môžeme
z pravidelných tendencií dedukovať, že pomerne vysoká presnosť cca 84\%, ktorá zachováva senzitivitu, a chybovosťou do 5\%
sa stabilne objavuje na diagonálach, kde $n$ je polovicou $f_s$. Vtedy nachádzame ideálny pomer medzi rozlíšením
v čase a frekvencii aj v spektrogramoch.

Veľmi nízka senzitivita nezriedka 10 - 30\% je dôsledkom nedokonalej spätnej rekonštrukcie spektrálneho profilu a exaktnou
lokalizáciou vrchola, čím sa sinusoidy môžu ocitnúť posunuté o pár frekvenčných vedierok vedľa v porovnaní s označením v datasete.
Malý počet význačných frekvencií spôsobuje, že na prejavenie sa efektu zdanlivého poklesu senzitivity stačí, keď sa
mierne zmení poloha jedinej.

Spektrogram \ref{fig:synthetic-spectrogram} znázorňuje na testovacích dátach schopnosť prúdového algoritmu na detekciu zmien
frekvencií sa vysporiadať s nesprávne identifikovanými špičkami. Na vizualizácii je patrné, že
registrovanie prvého vrchola na súvislom výstupku nastávalo oneskorene prekročením určitej amplitúdy.
Ostatné algoritmy sa prejavovali obdobne, najvýraznejšie rozdiely sú v rozmiestnení falošných vrcholov.

\begin{figure}[h]
	\centering
     \begin{subfigure}{\textwidth}
        \centering
     	\includegraphics[width=\textwidth]{figures/verification/Sythetic-FFT-A1-476-256.png}
     	\caption{Nájdené špičky algoritmom č.1 v posuvných oknách vyznačené bielym kruhom}
     \end{subfigure}
     \begin{subfigure}{\textwidth}
    	\centering
    	\includegraphics[width=\textwidth]{figures/verification/Sythetic-A1-events.png}
   		\caption{Zachytený priebeh udalostí frekvenčnej zmeny pri $t_{min} = 4$ a $t_{\Delta} = 1$}
     \end{subfigure}
     \caption{Spektrogramy detegovaných špičiek a udalostí pri $f_s = 476$ Hz a $N = 256$}
     \label{fig:synthetic-spectrogram}
\end{figure}

Parametre klasifikácie špičiek nájdené mriežkovým hľadaním, s ktorými sme dosiahli na konkrétnom syntetickom signále
najvyššie presnosti sú v tabuľke \ref{tab:grid-serach-parameters}.
\begin{table}[h]
\def\arraystretch{1.1}
\centering
\begin{tabular}{|c|c|ccc|cc|cccc|}
\hline
\multirow{2}{*}{\textbf{\begin{tabular}[c]{@{}c@{}}$f_s$\end{tabular}}} & \multirow{2}{*}{\textbf{\begin{tabular}[c]{@{}c@{}}$n$\end{tabular}}} & \multicolumn{3}{c|}{\textbf{\begin{tabular}[c]{@{}c@{}}Algoritmus 1\end{tabular}}} & \multicolumn{2}{c|}{\textbf{\begin{tabular}[c]{@{}c@{}}Algoritmus 2\end{tabular}}} & \multicolumn{4}{c|}{\textbf{\begin{tabular}[c]{@{}c@{}}Algoritmus 2\end{tabular}}}  \\ \cline{3-11}
                                                                                  &                                                                         & \multicolumn{1}{c|}{\hspace*{3mm}$k$\hspace*{3mm}}         & \multicolumn{1}{c|}{\hspace*{3mm}$\epsilon$\hspace*{3mm}}         & $h_{rel}$       & \multicolumn{1}{c|}{\hspace*{4mm} $k$ \hspace*{4mm}}                               &  $s$                              & \multicolumn{1}{c|}{$t$} & \multicolumn{1}{c|}{$h$} & \multicolumn{1}{c|}{$p$} & $i$ \\ \hline
238                                                                               & 128                                                                     & \multicolumn{1}{c|}{12}          & \multicolumn{1}{c|}{3}                  & 32                & \multicolumn{1}{c|}{2}                                 & 12                                & \multicolumn{1}{c|}{16}  & \multicolumn{1}{c|}{0}   & \multicolumn{1}{c|}{10}  & 8   \\ \hline
238                                                                               & 256                                                                     & \multicolumn{1}{c|}{3}           & \multicolumn{1}{c|}{1}                  & 32                & \multicolumn{1}{c|}{2}                                 & 16                                & \multicolumn{1}{c|}{16}  & \multicolumn{1}{c|}{0}   & \multicolumn{1}{c|}{10}  & 12  \\ \hline
238                                                                               & 512                                                                     & \multicolumn{1}{c|}{3}           & \multicolumn{1}{c|}{4}                  & 32                & \multicolumn{1}{c|}{2}                                 & 26                                & \multicolumn{1}{c|}{10}  & \multicolumn{1}{c|}{4}   & \multicolumn{1}{c|}{38}  & 0   \\ \hline
476                                                                               & 128                                                                     & \multicolumn{1}{c|}{3}           & \multicolumn{1}{c|}{4}                  & 16                & \multicolumn{1}{c|}{2}                                 & 9                                 & \multicolumn{1}{c|}{10}  & \multicolumn{1}{c|}{0}   & \multicolumn{1}{c|}{10}  & 0   \\ \hline
476                                                                               & 256                                                                     & \multicolumn{1}{c|}{12}          & \multicolumn{1}{c|}{4}                  & 32                & \multicolumn{1}{c|}{2}                                 & 12                                & \multicolumn{1}{c|}{16}  & \multicolumn{1}{c|}{0}   & \multicolumn{1}{c|}{10}  & 0   \\ \hline
476                                                                               & 512                                                                     & \multicolumn{1}{c|}{3}           & \multicolumn{1}{c|}{4}                  & 32                & \multicolumn{1}{c|}{1}                                 & 20                                & \multicolumn{1}{c|}{10}  & \multicolumn{1}{c|}{4}   & \multicolumn{1}{c|}{33}  & 0   \\ \hline
952                                                                               & 128                                                                     & \multicolumn{1}{c|}{3}           & \multicolumn{1}{c|}{4}                  & 0                 & \multicolumn{1}{c|}{2}                                 & 5                                 & \multicolumn{1}{c|}{10}  & \multicolumn{1}{c|}{0}   & \multicolumn{1}{c|}{10}  & 0   \\ \hline
952                                                                               & 256                                                                     & \multicolumn{1}{c|}{6}           & \multicolumn{1}{c|}{4}                  & 24                & \multicolumn{1}{c|}{2}                                 & 5                                 & \multicolumn{1}{c|}{16}  & \multicolumn{1}{c|}{0}   & \multicolumn{1}{c|}{10}  & 0   \\ \hline
952                                                                               & 512                                                                     & \multicolumn{1}{c|}{6}           & \multicolumn{1}{c|}{4}                  & 32                & \multicolumn{1}{c|}{2}                                 & 12                                & \multicolumn{1}{c|}{16}  & \multicolumn{1}{c|}{0}   & \multicolumn{1}{c|}{10}  & 0   \\ \hline
\end{tabular}
\caption{Parametre algoritmov detekcie špičiek na syntetických dátach (názvy sú skrátené na prvé písmená)}
\label{tab:grid-serach-parameters}
\end{table}

Manuálnym odladením parametrov na ukážkových záznamoch z vozidiel sme odskúšali rozdiely v schopnostiach algoritmov
na hľadanie špičiek povšimnúť si pretrvávajúce harmonické zložky. Výrez časovo-premenného spektra na obr. \ref{spectrum-slice}
s nájdenými špičkami zvýrazňuje prekážky správneho odlíšenia šumu alebo navzájom splývajúcich vrcholov. Niektoré lokálne extrémy
sú prehliadnuté pre nevýraznosť nad svojím okolím alebo pre prílišnú sploštenosť.
\begin{figure}[h]
   \centering
    \includegraphics[width=\textwidth]{figures/verification/L83-slice-t-20-A1.png}
   \caption{Prierez spektrogramu okna 256 vzoriek s vrcholmi označenými algoritmom č.1
   v 20. sekunde záznamu \emph{L83\_4940\_Alexyho\_Svantnerova.csv}}
   \label{spectrum-slice}
\end{figure}

Nie je úplne zrejmé aké vodorovné konštelácie výrazných čŕt z obr. \ref{dataset-detection} sú korektne vyznačené,
a ktoré sú primerane agregované do spoločných udalostí. Podstatné frekvencie v stabilnej oblasti medzi časmi 22 a 33
sekundou ako napr. 11, 26, 56 Hz  boli univerzálne zaevidované. Detekcie z iného datasetu sú ilustrované v prílohe
\ref{appendix:spectrogram}.

\begin{figure}[h]
	\centering
     \begin{subfigure}{\textwidth}
        \centering
     	\includegraphics[width=\textwidth]{figures/verification/L83-dataset-A1.png}
     	\caption{Algoritmus č.1: $k = 5$, $\epsilon = 0$, $h_{rel} = 8$}
     \end{subfigure}
     \begin{subfigure}{\textwidth}
    	\centering
        \includegraphics[width=\textwidth]{figures/verification/L83-dataset-A2.png}
        \caption{Algoritmus č.2: $k = 3$, $s = 7$}
     \end{subfigure}
      \begin{subfigure}{\textwidth}
    	\centering
        \includegraphics[width=\textwidth]{figures/verification/L83-dataset-A3.png}
        \caption{Algoritmus č.3: $t = 8$, $h = 1$, $p = 5$, $i = 0$}
     \end{subfigure}
     \caption{Detekcia udalostí v datasete \emph{L83\_4940\_Alexyho\_Svantnerova.csv} s $f_s = 500$ Hz, trvaním 60 s, dĺžkou okna 256,
     pri $t_{min} = 10$ a $t_{\Delta} = 4$}
     \label{dataset-detection}
\end{figure}


\chapter{Zhodnotenie}
Vytvorili sme model spracovania signálu z vibrácii na existujúcom vzdiealene konfigurovateľnosm IoT zariadení.
Známe algoritmy detekcie špičiek sme aplikovali na náš problém s ich porovnaním a
odladením vlastných algoritmov detekcie udalostí.
Rozšírenia sú početné:
\begin{itemize}[noitemsep]
\item Podrobiť (analýze/rozboru) rôzne situácie na vybraných úsekoch ciest opakovane a vypracovať metodiku anotovania datasetu,
kvôli vzájomnému porovnaniu súvislostí
\item Testovanie generátora udalostí na vibrátore s vytváraním známeho frekvenčného priebehu nielen na syntetických dátach
\item Umožniť označiť a pomenovať profil známych javov a dokázať ich odlíšiť pri notifikáciach.
\item umožniť filtrovanie nad frekvenčným rozsahom záujmu
\item Brať do úvahy postupný drift alebo tolerovateľné disturbancie, alebo iné stabilné opakujúce sa vzory a zamyslieť sa
nad ich vhodným oznamovaním
\item Koordinovať viacero senzorov v prevážanom boxe
\item Určenie prevažného priestorového smeru pôsobenia frekvenčných zložiek vibrácii napríklad viacrozmernou FFT
\item Autonómna kalibrácia parametrov hľadania špičiek.
\item Bližšie sa venovať úspore, ktoré by poskytli verzie DCT svojou lepšiou distribúciou energie v koeficientoch
\item  Preskúmať reálne nasadenie a monitorovanie v reálnom čase napríklad cez mobilnú aplikáciu cez Bluetooth komunikáciu,
ktorá sa ukázalo na protokole s väčšími prenosovými rýchlosťami, že by mohli postačovať.
\end{itemize}

\cleardoublepage

\printbibliography[heading=bibintoc, title={Literatúra}]

%  Prílohy -----------------------------------------------------------------------
\addtocontents{toc}{\protect\setcounter{tocdepth}{0}}
\addtocontents{toc}{\cftpagenumbersoff{chapter}}
\let\svaddcontentsline\addcontentsline
\renewcommand\addcontentsline[3]{%
  \ifthenelse{\equal{#1}{lof}}{}%
  {\ifthenelse{\equal{#1}{lot}}{}{\svaddcontentsline{#1}{#2}{#3}}}}

\appendix
\titleformat{\chapter}{\normalfont\huge\bf}{Príloha \thechapter:}{1em}{}
\renewcommand{\chaptermark}[1]{\markboth{\MakeUppercase{Príloha \thechapter.\ #1}}{}}

\thispagestyle{empty}
\chapter{Plán práce}
\pagenumbering{arabic}
\renewcommand*{\thepage}{A-\arabic{page}}

\section{Zimný semester}

\begin{table}[h!]
\def\arraystretch{1.25}
\begin{tabular}{|l|p{12cm}|}
\hline
\textbf{Obdobie} & \textbf{Náplň práce}                                                                                                                                                                                                                         \\ \hline
1. týždeň         & Základný prehľad relevantnej literatúry.                                                                                                                                                                                                      \\ \hline
2. týždeň         & Štúdium literatúry ohľadom montorovania vibrácií. Pokusný zber dát akcelerácie z MHD pomocou akcelerometra na smartfóne a ich prieskumná analýza.                                             \\ \hline
3. týždeň         & Štúdium článkov o frekvenčnej analýze a rešerš algoritmov na hľadanie špičiek. Implementácia objavených prístupov hľadania špičiek a aplikovanie na merania vibrácií z električiek a autobusu. \\ \hline
4. týždeň         & Flashovanie firmvéru na vývojový kit iCOMOX od Shiratech.                                                                                                                                                                                                  \\ \hline
5. týždeň         & Osnova práce s referenciami na nájdenú literatúru.                                                                                                                                                                                            \\ \hline
6. týždeň         & Firmvér pre dosku na platforme ESP32 pre záznam akceleračných dát na SD kartu cez OpenLog \\ \hline
7. týždeň         & Merania vibrácií v MHD a analýza získaných záznamov v Jupyter notebooku. Doplnenie zdrojov pre časti osnovy s málo referenciami.                                                                  \\ \hline
8. týždeň         & Sekcia 2.1. práce o monitorovaní vibrácií a šoku.                                                                                                                                                                                             \\ \hline
9. týždeň         & Doplnenie typov akcelerometrov a časti o numerickej kvadratúre k sekcii 2.1. Úvod do sekcie 2.2. o analýze v časovej doméne.                                                                       \\ \hline
10. týždeň        & Deskriptívne štatistiky a algoritmy na identifikáciu špičiek.                                                                                                                                                                                 \\ \hline
11. týždeň        & Sekcia 2.2. o frekvenčnej a časovo-frekvenčnej analýze signálu.                                                                                                                                                                               \\ \hline
12. týždeň        & Sekcia 2.3 o architektúre senzorových sietí a ich obmedzeniach. Návrh riešenia a úvod k priebežnej správe BP1. \\ \hline
13. týždeň        & Zapracované pripomienky k prezentovanému návrhu.                                                                                                                                                                             \\ \hline
\end{tabular}
\end{table}

Rozvrhnutie pred začiatkom zimného semestra sa držalo dvoch oporných termínov a síce 6. týždňa a 12. týždňa.
V 6. týždni sme chceli zavŕšiť rešerš podstatných zdrojov literatúry podľa predstavy o charaktere vibračných signálov
nadobutnutých aj prieskumnými meraniami. V druhej polovici semestra sme tak vedeli zostaviť osnovu a každý týždeň
sa venovať jednej sekcii analýzy až do 12. týždňa.

\clearpage
\newpage


\section{Letný semester}
\begin{table}[h!]
\def\arraystretch{1.25}
\begin{tabular}{|l|p{12cm}|}
\hline
\textbf{Obdobie} & \textbf{Náplň práce}                                                                                                                                            \\ \hline
1. týždeň        & Tvorba generátora syntetického signálu s mechanizmom vyhodnocovania metrík  klasifikácie detektorov.                                                            \\ \hline
2. týždeň        & Pripravenie vývojového prostredia s ESP-IDF SDK a výber  vhodných knižníc  pre DSP a Message Pack.                                                              \\ \hline
3. týždeň        & Odlaďovanie ovládania hardvérových periférií: akcelerometer, pripojenie na WiFi.  Návrh krokov dátovej pipeline.                                                \\ \hline
4. týždeň        & Zakomponovanie posielania vzoriek cez MQTT. Validácia na syntetických dátach rozdelených na trénovaciu a testovaciu sadu.                                       \\ \hline
5. týždeň        & Implementácia kostry dátovej pipeline na IoT zariadenie. Message Pack serializácia  konfigurácie a jej publikovanie  cez MQTT. \\ \hline
6. týždeň        & Parser prijatej konfigurácie, uloženie a aplikácia nastavení na zariadení.  Optimalizácia alokovania dostupnej pamäte.                                          \\ \hline
7. týždeň        & Návrh algoritmu na identifikáciu udalostí. Jednoduché jednotkové testy na validáciu funkčnosti systému a prvotné výkonnostné testy. Tvorba doxygen dokumentácie. \\ \hline
8. týždeň        & Experimentálne merania pamäťovej a časovej efektivity. Vyhodnocovanie úspešnosti hľadania špičiek podľa hyperparametrov. \\ \hline
9. týždeň        & Ilustrácie a diagramy zahrnuté do kapitoly návrhu. \\ \hline
10. týždeň       & Písanie textu 3. kapitoly ,,Návrh riešenia'' a 4. kapitoly ,,Implementácia''. \\ \hline
11. týždeň       & Písanie textu, vyhotovenie grafov a tabuliek pre zvyšné kapitoly  hlavnej časti práce. \\ \hline
12. týždeň       & Doplnenie príloh práce, najmä technickej dokumentácie a  používateľskej príručky.                                                                                 \\ \hline
13. týždeň       & Prezentácia celkového vypracovania vedúcemu práce a zapracovanie pripomienok.                                                                                                           \\ \hline
\end{tabular}
\end{table}

Pôvodný plán vychádzal z trojtýždenných cyklov, kde po každom by kompletná daná časť systému, v skutočnosti sa
prirodzene prelínali a dopĺňali. Do konca 3. týždňa sme plánovali odladenie modelov na monitorovanie vibrácií na základe
analyzovaných algoritmov a meranie úspešnosti na synteticky generovaných dátach. Do 6.týždňa mala byť funkčný
záznamom udalostí na pamäťovú kartu vo firmvéri. Do 9. týždňa sa mala uskutočniť optimalizácia posielaných dát
a vzdialená konfigurácia. Na posledný beh pripadali experimenty a ich vyhodnotenie, počas ktorých bol už písaný text práce.
Konzultácie raz za dva týždne tvorili kontrolné body, kedy sme konfrontovali plnenie plánu s postupom.
\clearpage


 \thispagestyle{empty}
\chapter{Technická dokumentácia}
\pagenumbering{arabic}
\renewcommand*{\thepage}{B-\arabic{page}}

\section{Doxygen dokumentácia}
Nástroj Doxygen zhotovil podľa komentárov v zdrojom kóde prehľadnú technickú dokumentáciu,
ktorá je po typografickej úprave súčasťou tejto prílohy.

\subsection{Moduly}
\begin{itemize}[noitemsep, topsep=0pt]
	\item \textbf{Udalosti} (\ref{modules:events}) - Binárne klasifikátory na označenie význačných úrovní v posuvnom okne vzoriek. 
	Zdrojový kód: \verb|events.h|, \verb|events.c|.
	\item \textbf{Akcelerometer} (\ref{modules:imu}) - Adaptér pre SPI rozhranie senzora LSM9DS1 lineárnej 3D akcelerácie (IMU).
	Zdrojový kód: \verb|inertial_unit.h|, \verb|inertial_unit.c|
	\item \textbf{Hardvérové adaptéry} (\ref{modules:hardware}) - Rozhrania na komunikáciu s perifériami.
	Zdrojový kód: \verb|peripheral.h|, \verb|peripheral.c|
	\item \textbf{Dátová pipeline} (\ref{modules:pipeline}) - Fázy spracovania zdrojového signálu. Zdrojový kód: \verb|pipeline.h|
	\begin{itemize}[noitemsep, topsep=0pt]
		\item \textbf{Oknové funkcie} - \verb|window.c|
		\item \textbf{Správa pamäti pipeline} - \verb|pipeline.c|
		\item \textbf{Fázy spracovania oknovaného signálu} - \verb|pipeline.c|
		\item \textbf{Message Pack serializácia} - Serializácia a parsovanie nameraných dát a konfigurácie. \verb|serialize.c|
	\end{itemize}
	\item \textbf{Deskriptívna štatistika} (\ref{modules:statistics}) - výpočet popisných štatistík. Zdrojový kód: \verb|statistics.h|, \verb|statistics.c|
\end{itemize}



\subsection{Udalosti} \label{modules:events}
Modul s binárnymi klasifikátormi na označenie význačných úrovní v posuvnom okne vzoriek.

\subsubsection*{Dátové štruktúry}
\textbf{struct SpectrumEvent} - Stav udalosti frekvenčného vedierka.
\begin{itemize}[noitemsep, topsep=0pt]
	\item \textbf{SpectrumEventAction action}: Značka vymedzenia udalosti: žiadna, začiatok, koniec
	\item \textbf{uint32\_t start}: Časová pečiatka alebo poradie posuvného okna, kedy začala aktuálne aktívna udalosť
	\item \textbf{uint32\_t duration}: Trvanie aktívnej udalosti v počte posuvných okien
	\item \textbf{int32\_t last\_seen}: Počet posuvných okien do minulosti, kedy bol detegovaný posledný výskyt špičky vo frekvencii
	\item \textbf{float amplitude}: riemerná amplitúda frekvenčného vedierka počas trvania udalosti
\end{itemize}

\subsubsection*{Enumerácie}
\textbf{enum SpectrumEventAction} - Časové vymedzenie udalosti frekvenčného spektra.
	\begin{itemize}[noitemsep, topsep=0pt, label=$\star$]
		\item \verb|SPECTRUM_EVENT_NONE|: Udalosť prebieha alebo žiadna nie je aktívna.
		\item \verb|SPECTRUM_EVENT_START|: Značka začiatku udalosti.
		\item \verb|SPECTRUM_EVENT_FINISH|: Značka ukončenia udalosti.
	\end{itemize}

\subsubsection*{Funkcie}
\begin{lstlisting}[style=docs]
void find_peaks_above_threshold (
    bool *peaks, const float *y, int n, float t
) 
\end{lstlisting}
    Hľadanie špičiek absolútnou prahovou úrovňou amplitúdy signálu. \\
\textbf{Parametre}:
\begin{itemize}[noitemsep, topsep=0pt, label=$\star$]
	\item \textbf{peaks} (out): Nájdené špičky v signále. Dĺžka poľa musí byť rovnaká počtu vzoriek
	\item \textbf{y} (in): Vzorky signálu 
	\item \textbf{n} (in): Počet vzoriek 
	\item \textbf{t} (in) Prahová úroveň amplitúdy. Odporúčaná hodnota je z prípustných hodnôt rozsahu pre vzorky 
\end{itemize}
\bigbreak
\hrule

\begin{lstlisting}[style=docs]
void find_peaks_neighbours (
    bool *peaks, const float *y, int n,  int k, 
    float e, float h_rel, float h
)
\end{lstlisting}
   Hľadanie špičiek s algoritmom význačnosti vrchola spomedzi susedov. $ f[t-i] < f[t] > f[t+i],\quad \forall i \in 1, 2, ..., k $ \\
\textbf{Parametre}:
\begin{itemize}[noitemsep, topsep=0pt, label=$\star$]
	\item \textbf{peaks} (out): Nájdené špičky v signále. Dĺžka poľa musí byť rovnaká počtu vzoriek
	\item \textbf{y} (in): Vzorky signálu 
	\item \textbf{n} (in): Počet vzoriek 
	\item \textbf{k} (in): Počet najbližších uvažovaných susedov na každú zo strán od kandidátnej špičky $[t-k; t+k]$;  Rozsah: $[1, n / 2]$
	\item \textbf{e} (in): Relatívna tolerancia pre vyššiu úroveň v susedstve od vrchola
	\item \textbf{h\_rel} (in): Minimálna relatívna výška špičky v susedstve
 	\item \textbf{h} (in): Absolútna prahová úroveň amplitúdy špičky
\end{itemize}
\bigbreak
\hrule

\begin{lstlisting}[style=docs]
void find_peaks_zero_crossing (
	bool *peaks, const float *y, int n, int k, float slope
)
\end{lstlisting}
   Hľadanie špičiek s algoritmom prechodu nulou do záporu. $Delta f[i] = 0$ \\
\textbf{Parametre}:
\begin{itemize}[noitemsep, topsep=0pt, label=$\star$]
	\item \textbf{peaks} (out): Nájdené špičky v signále. Dĺžka poľa musí byť rovnaká počtu vzoriek
	\item \textbf{y} (in): Vzorky signálu 
	\item \textbf{n} (in): Počet vzoriek 
	\item \textbf{k} (in): Dĺžka sečnice na každú stranu od kandidátnej špičky $[t-k; t+k]$; Rozsah: $[1, n / 2]$
	\item \textbf{slope} (in): Prahová úroveň strmosti kopca, čiže rozdielu medzi hladiny medzi koncami sečnice. Rozsah: $slope \geq 0 $
\end{itemize}
\bigbreak
\hrule

\begin{lstlisting}[style=docs]
void find_peaks_hill_walker (
	bool *peaks, const float *y, int n, 
	float tolerance, int hole, 
	float prominence, float isolation
)
\end{lstlisting}
   Hľadanie špičiek s modikovaným algoritmom horského turistu. \\
\textbf{Parametre}:
\begin{itemize}[noitemsep, topsep=0pt, label=$\star$]
	\item \textbf{peaks} (out): Nájdené špičky v signále. Dĺžka poľa musí byť rovnaká počtu vzoriek
	\item \textbf{y} (in): Vzorky signálu 
	\item \textbf{n} (in): Počet vzoriek 
	\item \textbf{tolerance} (in): Prahová úroveň vo vertikálnej osi. Rozsah: $[min(y), max(y)]$
 	\item \textbf{hole} (in): Prahová úroveň v horizontálnej osi. Rozsah: $[0, n]$
 	\item \textbf{prominence} (in):  Relatívna výška oproti predošlej navštívenej doline. Rozsah: $[0, \max(y) - \min(y)]$
 	\item \textbf{isolation} (in)  Vzdialenosť ku najbližšiemu predošlému vrcholu. Rozsah: $[0, n]$
\end{itemize}
\bigbreak
\hrule

\begin{lstlisting}[style=docs]
void event_init (SpectrumEvent *events, uint16_t bins)
\end{lstlisting}
   Nastavenie počiatočného stavu online detektora udalostí vo frekvenciách. \\
\textbf{Parametre}:
\begin{itemize}[noitemsep, topsep=0pt, label=$\star$]
	\item \textbf{events} (out): Pole udalostí frekvenčného spektra
	\item \textbf{bins} (in): Počet frekvenčných vedierok a zároveň dĺžka poľa udalostí
\end{itemize}
\bigbreak
\hrule

\begin{lstlisting}[style=docs]
size_t event_detection (
	size_t t, SpectrumEvent *events, const bool *peaks, 
	const float *spectrum, uint16_t bins,
	uint16_t min_duration, uint16_t time_proximity
)
\end{lstlisting}
   Online detekcia zmien v časovom priebehu frekvenčných spektier. \\
\textbf{Parametre}:
\begin{itemize}[noitemsep, topsep=0pt, label=$\star$]
\item \textbf{t}: Poradové číslo posuvného okna. S každým ďalším volaním funkcie musí byť navýšené o 1 
\item \textbf{events}: Pole udalostí frekvenčného spektra s dĺžkou počtu vedierok
\item \textbf{peaks} (in): Nájdené špičky vo aktuálnom frekvenčnom spektre jedným z klasifikátorov find\_peak\_*
\item \textbf{spectrum} (in): Frekvenčné spektrum aktuálneho posuvného okna vzoriek na zistenie priemernej amplitúdy udalostí
\item \textbf{bins} (in): Počet frekvenčných vedierok
\item \textbf{min\_duration} (in): Minimálne trvanie po koľkých oknách je vyhlásená špička za udalosť. Udáva oneskorenie vyhlásenia začiatku udalosti.
\item \textbf{time\_proximity} (in): Najväčšia vzdialenosť súvislej udalosti v počte okien. Najväčšia dĺžka časovej medzery medzi nájdenými špičkami. Udáva oneskorenie vyhlásenia ukončenia udalosti.
 
\item \textbf{Návratová hodnota}: Počet detegovaných zmien, čiže začiatočných a koncov udalostí v danom spektre posuvného okna
\end{itemize}
\bigbreak
\hrule



\subsection{Akcelerometer} \label{modules:imu}
Modul adaptéra pre SPI rozhranie senzora LSM9DS1 lineárnej 3D akcelerácie (IMU)

\subsubsection*{Dátové štruktúry}
\textbf{struct InertialUnit} - Inerciálna meracia jednotka.
\begin{itemize}[noitemsep, topsep=0pt]
	\item \textbf{gpio\_num\_t clk}: GPIO pin SPI hodinového signálu
	\item \textbf{gpio\_num\_t miso}: GPIO pin SPI Master In Slave Out
	\item \textbf{gpio\_num\_t xgcs}: GPIO pin SPI Chip select akcelometra a gyroskopu
	\item \textbf{gpio\_num\_t mcs}: GPIO pin SPI Chip select magnetometra
	\item \textbf{gpio\_num\_t int1}: GPIO vstup prerušenia č.1
	\item \textbf{gpio\_num\_t int2}: GPIO vstup prerušenia č.2
	\item \textbf{gpio\_num\_t en\_data}: GPIO vstup príznaku pripravených dát
	\item \textbf{gpio\_num\_t isr\_int1}: Podprogram prerušenia pre INT č.1
	\item \textbf{gpio\_num\_t isr\_int2}: Podprogram prerušenia pre INT č.2
	\item \textbf{spi\_host\_device\_t spi}: SPI zbernica
	\item \textbf{spi\_device\_handle\_t dev}: SPI zariadenie pre akcelerometer
	\item \textbf{AccelerationPrecision precision}: Zistená citlivosť akcelerometra v mg/LSB
\end{itemize}

\subsubsection*{Enumerácie}
\textbf{enum AccelerationRange} - Dynamický rozsah akcelerometra v g.
	\begin{itemize}[noitemsep, topsep=0pt, label=$\star$]
		\item \verb|IMU_2G|: Rozsah $\pm 2\;\mathrm{g}  = \pm 19.6133 \;\mathrm{m/s^2}$
		\item \verb|IMU_4G|: Rozsah $ \pm 4\;\mathrm{g} = \pm 39.2266 \;\mathrm{m/s^2}$
		\item \verb|IMU_8G|: Rozsah $ \pm 8\;\mathrm{g}  = \pm 78.4532 \;\mathrm{m/s^2}$
		\item \verb|IMU_16G|: Rozsah $ \pm 16\;\mathrm{g}  = \pm 156.9064 \;\mathrm{m/s^2}$
		\item \verb|IMU_RANGE_COUNT|: Počet možností nastavenia rozsahu na účely serializácie
	\end{itemize}
\bigbreak
\noindent\textbf{typedef float AccelerationPrecision} - Citlivosť akcelerometra podľa dynamického rozsahu v mg/LSB.
	

\subsubsection*{Funkcie}

\begin{lstlisting}[style=docs]
esp_err_t imu_setup(InertialUnit *imu)
\end{lstlisting}
   Inicializácia senzora lineárnej akcelerácie. \\
\textbf{Parametre}:
\begin{itemize}[noitemsep, topsep=0pt, label=$\star$]
	\item \textbf{imu}: Senzor
	\item \textbf{Návratová hodnota}: Úspešnosť inicializácie senzora
\end{itemize}
\bigbreak
\hrule

\begin{lstlisting}[style=docs]
void imu_acceleration_range(
	InertialUnit *imu, AccelerationRange range
)
\end{lstlisting}
   Nastavenie dynamického rozsah lineárneho 3D akcelerometra v g. \\
\textbf{Parametre}:
\begin{itemize}[noitemsep, topsep=0pt, label=$\star$]
	\item \textbf{imu}: Senzor
	\item \textbf{range}: Dynamický rozsah akcelerometra
\end{itemize}
\bigbreak
\hrule

\begin{lstlisting}[style=docs]
void imu_output_data_rate(InertialUnit *imu, uint16_t fs)
\end{lstlisting}
   Nastavenie výstupného dátového toku (ODR) akcelerometra podľa vzorkovanej frekvencie. \\
\textbf{Parametre}:
\begin{itemize}[noitemsep, topsep=0pt, label=$\star$]
	\item \textbf{imu}: Senzor
	\item \textbf{fs}: Vzorkovacia frekvencia v Hz. Hardvér povoľuje max. ODR 956 Hz
\end{itemize}
\bigbreak
\hrule

\begin{lstlisting}[style=docs]
void imu_output_data_rate(InertialUnit *imu, uint16_t fs)
\end{lstlisting}
   Meranie aktuálnej hodnoty 3D akcelerácie v $m/s^2$. \\
\textbf{Parametre}:
\begin{itemize}[noitemsep, topsep=0pt, label=$\star$]
	\item \textbf{imu}: Senzor
	\item \textbf{x} (out): Zrýchlenie v osi x. Rozsah je podľa nastavenia dynamického rozsahu.
 	\item \textbf{y} (out): Zrýchlenie v osi y
    \item \textbf{z} (out): Zrýchlenie v osi z
\end{itemize}


\subsection{Hardvérové adaptéry} \label{modules:hardware}

\subsubsection*{Dátové štruktúry}
\textbf{struct OpenLog} - SparkFun OpenLog - zaznenávač údajov na SD kartu cez sériovú linku.
\begin{itemize}[noitemsep, topsep=0pt]
	\item \textbf{gpio\_num\_t vcc}: GPIO pin na ovládanie napájania cez FET tranzistor
	\item \textbf{uint8\_t uart}: Číslo UART rozhrania
	\item \textbf{gpio\_num\_t rx}: GPIO pin UART RX
	\item \textbf{gpio\_num\_t tx}: GPIO pin UART TX
	\item \textbf{int baudrate}: Baudová rýchlosť komunikácie. Rovnaká rýchlosť musí byť nastavená v \verb|config.txt| na SD karte
	\item \textbf{int buffer}: Dĺžka vyrovnávacej pamäte pre UART
\end{itemize}
\bigbreak

\noindent\textbf{struct MqttAxisTopics} - MQTT témy pre os zrýchlenia.
\begin{itemize}[noitemsep, topsep=0pt]
	\item \textbf{char stats[TOPIC\_LENGTH]}: Názov témy pre štatistické údaje
	\item \textbf{char spectra[TOPIC\_LENGTH]}: Názov témy pre frekvenčné spektrum
	\item \textbf{char events[TOPIC\_LENGTH]}: Názov témy pre udalosti zmeny spektra
\end{itemize}


\subsubsection*{Funkcie}
\begin{lstlisting}[style=docs]
void axis_mqtt_topics (MqttAxisTopics *topics, int axis)
\end{lstlisting}
   Poskladanie názvu MQTT tém pre odosieanie dát o osi akcelerácie. \\
\textbf{Parametre}:
\begin{itemize}[noitemsep, topsep=0pt, label=$\star$]
	\item topics (out): MQTT témy zložené s označením osi x, y, z
	\item axis (in): Index osi vektora akcelerácie: 0, 1, 2 
\end{itemize}
\bigbreak
\hrule

\begin{lstlisting}[style=docs]
void clock_reconfigure (uint16_t frequency)
\end{lstlisting}
   Zmena frekvencie časovača. \\
\textbf{Parametre}:
\begin{itemize}[noitemsep, topsep=0pt, label=$\star$]
	\item \textbf{frequency} (in):	Vzorkovacia frekvencia v Hz 
\end{itemize}
\bigbreak
\hrule

\begin{lstlisting}[style=docs]
void clock_setup (uint16_t frequency, timer_isr_t action)
\end{lstlisting}
   Spustenie časovača na vzorkovanie signálu. \\
\textbf{Parametre}:
\begin{itemize}[noitemsep, topsep=0pt, label=$\star$]
	\item \textbf{frequency} (in): Vzorkovacia frekvencia v Hz
	\item \textbf{action} (in):	Obsluha prerušenia časovača s predpisom: \\ \verb|bool IRAM_ATTR f(void *args)|
\end{itemize}
\bigbreak
\hrule

\begin{lstlisting}[style=docs]
void mqtt_event_handler (
	void *handler_args, esp_event_base_t base, 
	int32_t event_id, void *event_data
)
\end{lstlisting}
Predbežná deklarácia spätného volania. Implementáciu musí poskytnúť hlavný program. Používa sa v \verb|mqtt_setup()|. \\
\bigbreak
\hrule

\begin{lstlisting}[style=docs]
esp_mqtt_client_handle_t mqtt_setup (
	const char *broker_url
)
\end{lstlisting}
   Pripojenie sa k MQTT broker a zaregistrovanie spätného volanie pre všetky udalosti. \\
\textbf{Parametre}:
\begin{itemize}[noitemsep, topsep=0pt, label=$\star$]
	\item \textbf{broker\_url}: URL MQTT broker 
\end{itemize}
\bigbreak
\hrule

\begin{lstlisting}[style=docs]
esp_err_t nvs_load (
	Configuration *conf, Provisioning *login
)
\end{lstlisting}
   Načítanie nastavení systému z nevolatilného úložiska. \\
\textbf{Parametre}:
\begin{itemize}[noitemsep, topsep=0pt, label=$\star$]
	\item \textbf{conf} (out): Globálne nastavenia spracovania dát
	\item \textbf{login} (out): Nastavenie sieťového pripojenia
\end{itemize}
\bigbreak
\hrule

\begin{lstlisting}[style=docs]
esp_err_t nvs_save_config (const Configuration *conf)
\end{lstlisting}
   Uloženie nastavení systému na nevolatilné úložisko. \\ Používa sa v \verb|mqtt_event_handler()|. \\
\textbf{Parametre}:
\begin{itemize}[noitemsep, topsep=0pt, label=$\star$]
	\item \textbf{conf} (in): Globálne nastavenia spracovania dát
\end{itemize}
\bigbreak
\hrule

\begin{lstlisting}[style=docs]
esp_err_t nvs_save_login (const Provisioning *login)
\end{lstlisting}
   Uloženie nastavení sieťového pripojenia na nevolatilné úložisko. \\ Používa sa v \verb|mqtt_event_handler()|. \\
\textbf{Parametre}:
\begin{itemize}[noitemsep, topsep=0pt, label=$\star$]
	\item \textbf{login} (in): Nastavenie sieťového pripojenia 
\end{itemize}
\bigbreak
\hrule

\begin{lstlisting}[style=docs]
void openlog_setup (OpenLog *logger)
\end{lstlisting}
Nastavenie UART rozhrania pre zariadenie OpenLog.
\bigbreak
\hrule

\begin{lstlisting}[style=docs]
void wifi_connect (wifi_config_t *wifi_config)
\end{lstlisting}
Pripojenie sa k Wifi AP blokujúce.


\subsection{Dátová pipeline} \label{modules:pipeline}

\subsubsection*{Dátové štruktúry}
\noindent\textbf{struct SamplingConfig} - Nastavenie vzorkovania signálu.
\begin{itemize}[noitemsep, topsep=0pt]
	\item \textbf{uint16\_t frequency}: Vzorkovacia frekvencia v Hz. Najviac \verb|MAX_FREQUENCY|
	\item \textbf{AccelerationRange range}: Dynamický rozsah akcelerometra
	\item \textbf{uint16\_t n}: Veľkosť posuvného okna. Musí byť mocninou dvojky a najviac \verb|MAX_BUFFER_SAMPLES|
	\item \textbf{float overlap}: Pomer prekryvu posuvných okien. Rozsah: 0 až \verb|MAX_OVERLAP|
	\item \textbf{bool axis[AXIS\_COUNT]}: Osi akcelerácie povolené na spracovanie
\end{itemize}
\bigbreak

\noindent\textbf{struct SmoothingConfig} - Nastavenie vyhladzovania časovo premenného signálu alebo frekvenčného spektra.
\begin{itemize}[noitemsep, topsep=0pt]
	\item \textbf{bool enable}: Vyhladzovanie signálu povolené
	\item \textbf{uint16\_t n}: Dĺžka konvolučnej masky
	\item \textbf{uint8\_t repeat}: Počet prechodov konvolučnej masky. Najviac \verb|MAX_SMOOTH_REPEAT|
\end{itemize}
\bigbreak

\noindent\textbf{struct StatisticsConfig} - Nastavenie zberu štatistík posuvného okna.
\begin{itemize}[noitemsep, topsep=0pt]
	\item \textbf{bool min}: Výpočet minimálnej hodnoty povolený
	\item \textbf{bool max}: Výpočet maximálnej hodnoty povolený
	\item \textbf{bool rms}: Výpočet strednej kvadratickej odchýlky povolený
	\item \textbf{bool mean}: Výpočet aritmetického priemeru povolený
	\item \textbf{bool variance}: Výpočet rozptylu povolený
	\item \textbf{bool std}: Výpočet smerodajnej odchýlky povolený
	\item \textbf{bool skewness}: Výpočet šikmosti povolený
	\item \textbf{bool kurtosis}: Výpočet špicatosti povolený
	\item \textbf{bool median}: Výpočet mediánu povolený
	\item \textbf{bool mad}: Výpočet mediánovej absolútnej odchýlky povolený
	\item \textbf{bool correlation}: Výpočet korelácie medzi osami povolený
\end{itemize}
\bigbreak

\noindent\textbf{struct FFTTransformConfig} - Nastavenie frekvenčnej transformácie.
\begin{itemize}[noitemsep, topsep=0pt]
	\item \textbf{WindowTypeConfig window}: Oknová funkcia
	\item \textbf{FrequencyTransform func}: Typ frekvenčnej transformácie
	\item \textbf{bool log}: Prevod magnitúdy frekvencie do dB
\end{itemize}
\bigbreak

\noindent\textbf{struct SaveFormatConfig} - Nastavenia ukladania a posielania spracovaných dát.
\begin{itemize}[noitemsep, topsep=0pt]
	\item \textbf{bool local}: Záznam vzoriek na SD kartu povolený. OpenLog bude zapnutý po spustení
	\item \textbf{bool mqtt}: Posielanie cez MQTT povolené. Wifi a MQTT klient bude zapnutý po spustení. Pozor: po deaktivácii
	sa zariadenie nedá vzdialene rekonfigurovať. Na znovu povolenie sa musí nahrať firmvér so touto možnosťou povolenou. 
	\item \textbf{bool mqtt\_stats}: Odosielanie štatistík cez MQTT na topic podľa \verb|MQTT_TOPIC_STATS|
	\item \textbf{bool mqtt\_events}:  Odosielanie zmien spektra cez MQTT na topic podľa \verb|MQTT_TOPIC_EVENT|
	\item \textbf{SendUnprocessed mqtt\_samples}:  Odosielanie nespracovaných vzoriek alebo frekvencií cez MQTT na topic podľa \verb|MQTT_TOPIC_STREAM|, \verb|MQTT_TOPIC_SPECTRUM|.
	\item \textbf{uint16\_t subsampling}: Podvzorkovanie pre záznam vzoriek bez ďalšieho spracovania. Preskočí sa každých \verb|subsampling| vzoriek
\end{itemize}
\bigbreak

\noindent\textbf{struct FFTTransformConfig} - Nastavenia algoritmov na detekciu udalostí a ich parametrov (popis s nachádza pri 
funkciách z modulu ,,Udalosti'' \ref{modules:events}.
\bigbreak

\noindent\textbf{struct Configuration} - Systémová konfigurácia pipeline spracovania vzoriek z akcelerometra.
\bigbreak

\noindent\textbf{struct Provisioning} - Sieťové nastavenia pre pripojenie na Wifi AP s WPA2 a MQTT broker.
\bigbreak

\noindent\textbf{struct Correlation} - Medzivýsledky korelácie zdieľanej všetkými osami spracovania s prístupom
cez zahrnutú synchronizačnú bariéru.
\bigbreak

\noindent\textbf{struct Statistics} - Výsledky všetkých dostupných štatistík.
\bigbreak

\noindent\textbf{struct BufferPipelineKernel} - Vyrovnávacie pamäte spoločné pre celú pipeline.
\bigbreak

\noindent\textbf{struct BufferPipelineAxis} - Vyrovnávacie pamäte samostatné pre každú os akcelerácie.
\bigbreak

\noindent\textbf{struct Sender} - Fronta pre záznam vzoriek bez ďalšiho spracovania.
\bigbreak

\subsubsection*{Konštanty}
V zátvorkách sú uvedené predvolené hodnoty
\begin{itemize}[noitemsep, topsep=0pt]
	\item \verb|AXIS_COUNT|: Celkový počet osí akcelerácie (3)
	\item \verb|MAX_MPACK_FIELDS_COUNT|: Maximálny počet dvojíc ,,kľúč - hodnota'' v Message Pack slovníku (20)
	\item \verb|SAMPLES_QUEUE_SLOTS|: Násobok veľkosti posuvného okna ako počet vzoriek čakajúcich na spracovanie vo fronte (3)
	\item \verb|MAX_CREDENTIALS_LENGTH|: Maximálna dĺžka prihlasovacieho údaju Wifi pripojenia (64)
	\item \verb|MAX_MQTT_URL|: Dĺžka URL adresy na MQTT broker (256)
	\item \verb|MAX_BUFFER_SAMPLES|: Najdlhšie povolené posuvné okno, vyššia mocnina 2 ako 1024 sa nezmestí do DRAM (1024)
	\item \verb|MAX_FREQUENCY|: Najvyššia vzorkovacia frekvencia daná fyzickým obmedzením akcelerometra (952)
	\item \verb|MAX_OVERLAP|:  Maximálny prekryv posuvných okien (0.8)
	\item \verb|MAX_SMOOTH_REPEAT|:  Maximálny počet prechodu konvolučnej masky vyhladzovacieho filtra (8)
	\item \verb|LARGEST_MESSAGE|: Najväčšia veľkosť vyrovnávacej pamäte pre serializáciu vzoriek (14000)
	\item \verb|LARGEST_CONFIG|: Najväčšia veľkosť serializovanej konfigurácie (480)
\end{itemize}


\subsubsection*{Enumerácie}

\noindent\textbf{enum WindowTypeConfig} - Oknové funkcie.
	\begin{itemize}[noitemsep, topsep=0pt, label=$\star$]
		\item \verb|BOXCAR_WINDOW|: Obdĺžníkové okno
		\item \verb|BARTLETT_WINDOW|: Bartlettovo okno
		\item \verb|HANN_WINDOW|: Hannovo okno
		\item \verb|HAMMING_WINDOW|: Hammingovo okno
		\item \verb|BLACKMAN_WINDOW|: Blackmanovo okno
		\item \verb|WINDOW_TYPE_COUNT|: Počet dostupných oknových funkcií. Potrebné pre serializáciu.
	\end{itemize}
\bigbreak

\noindent\textbf{enum PeakFindingStrategy} - Algoritmy na hľadanie špičiek.
	\begin{itemize}[noitemsep, topsep=0pt, label=$\star$]
		\item \verb|THRESHOLD|: Špičky nad prahovou úrovňou.
		\item \verb|NEIGHBOURS|: Špičky najvýznačnejšieho bodu spomedzi susedov.
		\item \verb|ZERO_CROSSING|: Špičky prechodou nulou do záporu.
		\item \verb|HILL_WALKER|: Špičky algoritmom horského turistu. 
		\item \verb|STRATEGY_COUNT|: Počet možností na účely serializácie
	\end{itemize}
\bigbreak

\noindent\textbf{enum FrequencyTransform} - Frekvenčné transformácie.
	\begin{itemize}[noitemsep, topsep=0pt, label=$\star$]
		\item \verb|DFT|: Rýchla Fourierová transformácia radix-2
		\item \verb|DCT|: Konsínusová transformácia DCT II
		\item \verb|TRANSFORM_COUNT|: Počet dostupných frekvenčných transformácií. Potrebné pre serializáciu
	\end{itemize}
\bigbreak

\noindent\textbf{enum SendUnprocessed} - Doména odosielaných nespracovaných vzoriek.
	\begin{itemize}[noitemsep, topsep=0pt, label=$\star$]
		\item \verb|RAW_NONE_SEND|: Žiadne nespracované vzorky 	
		\item \verb|RAW_TIME_SEND|: Nespracované vzorky v časovej oblasti 	
		\item \verb|RAW_FREQUENCY_SEND|: Nespracované vzorky vo frekvenčnej oblasti 
		\item \verb|SEND_UNPROCESSED_COUNT|: Počet možností na účely serializácie
	\end{itemize}
\bigbreak


\subsubsection*{Oknové funkcie}
\textbf{Parametre všetkých oknových funkcií}:
\begin{itemize}[noitemsep, topsep=0pt, label=$\star$]
	\item w (out): Váhy oknovej funkcie
	\item n (in): Dĺžka okna  
\end{itemize}

\hrule
\begin{lstlisting}[style=docs]
void bartlett_window (float *w, int n)
\end{lstlisting}
   Bartlettovo okno. $w(n) = \frac{2}{N - 1}\left(\frac{N - 1}{2} - \left|n - \frac{N - 1}{2} \right|\right)$
\bigbreak
\hrule

\begin{lstlisting}[style=docs]
void blackman_window (float *w, int n)
\end{lstlisting}
   Blackmanovo okno. $w(n) = 0.42 - 0.5\cos(2\pi n / N) + 0.08\cos(4\pi n / N)$
\bigbreak
\hrule

\begin{lstlisting}[style=docs]
void boxcar_window (float *w, int n)
\end{lstlisting}
   Obdĺžníkové okno.  $w(n) = 1 $
\bigbreak
\hrule

\begin{lstlisting}[style=docs]
void hamming_window (float *w, int n)
\end{lstlisting}
   Hammingovo okno.  $w(n) = 0.54 - 0.46\cos(2\pi n / N)$
\bigbreak
\hrule

\begin{lstlisting}[style=docs]
void hann_window(float *w, int n)
\end{lstlisting}
   Hannovo okno.  $w(n) = \sin^2(\pi n / N)$
\bigbreak
\hrule

\begin{lstlisting}[style=docs]
void mean_kernel (float *w, int n)
\end{lstlisting}
   Vyhladzovací filter kĺzavého priemeru.  $w(n) = \frac{1}{n},\, n = 0, 1, ..., N - 1$
\bigbreak
\hrule

\begin{lstlisting}[style=docs]
void window (WindowTypeConfig type, float *w, int n)
\end{lstlisting}
   Oknová funkcia podľa voľby. \\ 
\textbf{Parametre}
\begin{itemize}[noitemsep, topsep=0pt, label=$\star$]
	\item type (out): Oknová funkcia
	\item w (out): Váhy oknovej funkcie
	\item n (in): Dĺžka okna  
\end{itemize}
\bigbreak
\hrule

\subsubsection*{Správa pamäti pipeline}

\begin{lstlisting}[style=docs]
void axis_allocate (BufferPipelineAxis *p, const Configuration *conf)
\end{lstlisting}
   Alokácia dynamických vyrovnávacích pamätí pre jednu os akcelerácie. \\ 
\textbf{Parametre}:
\begin{itemize}[noitemsep, topsep=0pt, label=$\star$]
	\item \textbf{p} (out): Dynamické vyrovnávacie pamäte s dĺžkami podľa nastavení
	\item \textbf{conf} (in): Konfigurácia systému  
\end{itemize}
\bigbreak
\hrule

\begin{lstlisting}[style=docs]
void axis_release (BufferPipelineAxis *p)
\end{lstlisting}
   Uvoľnenie pamäte pre dynamické vyrovnávacie pamäte pre jednu os akcelerácie. \\ 
\textbf{Parametre}:
\begin{itemize}[noitemsep, topsep=0pt, label=$\star$]
	\item \textbf{p} (out): Dynamické vyrovnávacie pamäte
\end{itemize}
\bigbreak
\hrule

\begin{lstlisting}[style=docs]
void process_allocate (
	BufferPipelineKernel *p, const Configuration *conf
)
\end{lstlisting}
   Alokácia a inicializácia dynamických vyrovnávacích pamätí a synchronizačných primitív. \\ 
\textbf{Parametre}:
\begin{itemize}[noitemsep, topsep=0pt, label=$\star$]
	\item \textbf{p} (out): Dynamické vyrovnávacie pamäte s dĺžkami podľa nastavení
	\item \textbf{conf} (in): Konfigurácia systému  
\end{itemize}
\bigbreak
\hrule

\begin{lstlisting}[style=docs]
void process_release (BufferPipelineKernel *p)
\end{lstlisting}
   Uvoľnenie pamäte pre dynamické vyrovnávacie pamäte. \\ 
\textbf{Parametre}:
\begin{itemize}[noitemsep, topsep=0pt, label=$\star$]
	\item \textbf{p} (out): Dynamické vyrovnávacie pamäte
\end{itemize}
\bigbreak
\hrule

\begin{lstlisting}[style=docs]
void axis_release (BufferPipelineAxis *p)
\end{lstlisting}
   Uvoľnenie pamäte pre dynamické vyrovnávacie pamäte pre jednu os akcelerácie. \\ 
\textbf{Parametre}:
\begin{itemize}[noitemsep, topsep=0pt, label=$\star$]
	\item \textbf{p} (out): Dynamické vyrovnávacie pamäte
\end{itemize}
\bigbreak
\hrule

\begin{lstlisting}[style=docs]
void sender_release (Sender *sender)
\end{lstlisting}
   Odstránenie fronty na odosielanie nameraných vzoriek. \\ 
\textbf{Parametre}
\begin{itemize}[noitemsep, topsep=0pt, label=$\star$]
	\item \textbf{sender}: Fronta s vyhradenou kapacitou
\end{itemize}
\bigbreak
\hrule


\subsubsection*{Fázy spracovania oknovaného signálu}

\begin{lstlisting}[style=docs]
void buffer_shift_left(
	float *buffer, uint16_t n, uint16_t k
)
\end{lstlisting}
Posun vzoriek vo vyrovnávacej pamäti doľava, čím sa dosahuje prekryv okien. 
Nadbytočné hodnoty od začiatku poľa budú nahradené vzorkami o \verb|k| pozícii vpravo. \\ 
\textbf{Parametre}:
\begin{itemize}[noitemsep, topsep=0pt, label=$\star$]
	\item \textbf{buffer}: Vyrovnávacia pamäť, ktorej obsah bude posunutý
 	\item \textbf{n} (in): Dĺžka vyrovnávacej pamäte
	\item \textbf{k} (in): Počet pozícii o koľko sa majú posunúť hodnoty.
\end{itemize}
\bigbreak
\hrule

\begin{lstlisting}[style=docs]
void process_correlation(
	uint8_t axis, const float *buffer, 
	Statistics *stats, Correlation *corr, 
	const SamplingConfig *c
)
\end{lstlisting}
Korelácia medzi osami akcelerácie: XY, XZ, YZ. Dochádza k bariérovej synchronizácii. 
Úloha pre každú os zrýchlenia si nezávisle prepočíta rozdiely vzoriek od priemeru a smerodajné odchýlky.
Následne dochádza k bariérovej synchronizácii aktívnych osí. Každá úloha si dopočíta všetky korelácie
samostatne.  \\ 
\textbf{Parametre}:
\begin{itemize}[noitemsep, topsep=0pt, label=$\star$]
	\item \textbf{axis} (in): Os akcelerácie: 0, 1, 2
 	\item \textbf{buffer} (in): Posuvné okno vzoriek signálu          
 	\item \textbf{stats} (out): Zistené medzi-osové korelácie
 	\item \textbf{corr} (out): Pomocné polia pre výmenu predspracovaných údajov medzi úlohami (osami)
 	\item \textbf{corr} (in): Nastavenia vzorkovania. Využíva sa dĺžka posuvného okna a povolené osi.
\end{itemize}
\bigbreak
\hrule

\begin{lstlisting}[style=docs]
void process_smoothing(
	float *buffer, float *tmp, uint16_t n, 
	const float *kernel, const SmoothingConfig *c
)
\end{lstlisting}
Vyhladzovanie signálu. \\ 
\textbf{Parametre}:
\begin{itemize}[noitemsep, topsep=0pt, label=$\star$]
	\item \textbf{buffer}: Posuvné okno vzoriek signálu s dĺžkou \verb|n|, ktoré bude vyhladené
	\item \textbf{tmp}: Pomocné pole o dĺžke \verb|n + c.n - 1|
 	\item \textbf{window} (in): Váhy oknovej funkcie  s dĺžkou \verb|n|
 	\item \textbf{n} (in): Dĺžka posuvného okna
 	\item \textbf{kernel} (in): Konvolučná maska vyhladzovania
 	\item \textbf{c} (in): Nastavenia vyhladzovanie
\end{itemize}
\bigbreak
\hrule

\begin{lstlisting}[style=docs]
int process_spectrum(
	float *spectrum, const float *buffer, 
	const float *window, uint16_t n, 
	const FFTTransformConfig *c
)
\end{lstlisting}
Frekvenčné spektrum (FFT, FCT) posuvného okna vzoriek vynásobené váhami oknovej funkcie. \\ 
\textbf{Parametre}:
\begin{itemize}[noitemsep, topsep=0pt, label=$\star$]
	\item \textbf{spectrum} (out): Frekvenčné spektrum s dĺžkou $n / 2$  
	\item \textbf{buffer} (in): Posuvné okno vzoriek signálu s dĺžkou $n$
	\item \textbf{window} (in): Váhy oknovej funkcie  s dĺžkou $n$
	\item \textbf{n} (in): Dĺžka posuvného okna
	\item \textbf{c} (in): Nastavenia frekvenčnej transformácie
\end{itemize}
\bigbreak
\hrule

\begin{lstlisting}[style=docs]
void process_statistics(
	const float *buffer, uint16_t n, 
	Statistics *stats, const StatisticsConfig *c
)
\end{lstlisting}
Požadované štatistiky podľa nastavení. \\ 
\textbf{Parametre}:
\begin{itemize}[noitemsep, topsep=0pt, label=$\star$]
	\item \textbf{buffer} (in): Posuvné okno vzoriek signálu
 	\item \textbf{n} (in): Dĺžka posuvného okna
 	\item \textbf{stats} (out): Deskriptívne štatistiky zo vzoriek posuvného okna. Korektné hodnoty majú len tie povolené v nastaveniach `c`
	\item \textbf{c} (in): Povolenia pre zber vybraných štatistík
\end{itemize}
\bigbreak
\hrule

\begin{lstlisting}[style=docs]
void process_peak_finding(
	bool *peaks, const float *spectrum, 
	uint16_t bins, const EventDetectionConfig *c
)
\end{lstlisting}
Hľadanie špičiek vo frekvenčnom spektre podľa nastavení aktívneho algoritmu. \\ 
\textbf{Parametre}:
\begin{itemize}[noitemsep, topsep=0pt, label=$\star$]
	\item \textbf{peaks} (out): Váhy oknovej funkcie  s dĺžkou \verb|bins|
 	\item \textbf{spectrum} (in):  Frekvenčné spektrum s dĺžkou \verb|bins|  
 	\item \textbf{bins} (in): Počet frekvenčných vedierok
 	\item \textbf{c} (in): Nastavenia spracovania udalostí
\end{itemize}
\bigbreak
\hrule


\subsubsection*{Message Pack serializácia}

\begin{lstlisting}[style=docs]
size_t stream_serialize(
	char *msg, size_t size, 
	const float *stream, size_t n
)
\end{lstlisting}
Serializácia prúdu vzoriek v posuvnom okne do formátu Message Pack. \\ 
\textbf{Parametre}:
\begin{itemize}[noitemsep, topsep=0pt, label=$\star$]
	\item \textbf{msg} (out): Serializované vzorky signálu
	\item \textbf{size} (in): Vyhradená veľkosť pre správu do \verb|msg|
	\item \textbf{stream} (in): Vzorky signálu
	\item \textbf{n} (in): Počet vzoriek signálu
 	\item \textbf{Návratová hodnota}: Dĺžka serializovanej správy
\end{itemize}
\bigbreak
\hrule

\begin{lstlisting}[style=docs]
size_t stats_serialize(
	size_t timestamp, char *msg, size_t size, 
	const Statistics *stats, const StatisticsConfig *c
)
\end{lstlisting}
Serializácia štatistík signálu v posuvnom okne do formátu Message Pack. \\ 
\textbf{Parametre}:
\begin{itemize}[noitemsep, topsep=0pt, label=$\star$]
	\item \textbf{timestamp} (in): Poradové číslo posuvného okna
 	\item \textbf{msg} (out): Serializované štatistiky
 	\item \textbf{size} (in): Vyhradená veľkosť pre správu do \verb|msg|
 	\item \textbf{stats} (in): Štatistiky signálu
 	\item \textbf{n} (in): Nastavenia zberu štatistík. Do serializovanej správy sa zahrnú len aktívne štatistiky.
 	\item \textbf{Návratová hodnota}: Dĺžka serializovanej správy
\end{itemize}
\bigbreak
\hrule

\begin{lstlisting}[style=docs]
size_t spectra_serialize(
	size_t timestamp, char *msg, size_t size, 
	const float *spectrum, size_t n, uint16_t fs
)
\end{lstlisting}
Serializácia frekvenčného spektra posuvného okna do formátu Message Pack. \\ 
\textbf{Parametre}
\begin{itemize}[noitemsep, topsep=0pt, label=$\star$]
 	\item \textbf{timestamp} (in): Poradové číslo posuvného okna
 	\item \textbf{msg} (out): Serializované frekvenčné spektrum
 	\item \textbf{size} (in): Vyhradená veľkosť pre správu do \verb|msg|
 	\item \textbf{spectrum} (in): Frekvenčné spektrum
	\item \textbf{n} (in): Počet frekvenčných vedierok
	\item \textbf{fs} (in): Vzorkovacia frekvencia v Hz
 	\item \textbf{Návratová hodnota}: Dĺžka serializovanej správy
\end{itemize}
\bigbreak
\hrule

\begin{lstlisting}[style=docs]
size_t events_serialize(
	size_t timestamp, float bin_width, 
	char *msg, size_t size, 
	const SpectrumEvent *events, size_t n
)
\end{lstlisting}
Serializácia udalostí zmien frekvenčného spektra do formátu Message Pack. \\ 
\textbf{Parametre}:
\begin{itemize}[noitemsep, topsep=0pt, label=$\star$]
	\item \textbf{timestamp} (in): Poradové číslo posuvného okna
	\item \textbf{bin\_width} (in): Veľkosť frekvenčného vedierka v Hz: $fs / n$
	\item \textbf{msg} (in): Serializované udalosti
	\item \textbf{size} (in): Vyhradená veľkosť pre správu do  \verb|msg|
	\item \textbf{events} (in): Udalosti frekvenčného spektra s dĺžkou  \verb|n|. 
	Do správy budú pridané iba začiatočné a ukončujúce udalosti.
	\item \textbf{n} (in): Počet frekvenčných vedierok
	\item \textbf{Návratová hodnota}: Dĺžka serializovanej správy
\end{itemize}
\bigbreak
\hrule

\begin{lstlisting}[style=docs]
size_t config_serialize(
	char *msg, size_t size, 
	const Configuration *config
)
\end{lstlisting}
Serializácia systémovej konfigurácie do formátu Message Pack. \\ 
\textbf{Parametre}:
\begin{itemize}[noitemsep, topsep=0pt, label=$\star$]
	\item \textbf{msg} (out): Serializovaná konfigurácia
 	\item \textbf{size} (in): Vyhradená veľkosť pre správu do \verb|msg|
	\item \textbf{config} (in): Systémová konfigurácia
	\item \textbf{Návratová hodnota}: Dĺžka serializovanej správy
\end{itemize}
\bigbreak
\hrule

\begin{lstlisting}[style=docs]
bool config_parse(
	const char *msg, int size, 
	Configuration *conf, bool *error
)
\end{lstlisting}
Parsovanie systémovej konfigurácie z formátu Message Pack. \\ 
\textbf{Parametre}:
\begin{itemize}[noitemsep, topsep=0pt, label=$\star$]
	\item \textbf{msg} (in): Serializovaná konfigurácia
	\item \textbf{size} (in): Dĺžka konfigurácie v Message Pack
 	\item \textbf{conf} (out): Systémová konfigurácia
	\item \textbf{error} (out): Chyba pri parsovaní
	\item \textbf{Návratová hodnota}: Zmena konfigurácie oproti pôvodnému obsahu \verb|conf|
\end{itemize}
\bigbreak
\hrule

\begin{lstlisting}[style=docs]
size_t login_serialize(
	char *msg, size_t size,
	const Provisioning *conf
)
\end{lstlisting}
Serializácia údajov o sieťovom pripojení. \\ 
\textbf{Parametre}:
\begin{itemize}[noitemsep, topsep=0pt, label=$\star$]
	\item \textbf{msg} (out): Serializované nastavenia pripojenia
	\item \textbf{size} (in): Vyhradená veľkosť pre správu do \verb|msg|
	\item \textbf{config} (in): Nastavenia pripojenia
	\item \textbf{Návratová hodnota}: Dĺžka serializovanej správy
\end{itemize}
\bigbreak
\hrule

\begin{lstlisting}[style=docs]
bool login_parse(
	const char *msg, size_t size, 
	Provisioning *conf
)
\end{lstlisting}
Parsovanie nastavení sieťového pripojenia z formátu Message Pack. \\ 
\textbf{Parametre}:
\begin{itemize}[noitemsep, topsep=0pt, label=$\star$]
	\item \textbf{msg} (in): Serializované konfigurácia
 	\item \textbf{size} (in): Vyhradená veľkosť pre správu do \verb|msg|
	\item \textbf{conf} (out): Nastavenia pripojenia
	\item \textbf{Návratová hodnota}: Chyba pri parsovaní
\end{itemize}
\bigbreak
\hrule


\subsection{Deskriptívna štatistika} \label{modules:statistics}

\subsubsection*{Funkcie}
\hrule
\begin{lstlisting}[style=docs]
float minimum(const float *x, int n)
\end{lstlisting}
Najnižšia hodnota. \\ 
\textbf{Parametre}:
\begin{itemize}[noitemsep, topsep=0pt, label=$\star$]
	\item \textbf{x} (in): Vzorky signálu
	\item \textbf{n} (in):  Počet vzoriek signálu
	\item \textbf{Návratová hodnota}: Mimimum z hodnôt signálu
\end{itemize}
\bigbreak
\hrule

\begin{lstlisting}[style=docs]
float maximum(const float *x, int n)
\end{lstlisting}
Najvyššia hodnota. \\ 
\textbf{Parametre}:
\begin{itemize}[noitemsep, topsep=0pt, label=$\star$]
	\item \textbf{x} (in): Vzorky signálu
	\item \textbf{n} (in):  Počet vzoriek signálu
	\item \textbf{Návratová hodnota}: Maximum z hodnôt signálu
\end{itemize}
\bigbreak
\hrule

\begin{lstlisting}[style=docs]
float root_mean_square(const float *x, int n)
\end{lstlisting}
Stredná kvadratická odchýlka. \\ 
\textbf{Parametre}:
\begin{itemize}[noitemsep, topsep=0pt, label=$\star$]
	\item \textbf{x} (in): Vzorky signálu
	\item \textbf{n} (in):  Počet vzoriek signálu
	\item \textbf{Návratová hodnota}:  RMS z hodnôt signálu
\end{itemize}
\bigbreak
\hrule

\begin{lstlisting}[style=docs]
float mean(const float *x, int n)
\end{lstlisting}
Aritmetický výberový priemer. \\ 
\textbf{Parametre}:
\begin{itemize}[noitemsep, topsep=0pt, label=$\star$]
	\item \textbf{x} (in): Vzorky signálu
	\item \textbf{n} (in):  Počet vzoriek signálu
	\item \textbf{Návratová hodnota}:  Priemer hodnôt signálu
\end{itemize}
\bigbreak
\hrule

\begin{lstlisting}[style=docs]
float variance(const float *x, int n, float mean)
\end{lstlisting}
Rozptyl populácie (vychýlená štatistika). \\ 
\textbf{Parametre}:
\begin{itemize}[noitemsep, topsep=0pt, label=$\star$]
	\item \textbf{x} (in): Vzorky signálu
	\item \textbf{n} (in):  Počet vzoriek signálu
	\item \textbf{Návratová hodnota}:  Rozptyl hodnôt signálu
\end{itemize}
\bigbreak
\hrule

\begin{lstlisting}[style=docs]
float standard_deviation(float variance)
\end{lstlisting}
Smerodajná odchýlka. \\ 
\textbf{Parametre}:
\begin{itemize}[noitemsep, topsep=0pt, label=$\star$]
	\item \textbf{variiance} (in): Rozptyl signálu
	\item \textbf{Návratová hodnota}:  Smerodajná odchýlka hodnôt signálu
\end{itemize}
\bigbreak
\hrule

\begin{lstlisting}[style=docs]
float moment(const float *x, int n, int m, float mean)
\end{lstlisting}
Centrálny moment rádu \verb|m|. \\ 
\textbf{Parametre}:
\begin{itemize}[noitemsep, topsep=0pt, label=$\star$]
	\item \textbf{x} (in): Vzorky signálu
	\item \textbf{n} (in): Počet vzoriek signálu
	\item \textbf{m} (in): Rád centrálneho momentu. Kladné číslo väčšie ako 1.
	\item \textbf{mean} (in): Aritmetický priemer signálu
	\item \textbf{Návratová hodnota}:  Centrálny moment
\end{itemize}
\bigbreak
\hrule

\begin{lstlisting}[style=docs]
float skewness(const float *x, int n, float mean)
\end{lstlisting}
Šikmosť. \\ 
\textbf{Parametre}:
\begin{itemize}[noitemsep, topsep=0pt, label=$\star$]
	\item \textbf{x} (in): Vzorky signálu
	\item \textbf{n} (in): Počet vzoriek signálu
	\item \textbf{mean} (in): Aritmetický priemer signálu
	\item \textbf{Návratová hodnota}:  Šikmosť
\end{itemize}
\bigbreak
\hrule

\begin{lstlisting}[style=docs]
float kurtosis(const float *x, int n, float mean)
\end{lstlisting}
Špicatosť. \\ 
\textbf{Parametre}:
\begin{itemize}[noitemsep, topsep=0pt, label=$\star$]
	\item \textbf{x} (in): Vzorky signálu
	\item \textbf{n} (in): Počet vzoriek signálu
	\item \textbf{mean} (in): Aritmetický priemer signálu
	\item \textbf{Návratová hodnota}:  Špicatosť
\end{itemize}
\bigbreak
\hrule

\begin{lstlisting}[style=docs]
float correlation(
	const float *x_diff, const float *y_diff, int n, 
	float x_std, float y_std
)
\end{lstlisting}
Korelácia z medzivýsledkov. \\ 
\textbf{Parametre}:
\begin{itemize}[noitemsep, topsep=0pt, label=$\star$]
	\item \textbf{x\_diff} (in):  Predspracované vzorky prvého signálu odčítané od aritmetického priemeru: $(x_i - \bar{x})$
	\item \textbf{y\_diff} (in):  Predspracované vzorky druhého signálu odčítané od aritmetického priemeru: $(y_i - \bar{y})$
	\item \textbf{n} (in): Počet vzoriek signálu. Dĺžky oboch polí musia byť rovnaké.
	\item \textbf{x\_std} (in): Smerodajná odchýlka prvého signálu
	\item \textbf{y\_std} (in): Smerodajná odchýlka druhého signálu
	\item \textbf{Návratová hodnota}:  Pearsonov korelačný koeficient
\end{itemize}
\bigbreak
\hrule

\begin{lstlisting}[style=docs]
float quickselect(const float *x, int n, int k)
\end{lstlisting}
Quickselect. Algoritmus na nájdenie k-teho najmenšieho prvku v nezoradenom poli. Aby nedochádzalo
k modifikácii poradia pôvodného poľa kopíruje prvky do poľa z premenlivou dĺžkou (Variable-length array) 
 podľa \verb|n|, na zásobníku. \\ 
\textbf{Parametre}:
\begin{itemize}[noitemsep, topsep=0pt, label=$\star$]
	\item \textbf{x}: Vzorky signálu
	\item \textbf{n} (in): Počet vzoriek signálu
	\item \textbf{k} (in): Rád k-teho najmenšieho prvku
	\item \textbf{Návratová hodnota}: k-ty najmenší prvok
\end{itemize}
\bigbreak
\hrule

\begin{lstlisting}[style=docs]
float median(const float *x, int n)
\end{lstlisting}
Medián cez Quickselect. \\ 
\textbf{Parametre}:
\begin{itemize}[noitemsep, topsep=0pt, label=$\star$]
	\item \textbf{x} (in): Vzorky signálu
	\item \textbf{n} (in): Počet vzoriek signálu
	\item \textbf{Návratová hodnota}: Medián
\end{itemize}
\bigbreak
\hrule

\begin{lstlisting}[style=docs]
float median_abs_deviation(
	const float *x, int n, float med
)
\end{lstlisting}
Mediánová absolútna odchýlka (MAD).  Medzi-výsledky odchýlok na nájdenie mediánu ukladá
do poľa z premenlivou dĺžkou (Variable-length array) podľa \verb|n|, na zásobníku \\
\textbf{Parametre}:
\begin{itemize}[noitemsep, topsep=0pt, label=$\star$]
	\item \textbf{x} (in): Vzorky signálu
	\item \textbf{n} (in): Počet vzoriek signálu
	\item \textbf{med} (in):  Medián signálu
	\item \textbf{Návratová hodnota}: MAD
\end{itemize}
\bigbreak
\hrule

\begin{lstlisting}[style=docs]
float average_abs_deviation(
	const float *x, int n, float mean
)
\end{lstlisting}
Priemerná absolútna odchýlka (AAD). \\ 
\textbf{Parametre}:
\begin{itemize}[noitemsep, topsep=0pt, label=$\star$]
	\item \textbf{x} (in): Vzorky signálu
	\item \textbf{n} (in): Počet vzoriek signálu
	\item \textbf{mean} (in):  Aritmetický priemer signálu
	\item \textbf{Návratová hodnota}: AAD
\end{itemize}
\bigbreak
\hrule


\section{Prehľad MQTT topics} 
imu/[id]/
"imu started"
	"config received"
	"config malformed"
	"config applied"
	"login malformed"
	"login saved"

\begin{itemize}[noitemsep,topsep=0pt]
	\item \textbf{syslog}
	\item \textbf{config/request}
	\item \textbf{config/response}
	\item \textbf{config/set}
	\item \textbf{login/request}
	\item \textbf{login/response}
	\item \textbf{login/set}
	\item \textbf{samples}
	\item \textbf{stats/x}
	\item \textbf{spectrum/x}
	\item \textbf{events/x}
\end{itemize}
	
\section{Štruktúra Message Pack správ}
- samples
\begin{verbatim}
[-0.07836493849754333, -0.9104690551757812, -9.964909553527832, -1.7736798524856567, -16.43988800048828, -0.4007977843284607, 1.4985052347183228, 5.201996326446533, 1.499103307723999]
\end{verbatim}

- stats/x, stats/y, stats/z
\begin{verbatim}
{"t": 386, "min": -9.964909553527832, "max": 11.8468656539917, "rms": 8.343297004699707, "avg": 4.126497745513916, "std": 7.251387119293213, "skew": -0.6144028306007385, "kurt": -0.8219752311706543, "med": 5.772983551025391, "mad": 5.370391845703125}
\end{verbatim}

- spectrum/x spectrum/y  spectrum/z
\begin{verbatim}
{"t": 25, "fs": 8, "bins": [0.0, -0.026730040088295937, -38.7723274230957, -38.19501495361328]}
\end{verbatim}

- events/x events/y events/z
\begin{verbatim}
{
"t": 386, "df": 2.0,
"A": [{"i": 2, "t": 382, "d": 5, "h": -5.620919704437256}, {"i": 3, "t": 382, "d": 5, "h": -3.549427032470703}],
"Z": []
}
\end{verbatim}

\begin{verbatim}
{"sensor": {"fs": 8, "range": "2g", "n": 8, "overlap": 0.5, "axis": [false, false, true]},
"tsmooth": {"on": false, "n": 8, "repeat": 1},
"stats": {"min": true, "max": true, "rms": true, "avg": true, "var": true, "std": true, "skew": true, "kurtosis": true, "med": true, "mad": true, "corr": false},
"transform": {"w": "hann", "f": "dft", "log": true},
"fsmooth": {"on": false, "n": 8, "repeat": 1},
"peak": {"tmin": 4, "tprox": 5, "strategy": "threshold", "threshold": {"t": -15.0},
"neighbours": {"k": 9, "e": 0.0, "h": -100.0, "h_rel": 10.0},
"zero_crossing": {"k": 4, "slope": 3.0},
"hill_walker": {"t": 0.0, "h": 0, "p": 10.0, "i": 3.0}},
"logger": {"local": false, "mqtt": true, "samples": "t", "stats": true, "events": true, "subsamp": 1}}
\end{verbatim}

\section{Datasety z premávky}
Datasety z autobusov a električiek boli zaznamenané v dátumoch 1.11. a 3.11.2021 so vzorkovacou frekvenciou 500 Hz 
a rozlíšením $\pm 2$ g so zariadením opísaním v hlavnej časti. Vozidlá boli súčasťou bežnej výpravy metskej 
hromadnej dopravy prepravcu Dopravný podnik Bratislava, a.s. Mierne diskrepancie na úrovni vzoriek a nezmyselné 
presahy v nahrávaní boli odstránené. Asfaltové povrchy ciest boli pomerne nové a suché.

\paragraph{L3\_StnVinohrady\_Riazanska.csv}
\begin{itemize}[noitemsep, topsep=0pt]
  	\item \textbf{Linka:} 3
  	\item \textbf{Trvanie:} 120033 vzoriek (240,07 s)
  	\item \textbf{Zastávky:} Stn. Vinohrady (v pokoji pred semafórom), Nám. Biely Kríž, Mladá Garda, Riazanská
  	\item \textbf{Vozidlo:} električka Škoda 30 T
	\item \textbf{Umiestnenie:} pravá časť nápravy, vyvýšené sedenie v štvorke
\end{itemize}

\paragraph{L3\_Pionierska\_RacianskeMyto.csv}
\begin{itemize}[noitemsep, topsep=0pt]
  	\item \textbf{Linka:} 3
  	\item \textbf{Trvanie:} 70437 vzoriek (140,87 s)
  	\item \textbf{Zastávky:} Pionierska, Ursínyho, Račianske mýto
  	\item \textbf{Vozidlo:} električka Škoda 30 T
  	\item \textbf{Umiestnenie:} pravá časť nápravy, vyvýšené sedenie v štvorke
\end{itemize}

\paragraph{L9\_Postova\_KralovskeUdolie.csv}
\begin{itemize}[noitemsep, topsep=0pt]
  	\item \textbf{Linka:} 9
  	\item \textbf{Trvanie:} 149150 vzoriek (298,3 s)
  	\item \textbf{Zastávky:} Poštová, Kapucínska, (Tunel), Kráľovské údolie
  	\item \textbf{Vozidlo:} električka Škoda 29 T
  	\item \textbf{Umiestnenie:} ľavá časť v zníženom sedení medzi harmonikou a vyvýšenou zadnou plošinou
\end{itemize}
  		
\paragraph{L9\_Lanfranconi\_Riviera.csv}
\begin{itemize}[noitemsep, topsep=0pt]
  	\item \textbf{Linka:} 9
  	\item \textbf{Trvanie:} 79010 vzoriek (158,02 s)
  	\item \textbf{Zastávky:} Lanfranconi, Botanická záhrada, Riviéra
  	\item \textbf{Vozidlo:} električka Škoda 29 T
  	\item \textbf{Umiestnenie:} ľavá časť v zníženom sedení medzi harmonikou a vyvýšenou zadnou plošinou
\end{itemize}
  	
\paragraph{L35\_1915\_Most\_Kutiky\_neutral.csv}
	\begin{itemize}[noitemsep, topsep=0pt]
  	\item \textbf{Linka:} 35
  	\item \textbf{Trvanie:} 8315 vzoriek (16,63 s)
  	\item \textbf{Zastávky:} žiadne - zastavený na moste Kútiky kvôli prekážke na ceste
  	\item \textbf{Vozidlo:} midibus Solaris Urbino 8,6 \#1915
  	\item \textbf{Umiestnenie:} pravá zadná náprava, predposledné zadné sedenie proti smeru jazdy
  	\end{itemize}
\paragraph{L35\_1915\_Borska\_Zaluhy.csv}
	\begin{itemize}[noitemsep, topsep=0pt]
  	\item \textbf{Linka:} 35
  	\item \textbf{Trvanie:} 109502 vzoriek (219 s)
  	\item \textbf{Zastávky:} koniec Púpavovej ulice (jazda), Borská (zastávka), Záluhy (jazda, hneď za pravotočivou zákrutou križovatky na smer Lamač)
  	\item \textbf{Vozidlo:} midibus Solaris Urbino 8,6 \#1915
  	\item \textbf{Umiestnenie:} pravá zadná náprava, predposledné zadné sedenie proti smeru jazdy
  	\end{itemize}
 
\paragraph{L20\_3014\_Zaluhy\_Drobneho.csv}
	\begin{itemize}[noitemsep, topsep=0pt]
  	\item \textbf{Linka:} 20
  	\item \textbf{Trvanie:} 113001 vzoriek (226 s)
  	\item \textbf{Zastávky:} Záluhy (jazda, pred križovatkou smer Dúbravka od Lamača), Záluhy (zastávka), Švantnerova, Alexyho, Drobného, Podvornice (na polceste, jazda, križovatka)
  	\item \textbf{Vozidlo:} elektrobus SOR NS 12 Electric \#3014
  	\item \textbf{Umiestnenie:} ľavá zadná náprava, predposledné zadné sedenie, pod pravým sedadlom v dvojke
  	\end{itemize}
  
\paragraph{L83\_4940\_PriKrizi\_Alexyho.csv}
	\begin{itemize}[noitemsep, topsep=0pt]
  	\item \textbf{Linka:} 83
  	\item \textbf{Trvanie:} 208089 vzoriek (416,18 s)
  	\item \textbf{Zastávky:} Pri Kríži (tesne po rozbehnutí), Homolova, Štepná, Žatevná, Pekníková, Drobného, Alexyho (križovatka, zastavenie na semafóre)
  	\item \textbf{Vozidlo:} kĺbový autobus Mercedes-Benz O 530 GL CapaCity \#4940
  	\item \textbf{Umiestnenie:} nad motorom vpravo vzadu, posledné zadné priečne sedadlo pred plošinou na batožinu
  	\end{itemize}
  
\paragraph{L83\_4940\_Alexyho\_Svantnerova.csv}
	\begin{itemize}[noitemsep, topsep=0pt]
  	\item \textbf{Linka:} 83
  	\item \textbf{Trvanie:} 44182 vzoriek (88,36 s)
  	\item \textbf{Zastávky:} Alexyho (zastávka), Švantnerova (zastávka, na zvažujúcom kopci)
  	\item \textbf{Vozidlo:} kĺbový autobus Mercedes-Benz O 530 GL CapaCity \#4940
  	\item \textbf{Umiestnenie:} nad motorom vpravo vzadu, posledné zadné priečne sedadlo pred plošinou na batožinu
  	\end{itemize}

\paragraph{L4\_7954\_ZaluhyKrizovatka\_KutikyObratisko.csv}
	\begin{itemize}[noitemsep, topsep=0pt]
  	\item \textbf{Linka:} 4
  	\item \textbf{Trvanie:} 97362 vzoriek (194,72 s)
  	\item \textbf{Zastávky:} Záluhy (semafór, pokoj),  Horné Krčace, Dolné Krčace, Kútiky obratisko za druhou výhybkou (pohyb)
  	\item \textbf{Vozidlo:} električka ČKD Tatra T6A5 \#7954
  	\item \textbf{Umiestnenie:} zadný vozeň v okolí nad ľavou časťou prednej nápravy
  	\end{itemize}

\clearpage
\clearpage

 \thispagestyle{empty}
\setcounter{figure}{0}
\chapter{Používateľská príručka}
\pagenumbering{arabic}
\renewcommand*{\thepage}{C-\arabic{page}}

\section{Inštalačný manuál}
Špecifikovať testovaciu platformu
(esp-idf) v4.4, FreeRTOS v9.0.0, esp-dsp v1.2, MPack v1.1, CMake - nástroj na automatizácie kompilácie (GNU 8.4.0), Platforma Manjaro Linux
	Linux ThinkPad 5.10.109-1-MANJARO (\url{https://ludocode.github.io/mpack/md_docs_expect.html}), Verzie a návod na inštaláciu v prílohe
- Python3.10, numpy, scipy (stats, signal, fft, interpolate), pandas, seaborn, matplotlib
- msgpack, json, cmd,  Eclipse Mosquitto

sudo pacman -S --needed gcc git make flex bison gperf python-pip cmake ninja ccache dfu-util libusb
\url{git clone -b v4.4 --recursive https://github.com/espressif/esp-idf.git}
\url{git clone -b v1.2.0 https://github.com/espressif/esp-dsp.git}
\url{wget https://github.com/ludocode/mpack/releases/download/v1.1/mpack-amalgamation-1.1.tar.gz}


cd ~/esp/esp-idf
./install.sh esp32


\textbf{Flashovanie a sprevádzkovanie}
\begin{lstlisting}[style=messages]
. ./export.sh
cd firmware/vibration-analyzer
idf.py build
idf.py -p /dev/ttyUSB0 flash
idf.py -p /dev/ttyUSB0 monitor
\end{lstlisting}

na SD kartu dať: config.txt

na MQTT server dať: mosquitto.conf
systemctl status


\section{Návod na použitie}
\textbf{MQTT klient na nahrávanie konfigurácií}
cd  main/tests/
pip install -r requirements.txt
python config\_tool.py

Príkazy:
\begin{itemize}[noitemsep, topsep=0pt]
	\item \textbf{connect}
		Device ID [1]:
      	Broker IP [192.168.1.103]
        Broker Port [1883]
	\item \textbf{disconnect}
	\item \textbf{end}
	\item \textbf{^C}
	\item \textbf{set}
		config>
	\item \textbf{config}
	\item \textbf{topic}
		topic>
		Listen time on topic in sec [10]:
	\item \textbf{login}
	\item \textbf{credentials}
\end{itemize}

\url{https://github.com/espressif/esp-idf}
\url{https://github.com/espressif/esp-dsp} s úpravou
\url{https://github.com/ludocode/mpack/releases/download/v1.1/mpack-amalgamation-1.1.tar.gz}

\textbf{Jupyter notebooky}
\begin{lstlisting}[style=messages]
cd measurements
pip install -r requirements.txt
jupyter notebook
\end{lstlisting}

\section{Replikácia experimentov}

Dôležité prepínače kompilátora cez \verb|idf.py menuconfig|:
    - Optimalizácia na veľkosť: -Os (\verb|CONFIG_COMPILER_OPTIMIZATION_SIZE=y|)
    - Taktovacia frekvencia: 160 MHz   (\verb|CONFIG_ESP32_DEFAULT_CPU_FREQ_MHZ=160|)
    - Tick operačného systému: 100 Hz tick  (\verb|CONFIG_FREERTOS_HZ=100|)
    
Kompilovanie a nahratie na ESP32:

\begin{lstlisting}[style=messages]
$ idf.py build
$ idf.py flash
\end{lstlisting}

Nastav počiatočnú konfiguráciu nástrojom \verb|python config_tool.py|:

\begin{lstlisting}[style=messages]
-> connect
-> set
    {"sensor": {"fs": 8, "range": "2g", "n": 8, "overlap": 0.5, "axis": [true, true, true]}, "tsmooth": {"on": false, "n": 8, "repeat": 1}, "stats": {"min": true, "max": true, "rms": true, "avg": true, "var": true, "std": true, "skew": true, "kurtosis": true, "med": true, "mad": true, "corr": false}, "transform": {"w": "hann", "f": "dft", "log": true}, "fsmooth": {"on": false, "n": 8, "repeat": 1}, "peak": {"tmin": 4, "tprox": 5, "strategy": "threshold", "threshold": {"t": -15.0}, "neighbours": {"k": 9, "e": 0.0, "h": -100.0, "h_rel": 10.0}, "zero_crossing": {"k": 4, "slope": 3.0}, "hill_walker": {"t": 0.0, "h": 0, "p": 10.0, "i": 3.0}}, "logger": {"local": false, "mqtt": true, "samples": "", "stats": false, "events": false, "subsamp": 1}}
\end{lstlisting}

\section{Experiment: Časová efektivita algoritmov na spracovanie signálu}
V \emph{main.c} je potrebné sa presvedčiť, že z direktív pre meranie je odkomentovaná len direktíva: 
\verb|MEMORY_MEASUREMENT|

Po každom riadku nastavením novej konfigurácie je potrebné čakať na 10 meraní časov z 
\begin{lstlisting}[style=messages]
idf.py monitor | tee memory_usage.txt
\end{lstlisting}

\begin{lstlisting}[style=messages]
-> set
{"sensor": {"n": 8}, "tsmooth": {"n": 8}, "fsmooth": {"n": 8}}
{"sensor": {"n": 16}, "tsmooth": {"n": 16}, "fsmooth": {"n": 16}}
{"sensor": {"n": 32}, "tsmooth": {"n": 32}, "fsmooth": {"n": 32}}
{"sensor": {"n": 64}, "tsmooth": {"n": 64}, "fsmooth": {"n": 64}}
{"sensor": {"n": 128}, "tsmooth": {"n": 128}, "fsmooth": {"n": 128}}
{"sensor": {"n": 256}, "tsmooth": {"n": 256}, "fsmooth": {"n": 256}}
{"sensor": {"n": 512}, "tsmooth": {"n": 512}, "fsmooth": {"n": 512}}
{"sensor": {"n": 1024}, "tsmooth": {"n": 1024}, "fsmooth": {"n": 1024}}
\end{lstlisting}

Následné fitrovanie relevantných riadkov:
\begin{lstlisting}[style=messages]
$ sed -rn 's/^(.*)(MEM.*)$/\2/p' memory_usage.txt > memory_usage_filter.csv
\end{lstlisting}


\section{Experiment: Časová efektivita algoritmov na spracovanie signálu}

V \emph{main.c} je potrebné sa presvedčiť, že z direktív pre meranie je odkomentovaná len direktíva: 
\verb|EXECUTION_TIME_ALGORITHMS|.

Nastav počiatočnú konfiguráciu nástrojom \verb|python config_tool.py|:
\begin{lstlisting}[style=messages]
{"sensor": {"axis": [false, false, true]}}
\end{lstlisting}

Po každom riadku nastavením novej konfigurácie je potrebné čakať na 10 meraní časov z 
\begin{lstlisting}[style=messages]
idf.py monitor | tee execution_algorithms.txt
\end{lstlisting}

\begin{lstlisting}[style=messages]
{"sensor": {"n": 32, "fs": 16}, "transform": {"f": "dft"}, "peak": {"strategy": "neighbours"}}
{"sensor": {"n": 32, "fs": 16}, "transform": {"f": "dft"}, "peak": {"strategy": "zero_crossing"}}
{"sensor": {"n": 32, "fs": 16}, "transform": {"f": "dft"}, "peak": {"strategy": "hill_walker"}}
{"sensor": {"n": 32, "fs": 16}, "transform": {"f": "dct"}, "peak": {"strategy": "neighbours"}}
\end{lstlisting}

Medzi každým:
\begin{lstlisting}[style=messages]
{"sensor": {"n": 64, "fs": 32}}
{"sensor": {"n": 128, "fs": 64}}
{"sensor": {"n": 256, "fs": 128}}
{"sensor": {"n": 512, "fs": 256}}
{"sensor": {"n": 1024, "fs": 512}}
\end{lstlisting}



\section{Experiment: Časová efektivita vyhladzovacieho filtra}

V \emph{main.c} je potrebné sa presvedčiť, že z direktív pre meranie je odkomentovaná len direktíva:
\verb|EXECUTION_TIME_SMOOTHING|.

Nastav počiatočnú konfiguráciu nástrojom \verb|python config_tool.py|:

\begin{lstlisting}[style=messages]
{"sensor": {"fs": 256, "n": 512, "axis": [false, false, true]},  {"tsmooth": {"on": true}}
\end{lstlisting}

Po každom riadku nastavením novej konfigurácie je potrebné čakať na 10 meraní časov:
\begin{lstlisting}[style=messages]
idf.py monitor | tee execution_time.txt
\end{lstlisting}

\begin{lstlisting}[style=messages]
-> set
{"tsmooth": {"n": 4, "repeat": 1}}
{"tsmooth": {"n": 16, "repeat": 1}}
{"tsmooth": {"n": 64, "repeat": 1}}
{"tsmooth": {"n": 4, "repeat": 4}}
{"tsmooth": {"n": 16, "repeat": 4}}
{"tsmooth": {"n": 64, "repeat": 4}}
{"tsmooth": {"n": 4, "repeat": 8}}
{"tsmooth": {"n": 16, "repeat": 8}}
{"tsmooth": {"n": 64, "repeat": 8}}
\end{lstlisting}


\section{Experiment: Časová efektivita dátovej pipeline}

V \emph{main.c} je potrebné sa presvedčiť, že z direktív pre meranie je odkomentovaná len nasledovná: 
\verb|EXECUTION_TIME_PIPELINE|.

Nastav počiatočnú konfiguráciu nástrojom \verb|python config_tool.py|:


{"sensor": {"fs": 16, "range": "2g", "n": 32, "overlap": 0.5, "axis": [false, false, true]}, "stats": {"min": false, "max": false, "rms": false, "avg": false, "var": false, "std": false, "skew": false, "kurtosis": false, "med": false, "mad": false, "corr": false}}
```

Po každom riadku nastavením novej konfigurácie je potrebné čakať na 10 meraní časov: 
\begin{lstlisting}[style=messages]
idf.py monitor | tee execution_time.txt
\end{lstlisting}

\paragraph{Tabuľka A}

\begin{lstlisting}[style=messages]
-> set
{"sensor": {"fs": 16, "n": 32, "axis": [false, false, true]}, "transform": {"f": "dft", "log": true}, "peak": {"strategy": "neighbours"}}
{"peak": {"strategy": "zero_crossing"}}
{"peak": {"strategy": "hill_walker"}}

{"sensor": {"fs": 128, "n": 256}, "peak": {"strategy": "neighbours"}}
{"peak": {"strategy": "zero_crossing"}}
{"peak": {"strategy": "hill_walker"}}

{"sensor": {"fs": 512, "n": 1024}, "peak": {"strategy": "neighbours"}}
{"peak": {"strategy": "zero_crossing"}}
{"peak": {"strategy": "hill_walker"}}

{"sensor": {"fs": 16, "n": 32, "axis": [false, false, true]}, "transform": {"f": "dct", "log": true}, "peak": {"strategy": "neighbours"}}
{"peak": {"strategy": "zero_crossing"}}
{"peak": {"strategy": "hill_walker"}}

{"sensor": {"fs": 128, "n": 256}, "peak": {"strategy": "neighbours"}}
{"peak": {"strategy": "zero_crossing"}}
{"peak": {"strategy": "hill_walker"}}

{"sensor": {"fs": 512, "n": 1024}, "peak": {"strategy": "neighbours"}}
{"peak": {"strategy": "zero_crossing"}}
{"peak": {"strategy": "hill_walker"}}
\end{lstlisting}

\paragraph{Tabuľka B (9x) - 3 osi}

\begin{lstlisting}[style=messages]
{"sensor": {"fs": 16, "n": 32, "axis": [true, true, true]}, "transform": {"f": "dft", "log": true}, "peak": {"strategy": "neighbours"}}
{"peak": {"strategy": "zero_crossing"}}
{"peak": {"strategy": "hill_walker"}}

{"sensor": {"fs": 128, "n": 256}, "peak": {"strategy": "neighbours"}}
{"peak": {"strategy": "zero_crossing"}}
{"peak": {"strategy": "hill_walker"}}

{"sensor": {"fs": 512, "n": 1024}, "peak": {"strategy": "neighbours"}}
{"peak": {"strategy": "zero_crossing"}}
{"peak": {"strategy": "hill_walker"}}

{"sensor": {"fs": 16, "n": 32, "axis": [true, true, true]}, "transform": {"f": "dct", "log": true}, "peak": {"strategy": "neighbours"}}
{"peak": {"strategy": "zero_crossing"}}
{"peak": {"strategy": "hill_walker"}}

{"sensor": {"fs": 128, "n": 256}, "peak": {"strategy": "neighbours"}}
{"peak": {"strategy": "zero_crossing"}}
{"peak": {"strategy": "hill_walker"}}

{"sensor": {"fs": 512, "n": 1024}, "peak": {"strategy": "neighbours"}}
{"peak": {"strategy": "zero_crossing"}}
{"peak": {"strategy": "hill_walker"}}
\end{lstlisting}

\paragraph{Tabuľka C (stats)}

\begin{lstlisting}[style=messages]
{"sensor": {"fs": 16, "n": 32, "axis": [false, false, true]},  "stats": {"min": true, "max": true, "rms": true, "avg": true, "var": true, "std": true, "skew": true, "kurtosis": true, "med": true, "mad": true, "corr": true},  "logger": {"samples": "f", "stats": true}, "transform": {"f": "dft", "log": true}, "peak": {"strategy": "neighbours"}}
{"peak": {"strategy": "zero_crossing"}}
{"peak": {"strategy": "hill_walker"}}

{"sensor": {"fs": 128, "n": 256}, "peak": {"strategy": "neighbours"}}
{"peak": {"strategy": "zero_crossing"}}
{"peak": {"strategy": "hill_walker"}}

{"sensor": {"fs": 512, "n": 1024}, "peak": {"strategy": "neighbours"}}
{"peak": {"strategy": "zero_crossing"}}
{"peak": {"strategy": "hill_walker"}}
\end{lstlisting}

\paragraph{Tabuľka D}

\begin{lstlisting}[style=messages]
{"sensor": {"fs": 16, "n": 32, "axis": [true, true, true]}, "transform": {"f": "dft", "log": true}, "peak": {"strategy": "neighbours"}}
{"peak": {"strategy": "zero_crossing"}}
{"peak": {"strategy": "hill_walker"}}

{"sensor": {"fs": 128, "n": 256}, "peak": {"strategy": "neighbours"}}
{"peak": {"strategy": "zero_crossing"}}
{"peak": {"strategy": "hill_walker"}}

{"sensor": {"fs": 512, "n": 1024}, "peak": {"strategy": "neighbours"}}
{"peak": {"strategy": "zero_crossing"}}
{"peak": {"strategy": "hill_walker"}}
\end{lstlisting}

Pre namiesto X: A, B, C, D

\begin{lstlisting}[style=messages]
sed -rn '/^(.*)(main:.*)$/p' pipeline_time-X.txt > pipeline_time-X_filter.csv
\end{lstlisting}




\clearpage

\thispagestyle{empty}
\setcounter{figure}{0}
\chapter{Spektrogramy detekcií}
\pagenumbering{arabic}
\renewcommand*{\thepage}{D-\arabic{page}}

L35 - Ilustrácia detegovaných špičiek pri fs 476Hz a N = 256 a parametrov podľa grid search
\begin{figure}[h]
	\centering
    \includegraphics[width=\textwidth]{figures/verification/L83-dataset-spectrum.png}
    \caption{Algoritmus č.1}
\end{figure}

\begin{figure}[h]
	\centering
    \includegraphics[width=\textwidth]{figures/verification/L35-dataset-spectrum.png}
    \caption{Algoritmus č.1}
\end{figure}

\begin{figure}[h]
	\centering
    \includegraphics[width=\textwidth]{figures/verification/L35-dataset-A1.png}
    \caption{Algoritmus č.1}
\end{figure}

\begin{figure}[h]
	\centering
    \includegraphics[width=\textwidth]{figures/verification/L35-dataset-A2.png}
    \caption{Algoritmus č.2}
\end{figure}

\begin{figure}[h]
	\centering
    \includegraphics[width=\textwidth]{figures/verification/L35-dataset-A3.png}
    \caption{Algoritmus č.3}
\end{figure}


\thispagestyle{empty}
\setcounter{figure}{0}
\chapter{Obsah digitálneho média}
\pagenumbering{arabic}
\renewcommand*{\thepage}{E-\arabic{page}}
\par Evidenčné číslo práce v informačnom systéme: \RegNo
\par Obsah digitálnej časti práce (archív ZIP):
\par Názov odovzdaného archívu: BP\_MiroslavHajek.zip

\begin{itemize}[noitemsep]
\item[\textbf{>}] \textbf{firmware} - Zdrojový kód firmvéru senzorovej jednotky
	\begin{itemize}
	\item[\textbf{>}] \textbf{esp-dsp} - ESP DSP Library - knižnica na optimalizáciu spracovania signálov na ESP32
	\item[\textbf{>}] \textbf{esp-idf} - Espressif IoT Development Framework - SDK pre hardvér ESP32
	\item[\textbf{>}] \textbf{mpack} - MPack knižnica enkódera a dekódera MessagePack serializačného formátu 
	\item[\textbf{>}] \textbf{vibration-analyzer} - Samotná implementácia aplikačnej logiky
		\begin{itemize}
		\item[\textbf{>}] \textbf{conf} - Konfiguračné súbory pre OpenLog (\emph{config.txt}), MQTT broker  (\emph{mosquitto.conf})
		a na zostavenie dokumentácie cez Doxygen (\emph{doxygen.conf})
		\item[\textbf{>}] \textbf{docs} - Doxygen dokumentácia. Úvodná stránka je \emph{index.html}.
		\item[\textbf{>}] \textbf{main}
			\begin{itemize}
			\item[\textbf{>}] \textbf{include} - Hlavičkové súbory \emph{.h} aplikácie
			\item[\textbf{>}] \textbf{src} - Zdrojové súbory \emph{.c} aplikácie
			\end{itemize}
		\item[\textbf{>}] \textbf{tests} - Jednotkové testy na kontrolu najdôležitejšej funkcionality a validujúce konzistenciu
		medzi Python a C implementáciou algoritmov: \emph{test\_events.c} (test prúdového algoritmu nájdenia zmeny frekvencií),
		\emph{test\_peaks.c} (test klasifikátorov špičiek), \emph{test\_config.py} (test vzdialenej konfigurácie zariadenia).
		Nástroj na interaktívny vzdialený prístup k IoT jednotke \emph{config\_tool.py}
		\end{itemize}
	\end{itemize}
\item[\textbf{>}] \textbf{measurements} - Analýza dátových sád a generovanie syntetických signálov v notebookoch \emph{.ipynb}
	\begin{itemize}
	\item[\textbf{>}] \textbf{datasets} - Vibračné záznamy z vozidiel verejnej dopravy
	\item[\textbf{>}] \textbf{experiments} - Pôvodné monitorovacie výpisy z experimentov slúžiace ako poklad na overenie riešenia
		\begin{itemize}
			\item[\textbf{>}] \textbf{accuracies} - Úspešnosti klasifikácie na syntetickom spektrálnom profile. Súbor \emph{results.txt}
			vychádza z analýzy v \emph{VibrationProcessingAlgorithms.ipynb}, do tabuľkovej podoby sa dostal v \emph{results-table.csv}. 
			\item[\textbf{>}] \textbf{execution-algorithms} - Časy vykonávania jednotlivých algoritmov.
			\item[\textbf{>}] \textbf{execution-pipeline} - Časy vykonávania celej dátovej pipeline
			\item[\textbf{>}] \textbf{memory-usage} - Spotreba pamäte flash
			\item[\textbf{>}] \textbf{network} - Odchytená sieťová komunikácie z MQTT broker v \emph{.pcap} súboroch
		\end{itemize}
	\item[\textbf{>}] \textbf{signals} - Užitočné funkcie ku prieskumnej analýze, vizualizácii a tvorbe modelov v Jupyter notebookoch.	
	\end{itemize}
\end{itemize}


\end{document}
