\setcounter{figure}{0}
%\setcounter{listing}{0}
\chapter{Technická dokumentácia}
\pagenumbering{arabic}
\renewcommand*{\thepage}{A-\arabic{page}}

Prílohy dopĺňajú hlavnú časť práce. Obsahujú napríklad podrobné informácie k jednotlivým
etapám riešenia projektu. Typicky sa tu uvádza aj podstatná časť technickej dokumentácie.
Pozor, prílohy nesmú obsahovať také informácie, ktoré sú pre pochopenie práce kľúčové. Tie
musí obsahovať hlavná časť práce, ktorá musí byť úplná, celistvá.

Súčasťou príloh nie je len textový obsah, ale aj ďalšie artefakty, ktoré sú výsledkom projektu,
napr. počítačový kód, dátové vzorky, vedecký článok či plagát. Zvláštnu pozornosť venujte tým
artefaktom, ktoré sú potrebné pre replikovateľnosť postupov opisovaných v práci (napr. aby
mohol oponent pri vyhodnocovaní práce zopakovať uvádzané postupy a prísť k rovnakým
záverom). 

Digitálne artefakty sa prikladajú na elektronickom médiu. K akémukoľvek
digitálnemu obsahu treba uviesť v dokumente priebežnej či záverečnej správy
bakalárskej/diplomovej práce primeraný textový opis, preto nezabudnite digitálne médium
zdokumentovať. Prinajmenšom medzi prílohy zaraďte kapitolu "Obsah elektronického média".
Na prílohy sa nezabudnite z hlavnej časti práce primerane odkazovať.

Obsah technickej dokumentácie závisí od povahy riešeného problému. Uvádza sa technická dokumentácia k systému (počítačový, softvérový), ktorý bol vytvorený v rámci riešenia projektu (ak sa toto v zadaní požadovalo). Samotný obsah a rozsah závisí aj od účelu vytvoreného systému (produkt, experimentovanie a pod.)
V prípade softvérového systému technická dokumentácia spravidla obsahuje časti v náväznosti na etapy tvorby softvérového systému:
\begin{itemize}
	\item dokumentáciu k etape špecifikácie požiadaviek
    \item dokumentáciu k etape návrhu projektu
    \item dokumentáciu k implementácii
    \item v prípade, že súčasťou riešenia sú programy, dokumentáciu k implementácii tvoria zdrojové texty programov
    \item v prípade, že súčasťou riešenia je návrh zariadenia, dokumentáciu k implementácii tvorí technická dokumentácia (schémy zapojenia, návrh dosiek plošných spojov, schémy rozmiestnenia súčiastok, zoznam použitých súčiastok, opis konektorov atď.)
    \item dokumentáciu k overeniu riešenia
    \item dokumentáciu k používaniu a údržbe (návody na použitie a údržbu projektu)
\end{itemize}
 
