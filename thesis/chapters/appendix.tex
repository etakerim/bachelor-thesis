% no page numbers for appendicies
\addtocontents{toc}{\protect\setcounter{tocdepth}{0}}
\addtocontents{toc}{\cftpagenumbersoff{chapter}}
\appendix
\titleformat{\chapter}{\normalfont\huge\bf}{Príloha \thechapter:}{1em}{}

\setcounter{figure}{0}
%\setcounter{listing}{0}
%\chapter{Technická dokumentácia}
%\pagenumbering{arabic}
%\renewcommand*{\thepage}{A-\arabic{page}}

% Harmonogram práce
\thispagestyle{empty}
\chapter{Harmonogram práce}
\pagenumbering{arabic}
\renewcommand*{\thepage}{A-\arabic{page}}
\section{Zimný semester}
\begin{table}[h!]
\renewcommand{\arraystretch}{1.3}
\begin{tabular}{|l|l|}
\hline
\textbf{Obdobie} & \textbf{Náplň práce}                                                                                                                                                                                                                         \\ \hline
1.týždeň         & Základný prehľad relevantnej literatúry.                                                                                                                                                                                                      \\ \hline
2.týždeň         & \begin{tabular}[c]{@{}l@{}}Štúdium literatúry ohľadom montorovania vibrácií. Pokusný zber dát \\ akcelerácie z MHD pomocou akcelerometra na smartfóne \\ a ich prieskumná analýza.\end{tabular}                                              \\ \hline
3.týždeň         & \begin{tabular}[c]{@{}l@{}}Štúdium článkov o frekvenčnej analýze a rešerš algoritmov \\ na hľadanie špičiek. Implementácia objavených prístupov hľadania \\ špičiek a aplikovanie na merania vibrácií z električiek a autobusu.\end{tabular} \\ \hline
4.týždeň         & Flashovanie firmvéru na vývojový kit iCOMOX.                                                                                                                                                                                                  \\ \hline
5.týždeň         & Osnova práce s referenciami na nájdenú literatúru.                                                                                                                                                                                            \\ \hline
6.týždeň         & \begin{tabular}[c]{@{}l@{}}Firmvér pre dosku na platforme ESP32 pre zaznám akceleračných \\ dát na SD kartu cez OpenLog\end{tabular}                                                                                                         \\ \hline
7.týždeň         & \begin{tabular}[c]{@{}l@{}}Merania vibrácií v MHD a analýza získaných záznamov v jupyter\\ notebooku. Doplnenie zdrojov pre časti osnovy s málo referenciami.\end{tabular}                                                                   \\ \hline
8.týždeň         & Sekcia 2.1. práce o monitorovaní vibrácií a šoku.                                                                                                                                                                                             \\ \hline
9.týždeň         & \begin{tabular}[c]{@{}l@{}}Doplnenie typov akcelerometov a časti o numerickej kvadratúre \\ k sekcii 2.1. Úvod do sekcie 2.2. o analýze v časovej doméne.\end{tabular}                                                                       \\ \hline
10.týždeň        & Deskriptívne štatistiky a algoritmy na identifikáciu špičiek.                                                                                                                                                                                 \\ \hline
11.týždeň        & Sekcia 2.2. o frekvenčnej a časovo-frekvenčnej analýze signálu.                                                                                                                                                                               \\ \hline
12.týždeň        & \begin{tabular}[c]{@{}l@{}}Sekcia 2.3 o architektúre senzorových sietí a ich obmedzeniach. \\ Návrh riešenia a úvod k priebežnej správe BP1\end{tabular}                                                                                     \\ \hline
13.týždeň        & Zapracované pripomienky k prezentovanému návrhu.                                                                                                                                                                             \\ \hline
\end{tabular}
\end{table}
\clearpage
\newpage
\section{Letný semester}
\begin{table}[h!]
\renewcommand{\arraystretch}{1.3}
\begin{tabular}{|l|l|}
\hline
\textbf{Obdobie} & \textbf{Orientačný plán práce}                                                                                                                                                       \\ \hline
1. -- 3. týždeň  & \begin{tabular}[c]{@{}l@{}}Odlaďovanie modelov na monitorovanie vibrácií na základe\\ analyzovaných algoritmov. Meranie úspešnosti na synteticky\\ generovaných dátach.\end{tabular} \\ \hline
4. -- 6. týždeň  & \begin{tabular}[c]{@{}l@{}}Návrh a prvotná implementácia firmvéru pre senzorovú jednotku\\ s loggovaním výstupných udalostí na pamäťovú kartu.\end{tabular}                          \\ \hline
7. -- 9. týždeň  & \begin{tabular}[c]{@{}l@{}}Optimalizácia posielaných dát cez zvolený bezdrôtový sieťový\\ protokol. Vzdialená konfigurácia zariadenia.\end{tabular}                                  \\ \hline
10 -- 13.týždeň  & \begin{tabular}[c]{@{}l@{}}Experimenty so senzorovou jednotkou v kontrolovanom\\ a reálnom prostredí. Vyhodnotenie výskumných otázok.\end{tabular}                                    \\ \hline
\end{tabular}
\end{table}

% Digitálne médium
%\thispagestyle{empty}

%\chapter{Obsah digitálneho média}
%\pagenumbering{arabic}
%\renewcommand*{\thepage}{C-\arabic{page}}
%\par Evidenčné číslo práce v informačnom systéme: \RegNo
%\par Obsah digitálnej časti práce (archív ZIP):
%\par Názov odovzdaného archívu: ...zip

 
