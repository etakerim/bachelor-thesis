%Anotácia
\thispagestyle{empty}
\section*{Anotácia}
\University \\
\uppercase{\Faculty}
\vspace{-8pt}
{\setlength{\mathindent}{0cm}
\begin{align*}
&\text{Študijný program:} && \text{\StudyProgramme} \\
&\text{Autor:} && \text{\Author} \\
&\text{\Thesis:} && \text{\Title} \\
&\text{Vedúci bakalárskej práce:} && \text{\Supervisor} \\
&\text{Pedagogický vedúci:} && \text{\PedagogicalSupervisor} \\
&\text{\Date}
\end{align*}}
V bakalárskej práci sa zameriavame na spôsoby spracovania signálov z vibrácií pri preprave, zachytených
senzorom akcelerácie. Zámerom je extrakcia čŕt záujmu z prúdu vzoriek do udalostí, čím sa redukuje vymieňaný
objem správ v senzorovej sieti. Fyzikálny opis javu nadväzuje na číslicové meraniu zrýchlenia s
MEMS akcelerometrom.

V časovej doméne nahliadame na sledovaný dej ako stochastický proces opísateľný metrikami deskriptívnej štatistiky.
Vibrácie obsahujú viaceré frekvenčné zložky separovateľné Fourierovou a kosínusovou transformáciou 
realizované algoritmom rýchlej Fourierovej transformácie vo viacerých obmenách za aplikovania oknových funkcií. 
Významné okamihy sa prejavujú prudkosťou zmeny alebo výraznou úrovňou lokálneho extrému, ktoré sú odlíšené algoritmami na
hľadanie špičiek vo forme binárnej klasifikácie. Úlohu pri úprave zdrojového signálu zohrávajú tiež filtre 
s konečnou impulznou odozvou. 

Prihliadame na obmedzenia vyplývajúce z nasadenia riešenia na zariadenia
Internetu vecí v kontexte Edge computing architektúry. Konfigurovateľný proces spracovania harmonických zložiek 
trojosovej akcelerácie sa uplatní na senzorovej jednotke komunikujúcej cez WiFi a aplikačný protokol MQTT. 
Firmvér je implementovaný na platforme mikrokontroléra ESP32. Overenie úspešnosti detekcie frekvencií bolo vykonané
na základe syntetických sínusových časových radov a vlastných záznamov z vozidiel verejnej dopravy.
\emptypage

%Anotácia EN
\thispagestyle{empty}
\section*{Annotation}
\UniversityEN \\
\uppercase{\FacultyEN}
\vspace{-8pt}
{\setlength{\mathindent}{0cm}
\begin{align*}
&\text{Degree course:} && \text{\StudyProgrammeEN} \\
&\text{Author:} && \text{\Author} \\
&\text{\ThesisEN:} && \text{\TitleEN} \\
&\text{Supervisor:} && \text{\SupervisorEN} \\
&\text{Departmental advisor:} && \text{\PedagogicalSupervisorEN} \\
&\text{\DateEN}
\end{align*}}
In the bachelor's thesis we focus on signal processing of vibrations during transport, captured using
acceleration sensors. The intention is to extract features of interest from the stream of samples into events, 
thereby reducing the message payload in the sensor network. The physical description of the 
phenomenon is followed by with its digital measurement with the MEMS accelerometer.

In the time domain, we look at the observed phenomenon as a stochastic process expressed by various descriptive statistics. 
The vibrations contain several frequency components separable by Fourier and cosine transform with the Fast Fourier transform 
algorithm in several variants alongside application of window functions. Significant moments are
are manifested therein by the change intensity or the significant level of the local extremes, which are distinguished by 
peak-finding algorithms taking a form of the binary classifier. Finite impulse response filters also play a role in adjusting
the source signal. 

We take into account the limitations of deployment on the Internet of Things devices in the context of the
Edge computing architecture. Configurable process of three-axis acceleration processing into harmonic components 
is applied to the sensor unit communicating via WiFi and the MQTT application protocol. The firmware is implemented on 
the ESP32 microcontroller platform. Verification of the frequency detection success rate was carried out against synthetic 
sinusoidal time series and custom recordings from public transport vehicles.
\emptypage 