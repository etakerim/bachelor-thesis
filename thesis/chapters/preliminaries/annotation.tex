%Anotácia
\thispagestyle{empty}
\section*{Anotácia}
\University \\
\uppercase{\Faculty}
\vspace{-8pt}
{\setlength{\mathindent}{0cm}
\begin{align*}
&\text{Študijný program:} && \text{\StudyProgramme} \\
&\text{Autor:} && \text{\Author} \\
&\text{\Thesis:} && \text{\Title} \\
&\text{Vedúci bakalárskej práce:} && \text{\Supervisor} \\
&\text{Pedagogický vedúci:} && \text{\PedagogicalSupervisor} \\
&\text{\Date}
\end{align*}}
%V bakalárskej práci sa zameriavame na spôsoby spracovania signálov z vibrácií pri preprave, zachytených
%senzorom akcelerácie. Zámerom je extrakcia čŕt záujmu z prúdu vzoriek do udalostí, čím sa redukujú vymieňané
%dáta v senzorovej sieti. 
%
%Analytická časť sa venuje fyzikálnemu opisu a číslicovému meraniu zrýchlenia s MEMS akcelerometrom,
%od ostatné kinematické veličiny. V časovej doméne nahliadame na sledovaný dej ako stochastický proces opísateľný
%metrikami deskriptívnej štatistiky. Významné okamihy sa prejavujú náhlou zmenou oproti predošlému správaniu alebo nadprahovou úrovňou, čo sa zachytáva algoritmami na detekciu špičiek formu binárneho klasifikácie. Vibrácie obsahujú rôzne frekvenčné zložky separovateľné 
%Fourierovou a kosínusovou transformáciou algoritmom rýchlej Fourierovej transformácie vo viacerých obmenách za aplikovania oknových funkcií. 
%Spomenuté sú tiež filtre s konečnou impulznou odozvou zohrávajúce úlohu pri úprave zdrojového signálu pred ďalšími stupňami 
%spracovania. Prihliadame na obmedzenia vyplývajúce z nasadenia riešenia na zariadenia Internetu vecí v kontexte Edge computing architektúry.

\emptypage

%Anotácia EN
\thispagestyle{empty}
\section*{Annotation}
\UniversityEN \\
\uppercase{\FacultyEN}
\vspace{-8pt}
{\setlength{\mathindent}{0cm}
\begin{align*}
&\text{Degree course:} && \text{\StudyProgrammeEN} \\
&\text{Author:} && \text{\Author} \\
&\text{\ThesisEN:} && \text{\TitleEN} \\
&\text{Supervisor:} && \text{\SupervisorEN} \\
&\text{Departmental advisor:} && \text{\PedagogicalSupervisorEN} \\
&\text{\DateEN}
\end{align*}}
%In the bachelor thesis we focus on signal processing of vibrations during transport, captured
%using acceleration sensors. The intention is to extract features of interest from the stream of samples into events, 
%thereby reducing data exchange in the sensor network. The analytical part deals with the physical description of acceleration
%and its digital measurement with the MEMS accelerometer, from which it is possible to obtain other kinematic quantities.
%
%In the time domain, we look at the observed phenomenon as a stochastic process expressed by various descriptive statistics. 
%Significant moments are manifested by a sudden change from the previous behavior or levels above the threshold
%noticed by peak detection algorithms, which are a form of binary classifier. The vibrations contain different frequency 
%components separable by Fourier and cosine transform by the Fast Fourier transform algorithm in several variants with the
%application of window functions.
%
%Finite impulse response filters are also mentioned as playing a role in adjusting the source signal before the next stages of 
%the pipeline. We take into account the limitations of deployment on the Internet of Things devices in the context of the Edge 
%computing architecture.

\emptypage 