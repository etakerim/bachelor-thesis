%Anotácia
\thispagestyle{empty}
\section*{Anotácia}
\University \\
\uppercase{\Faculty}
\vspace{-8pt}
{\setlength{\mathindent}{0cm}
\begin{align*}
&\text{Študijný program:} && \text{\StudyProgramme} \\
&\text{Autor:} && \text{\Author} \\
&\text{\Thesis:} && \text{\Title} \\
&\text{Vedúci bakalárskej práce:} && \text{\Supervisor} \\
&\text{Pedagogický vedúci:} && \text{\PedagogicalSupervisor} \\
&\text{\Date}
\end{align*}}
V bakalárskej práci sa zameriavame na spôsoby spracovania signálov z vibrácií pri preprave, zachytených
senzorom akcelerácie mikromechanickej konštrukcie. Zámerom je extrakcia čŕt záujmu z prúdu vzoriek do udalostí, 
čím sa redukuje objem posielaných dát v senzorovej sieti.

V časovej doméne nahliadame na sledovaný dej ako stochastický proces opísateľný metrikami deskriptívnej štatistiky.
Špecifické okolnosti umožňujú odvodiť zo zrýchlenia ostatné kinematické veličiny numerickou integráciou.
Vibrácie obsahujú frekvenčné zložky separovateľné Fourierovou a kosínusovou transformáciou 
realizovaných algoritmom FFT vo viacerých obmenách za aplikovania oknových funkcií. 
Významné okamihy sa prejavujú prudkosťou zmeny alebo výraznou úrovňou lokálneho extrému, ktoré sú odlíšené binárnou
klasifikáciou. Úlohu pri úprave zdrojového signálu zohrávajú tiež filtre s konečnou impulznou odozvou. 

Prihliadame na obmedzenia vyplývajúce z nasadenia riešenia na zariadenia
Internetu vecí v kontexte Edge computing architektúry. Konfigurovateľný postup spracovania harmonických zložiek 
trojosovej akcelerácie sa uplatní vo firmvéri na bezdrôtovo komunikujúcom mikrokontroléri vzorom Publish-Subscribe.
Úspešnosť detekcie frekvencií sa overuje na základe syntetických sínusových časových radov a vlastných záznamov
z vozidiel verejnej dopravy.
\emptypage

%Anotácia EN
\thispagestyle{empty}
\section*{Annotation}
\UniversityEN \\
\uppercase{\FacultyEN}
\vspace{-8pt}
{\setlength{\mathindent}{0cm}
\begin{align*}
&\text{Degree course:} && \text{\StudyProgrammeEN} \\
&\text{Author:} && \text{\Author} \\
&\text{\ThesisEN:} && \text{\TitleEN} \\
&\text{Supervisor:} && \text{\SupervisorEN} \\
&\text{Departmental advisor:} && \text{\PedagogicalSupervisorEN} \\
&\text{\DateEN}
\end{align*}}
In the bachelor's thesis we focus on signal processing of vibrations during transport, captured using
microelectromechanical acceleration sensor. The intention is to extract features of interest from the stream 
of samples into events, thereby reducing the amount of data sent in the sensor network.

In the time domain, we look at the observed phenomenon as a stochastic process expressed by various descriptive statistics.
Specific circumstances make it possible to derive other kinematic quantities from the acceleration by numerical integration. 
The vibrations contain frequency components separable by Fourier and cosine transform with FFT
algorithm in several variants alongside application of window functions. Significant moments are
are manifested therein by the change intensity or the significant level of the local extremes, which are distinguished by
binary classification. Finite impulse response filters also play a role in adjusting the source signal.

We take into account the limitations of deployment on the Internet of Things devices in the context of the
Edge computing architecture. The configurable harmonic component processing procedure for
three-axis acceleration is applied in the firmware of a microcontroller with wireless communication capability 
by the Publish-Subscribe pattern. The success of frequency detection is verified on the basis of synthetic sinusoidal time series and 
recordings from public transport vehicles.
\emptypage 