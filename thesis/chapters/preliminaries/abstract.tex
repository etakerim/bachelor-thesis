Abstract

An increasing number of connected IoT devices often rely on the aggregation and analytical processing of produced
signal by more powerful nodes, unnecessarily wasting the transmission capacity of the wireless line.
Created pipeline acts directly at the source of the data by adjusting measured levels of vibrational signal
obtained during transport from the MEMS three-axis accelerometer.

In the time domain, the reduction is achieved by descriptive statistics in the sliding windows. We compare three 
simple peak-finding algorithms that reveal significant harmonic components in the frequency domain. We also come up 
with a new stream algorithm to detect changes in the frequency component. We test the acquired efficiency by
firmware implementation on the ESP32 microcontroller allowing remote configuration and message sending in the Message Pack format 
via the MQTT protocol and WiFi. 

We never delay by more than the length of the buffer from the stream of samples. Manually tuned
parameters according to references from synthetically generated spectral profiles, the detector is able to identify
important frequencies in custom public transport vehicle datasets.

Keywords: vibrations, FFT, accelerometer, peak detection


Abstrakt

Narastajúce množstvo pripojených IoT zariadení sa často spolieha na agregáciu a analytické spracovanie produkovaného
signálu výkonnejšími uzlami, čím sa zbytočne plytvá prenosovou kapacitou bezdrôtovej linky. Priamo pri zdroji dát pôsobí 
zostavená ucelená postupnosť krokov na úpravu nameraných úrovní na signále vibrácií pri preprave z MEMS trojosového akcelerometra.
V časovej oblasti sa redukcia dosahuje deskriptívnymi štatistikami po posuvných oknách. 
Porovnávame tri jednoduché algoritmy na hľadanie špičiek, ktoré vo frekvenčnej doméne odhaľujú významné harmonické zložky. 
Prichádzame tiež s novým prúdovým algoritmom na detekciu zmien frekvenčnej zložky. Nadobudnutú efektivitu otestujeme implementáciou 
firmvéru na mikrokontroléri ESP32 umožňujúcom vzdialenú konfiguráciu a posielanie správ vo formáte Message Pack cez protokol 
MQTT a WiFi. Od prúdu zbieraných vzoriek sa neoneskorujeme nikdy o viac ako dĺžku vyrovnávacej pamäte. Manuálnym nastavením 
parametrov podľa referencie od synteticky generovaných spektrálnych profilov je detektor schopná identifikovať dôležité 
frekvencie na vlastných dátových sadách z vozidiel verejnej dopravy.


Kľúčové slová: vibrácie, FFT, akcelerometer, detekcia špičiek