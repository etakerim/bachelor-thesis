\chapter{Zhodnotenie}
Vytvorili sme model spracovania signálu z vibrácii na existujúcom vzdiealene konfigurovateľnosm IoT zariadení.
Známe algoritmy detekcie špičiek sme aplikovali na náš problém s ich porovnaním a
odladením vlastných algoritmov detekcie udalostí.
Rozšírenia sú početné:
\begin{itemize}[noitemsep]
\item Podrobiť (analýze/rozboru) rôzne situácie na vybraných úsekoch ciest opakovane a vypracovať metodiku anotovania datasetu,
kvôli vzájomnému porovnaniu súvislostí
\item Testovanie generátora udalostí na vibrátore s vytváraním známeho frekvenčného priebehu nielen na syntetických dátach
\item Umožniť označiť a pomenovať profil známych javov a dokázať ich odlíšiť pri notifikáciach.
\item umožniť filtrovanie nad frekvenčným rozsahom záujmu
\item Brať do úvahy postupný drift alebo tolerovateľné disturbancie, alebo iné stabilné opakujúce sa vzory a zamyslieť sa
nad ich vhodným oznamovaním
\item Koordinovať viacero senzorov v prevážanom boxe
\item Určenie prevažného priestorového smeru pôsobenia frekvenčných zložiek vibrácii napríklad viacrozmernou FFT
\item Autonómna kalibrácia parametrov hľadania špičiek.
\item Bližšie sa venovať úspore, ktoré by poskytli verzie DCT svojou lepšiou distribúciou energie v koeficientoch
\item  Preskúmať reálne nasadenie a monitorovanie v reálnom čase napríklad cez mobilnú aplikáciu cez Bluetooth komunikáciu,
ktorá sa ukázalo na protokole s väčšími prenosovými rýchlosťami, že by mohli postačovať.
\end{itemize}

\cleardoublepage