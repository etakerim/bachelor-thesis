\chapter{Zhodnotenie} \label{chapter:evaluation}
Zaoberali sme sa meraním vibrácií snímačom trojrozmerného zrýchlenia.
Signálových priebehy sme analyzovali viacerými existujúcimi metódami v časovej a frekvenčnej oblasti.
Podstatou riešeného problému bolo usporiť bezdrôtovo prenášané množstvo informácií, poskytnutím
prehľadu nad sledovanou situáciou upozornením na výskyt a amplitúdu významných frekvencií, či súhrnom
časových úsekov deskriptívnymi štatistikami. Dôležitú etapu v automatizovanej extrakcii
podstatných harmonických zložiek predstavovalo hľadanie špičiek, kde sme porovnali tri odlišné
elementárne koncepcie v literatúre často uplatňované na biologické signály.

Prínos spočíva v navrhnutí postupnosti krokov spracovania dátovej pipeline, špecifikovaním
modifikovateľných parametrov každého stupňa tejto sústavy, a implementácii vzdialene nastaviteľnej
pipeline do firmvéru senzorovej jednotky. Hardvér zariadenia bol poskytnutý už zhotovený.

Na identifikáciu udalostí zmien spektrálneho obsahu vibrácií v čase sme vytvorili nový prúdový algoritmus.
Poradí si s krátkodobými záchvevmi v označenej prítomnosti sekvencie vrcholov minimálnej dĺžky. Účinne pôsobí
v podstate na redukciu šumu z náhlych jednorázových výskytov špičiek. Cez minimálnu testovaciu infraštruktúru
zvolenými sieťovými protokolmi a serializačným formátom sme schopný posielať tematicky kategorizované správy vlastnej
štruktúry. Uznávame, že zotavenie firmvéru z chybových stavov spojené s nedostatkom prostriedkov sa mohlo
hlásiť obšírnejšie ako reštartom alebo zaslaním všeobecnej hlášky.

Validácia detekčných schopností sa opierala o originálny generátor syntetického signálu so šumom podľa
predpisu požadovaných zložiek, ktoré sme na vyjadrenie úspešnosti detekcie pridávali pseudonáhodne.
Stratégie spracovania sa konfrontovali s otrasmi v reálnej premávke datasetmi
zozbieraných s dostupným zariadením.

Program obsadzujúci 65\% voľnej pamäte na inštrukcie je schopný spoľahlivo pracovať s posuvnými
oknami do 512 bodov, v priamočiarejších scenároch kde sú výstupom len udalosti až do 1024 vzoriek.
Posielanie všetkých štatistík sa oplatí nad postupnosť 32 bodov. Počet naraz oznamovaných
zmien frekvencií by za zachovania efektívnosti nemal presiahnuť 12 - 18\% vedierok posuvného
okna. Skutočná priemerná prevalencia 0.5\% a maximálna na úrovni 6\% dokazuje ušetrenie v prenášanom obsahu.
Systém spĺňa kladené obmedzenia na rýchlosť spracovania pri taktovacej frekvencii procesora 160 Hz s ľubovoľnými
nastaveniami. Vyrovnávacia pamäť dokončí obrat cez etapy pipeline pred skompletizovaním ďalšieho posuvného okna.

Vybudovaný model úspešne zakomponovaný na Edge IoT zariadenie tvorí dobrý základ pre početné rozšírenia
vzťahujúce sa hlavne na detailnejšie preskúmanie vplyvov krokov frekvenčnej transformácie a filtrovania,
a možností vyplývajúcich z viaccestnej dátovej pipeline. Mohlo by sa jednať o využitie lepších kompresných vlastností
energetických koeficientov od DCT-IV a MDCT, než FFT. S filtrovaním sa viaže upravenie údaju o tolerancii podľa
aplikovaného kernelu vyhladzovania.

Pokiaľ je predmetom záujmu konkrétny frekvenčný rozsah mohlo by sa autonómne stanoviť ich filtrovanie a najlepšie rozlíšenie.
Rovnako tak sa pre potenciálne produkčné účely ukazuje nepostrádateľnosť samostatnej kalibrácie parametrov hľadania špičiek.
Zaujímavé by bolo prísť s ošetrením cyklického frekvenčného driftu a umožniť označenie profilu známych javov s ich odlíšením pri
notifikáciach. Ďalšia pridaná hodnota by spočívala v určení prevažného priestorového smeru frekvencie alebo koordinácii viacerých senzorov
v spoločnom transportnom boxe.

Dosiaľ dosiahnuté výsledky môžu byť obohatené o silnejšie závery úspešnosti detekcie z opakovaného snímania rôznych stavov
vozidla na vybraných cestách za vypracovania metodiky anotovania skompilovaného datasetu. Za kontrolovanejších podmienok
by sa zariadenie mohlo podrobiť skúšaniu na testovacej lavici.

Softvérové riešenie vychádzajúce z mnohostrannej analýzy problematiky je funkčné a napĺňa intencie
zadania tejto bakalárskej práce.
\cleardoublepage
