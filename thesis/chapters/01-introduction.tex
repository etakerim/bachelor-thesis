\chapter{Úvod}
Inteligentné senzorové systémy zariadení internetu vecí zaznamenávajú obrovskú kvantitu údajov z prostredia, kde
pôsobia. Prúdy vzoriek meraných veličín majú samy osebe nízku informačnú hodnotu. Zbytočne zaťažujú
prenosové pásmo komunikačných kanálov a kapacitu úložísk. Monitorovanie širokého rozsahu kladie požiadavky
na nízke výrobné náklady senzorových jednotiek a dlhodobú výdrž pri napájaní z batérií za minimálnej údržby.
Existuje preto potreba získané dáta spracovať do istej miery už v blízkosti ich zdroja, aby došlo k efektívnemu
využitiu dostupných prostriedkov.

Význam a dôležitosť sledovania vibrácií spočíva v ich výskyte u každého mechanického zariadenia pohybom jednotlivých súčiastok
a trením v ložiskách. Ich nadmerná prítomnosť býva spôsobená opotrebením dielov stroja alebo dôsledkom technických defektov.
Ďalšou oblasťou hojnej prítomnosti vibrácií je preprava osôb a tovaru. Tam sú zapríčinené nerovnosťami povrchu vozovky
alebo koľaje v bode styku s kolesami, či aparátom ovplyvňujúcim pohyb vozidla. Menovite ich vyvoláva točivý moment
spaľovacieho alebo elektrického motora a činnosť brzdového systému.

Detekciou nežiaducich vibrácií v preprave sa dokáže zabezpečiť bezpečnosť pasažierov včasnou výmenou súčiastky,
ktorá by ovplyvnila prevádzkyschopnosť v kritických momentoch. Ich odhalením predchádzame nenávratnému poškodeniu krehkých materiálov,
znehodnoteniu reaktívnych substancií, či ich aktivácii v prípade výbušnín a pyrotechniky. Vibrácie sú súčasťou
nebezpečných prírodných úkazov a správna identifikácia má za následok varovania na evakuáciu obyvateľstva
v oblasti postihnutej zemetrasením, či erupciou sopky, vedúcimi k ohrozenia zdravia osôb a poškodenia majetku.

Vibračný signál je merateľný v digitálnej podobe snímačom pohybového zrýchlenia mikromechanickej konštrukcie, o čom
pojednávame v kapitole \ref{chapter:analysis}. Na postupnosť pozorovaní sa nazerá ako vlnový priebeh,
ktorý sa sprehľadňuje agregačnými, korelačnými a testovacími štatistikami na odhalenie náhlych zmien.
Významne úrovne sa odlišujú od nevýznamných algoritmami na detekciu špičiek. Metódami transformácie do frekvenčnej oblasti
sa objavujú periodicky prítomné zložky. Modely spracovania majú byť nasadené do adekvátnej vrstvy senzorovej siete.
V kapitole \ref{chapter:design} popíšeme hardvér, pre ktorý navrhneme firmvér uskutočňujúci sústavu
krokov na extrakciu udalostí z vektora zrýchlenia a predstavíme dátové sady na validáciu funkčnosti.
Ďalej v kapitole \ref{chapter:implementation} je prezentovaná implementácia najdôležitejších štruktúr a komponentov. Nakoniec
riešenie overíme v kapitole \ref{chapter:verification} a dosiahnuté výsledky okomentujeme v kapitole \ref{chapter:evaluation}.


