\chapter{Úvod}
Inteligentné senzorové systémy zariadení Internetu vecí (IoT) zaznamenávajú obrovskú kvantitu údajov z prostredia, kde
pôsobia. Prúdy vzoriek meraných veličín majú sami o sebe nízku informačnú hodnotu a zbytočne zaťažujú
prenosové pásmo komunikačných kanálov a kapacitu úložísk. Monitorovanie širokého rozsahu kladie požiadavky
na čo najmenšie výrobné náklady senzorových jednotiek a dlhodobú výdrž pri napájaní z batérií za minimálnej údržby.
Existuje preto potreba získané dáta spracovať do istej miery už v blízkosti ich zdroja, aby došlo k efektívnemu
využitiu dostupných prostriedkov.

Zameriame sa na sledovanie vibrácií a dolovanie čŕt záujmu z nich. Význam a dôležitosť sledovania vibrácií spočíva v ich v
výskyte u každého mechanického zariadenia pohybom jednotlivých súčiastok a trením v ložiskách. Ich nadmerná prítomnosť
býva spôsobená opotrebením dielov stroja a dôsledkom technických defektov. Ďalšou oblasťou hojnej prítomnosti vibrácií
je preprava osôb alebo tovaru, kde ich zapríčiňujú nerovnosti povrchu vozovky alebo koľaje v bode styku s kolesami, či aparát
ovplyvňujúci pohyb vozidla, čiže spaľovací alebo elektrický motor a brzdový systém.

Detekciou nežiaducich vibrácií v preprave sa dokáže zabezpečiť bezpečnosť pasažierov včasnou výmenou súčiastky,
ktorá by ovplyvnila prevádzkyschopnosť v kritických momentoch, a predísť nenávratnému poškodeniu krehkých materiálov,
znehodnoteniu reaktívnych substancií, či ich aktivácii v prípade výbušnín a pyrotechniky. Vibrácie sú súčasťou
nebezpečných prírodných úkazov a ich správna identifikácia má za následok varovania pre evakuáciu obyvateľstva
v oblasti postihnutej zemetrasením, či erupciou sopky, vedúcimi k ohrozenia zdravia osôb a poškodenia majetku.

V analýze problematiky sa zapodievame fyzikálnym modelom opisom vibrácií, od čoho sa odvíja metodika ich snímania
akcelerometrami typu MEMS a prevod do číslicovej podoby analógovo-digitálnym prevodníkom. Na spracovanie priebehu
signálu zrýchlenia sa pozrieme z troch hlavných hľadísk.

Integračné metódy umožňujú nadobudnúť odhad o relatívnej rýchlosti a polohe z akcelerácie. Na postupnosti pozorovaní
je možné nahliadať tiež v časovej doméne zužitkovaním základných agregačných a korelačných štatistík na odhalenie
náhlych zmien. Významne vyčlenené úrovne sú detegované
algoritmami na detekciu špičiek za rozličnej úspešnosti. Metódami transformácie do frekvenčnej oblasti sa
objavujú periodicky prítomné zložky, kde je opäť žiaduce upozorniť na momentálne prevládajúci spektrálny obsah.
Modely operujúce s vibračnými dátami by mali nasadené na IoT zariadenie kladúce svoje špecifické nároky a
obmedzenia.

Navrhneme kroky postupov na generovanie významných udalostí z nameranej akcelerácie. Po vyhodnotení úspešnosti
modelov v vlastných dátových sadách bude zámerom implementácia konfigurovateľnej senzorovej IoT jednotky.
