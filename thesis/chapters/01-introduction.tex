\chapter{Úvod}
Inteligentné senzorové systémy zariadení internetu vecí zaznamenávajú obrovskú kvantitu údajov z prostredia, kde
pôsobia. Prúdy vzoriek meraných veličín majú sami o sebe nízku informačnú hodnotu a zbytočne zaťažujú
prenosové pásmo komunikačných kanálov a kapacitu úložísk. Monitorovanie širokého rozsahu kladie požiadavky
na čo najmenšie výrobné náklady senzorových jednotiek a dlhodobú výdrž pri napájaní z batérií za minimálnej údržby.
Existuje preto potreba získané dáta spracovať do istej miery už v blízkosti ich zdroja, aby došlo k efektívnemu
využitiu dostupných prostriedkov.

Význam a dôležitosť sledovania vibrácií spočíva v ich v výskyte u každého mechanického zariadenia pohybom jednotlivých súčiastok
a trením v ložiskách. Ich nadmerná prítomnosť býva spôsobená opotrebením dielov stroja a dôsledkom technických defektov. 
Ďalšou oblasťou hojnej prítomnosti vibrácií je preprava osôb alebo tovaru, kde ich zapríčiňujú nerovnosti povrchu vozovky 
alebo koľaje v bode styku s kolesami, či aparát ovplyvňujúci pohyb vozidla, čiže spaľovací alebo elektrický motor a brzdový systém.

Detekciou nežiaducich vibrácií v preprave sa dokáže zabezpečiť bezpečnosť pasažierov včasnou výmenou súčiastky,
ktorá by ovplyvnila prevádzkyschopnosť v kritických momentoch, a predísť nenávratnému poškodeniu krehkých materiálov,
znehodnoteniu reaktívnych substancií, či ich aktivácii v prípade výbušnín a pyrotechniky. Vibrácie sú súčasťou
nebezpečných prírodných úkazov a ich správna identifikácia má za následok varovania pre evakuáciu obyvateľstva
v oblasti postihnutej zemetrasením, či erupciou sopky, vedúcimi k ohrozenia zdravia osôb a poškodenia majetku.

Vibračný signál je merateľný v digitálnej podobe snímačom pohybového zrýchlenia mikromechanickej konštrukcie. 
Na postupnosť pozorovaní sa nazerá ako vlnový priebeh, ktorý sa sprehľadňuje agregačnými, korelačnými a testovacími 
štatistikami na odhalenie náhlych zmien. Významne úrovne sa odlišujú od nevýznamných algoritmami na detekciu špičiek.
Metódami transformácie do frekvenčnej oblasti sa objavujú periodicky prítomné zložky, kde je opäť žiaduce upozorniť 
na momentálne prevládajúci spektrálny obsah. 

Modely operujúce s dátami prijímanými bezdrôtovým spojením sú podľa  potrieb rozsahu záberu nad skupinami zariadení alebo geografickými oblasťami, či výpočtovej náročnosti sa nasadzujú do adekvátnej vrstvy senzorovej siete. Imperatívnou požiadavkou je čím skôr 
premeniť informačnú záplavu na poznatky užitočné koncovému užívateľovi.
