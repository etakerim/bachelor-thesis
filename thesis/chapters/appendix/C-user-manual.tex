 \thispagestyle{empty}
\setcounter{figure}{0}
\chapter{Používateľská príručka}
\pagenumbering{arabic}
\renewcommand*{\thepage}{C-\arabic{page}}

\section{Inštalačný manuál}
Vývojovou platformou bola Linux distribúcia \emph{Manjaro 21.2.6.} KDE Plasma s jadrom verzie 5.10. 
Uvedenie senzorovej jednotky do prevádzky popisuje ďalej uvedený postup:

\begin{enumerate}
\item {Najprv je potrebné nainštalovať systémové závislosti pre ESP-IDF SDK a MQTT broker:
\begin{lstlisting}[style=messages]
$ sudo pacman -S --needed mosquitto gcc git make flex bison gperf python-pip cmake ninja ccache dfu-util libusb
\end{lstlisting}}

\item {Následne stiahneme knižnice v požadovaných verziách. Pokiaľ použijeme knižnice už
pribalené na digitálnom médiu v priečinku firmvér, \textbf{môžeme vynechať tento krok} a nie je už nutné 
pridávať knižnice do CMake zostavenia a vykonať úpravu DCT v esp-dsp. ESP-IDF používame verzie 4.4.1, ESP-DSP 
je verzie 1.2, a MPack je verzie 1.1:
\begin{lstlisting}[style=messages]
$ git clone -b v4.4.1 --recursive https://github.com/espressif/esp-idf.git
$ git clone -b v1.2.0 https://github.com/espressif/esp-dsp.git
$ wget https://github.com/ludocode/mpack/releases/download/v1.1/mpack-amalgamation-1.1.tar.gz && tar -xvf mpack-amalgamation-1.1.tar.gz
\end{lstlisting}}

\item {Nainštalujeme nástroje používané ESP-IDF na kompiláciu programu, spustením príkazu z priečinku \verb|firmware/esp-idf|:
\begin{lstlisting}[style=messages]
$ ./install.sh esp32
\end{lstlisting}}

\item{Na prehliadanie Jupyter notebookov prieskumných analýz doinštalujeme balíčky pre \emph{Python} 3.10, 
odporúčane vo virtuálnom prostredí, z priečinku \verb|measurements|. Nástroj príkazového riadku na vzdialenú 
konfiguráciu ESP32 má závislosti v osobitom súbore:
\begin{lstlisting}[style=messages]
$ pip install -r requirements.txt
$ pip install -r test-requirements.txt
\end{lstlisting}}

\item{MQTT broker \emph{Eclipse Mosquitto} v2.0.14 potrebuje na povolenie pripájania klientov v lokálnej sieti
aplikovať konfiguráciu zo súboru \verb|mosquitto.conf| a následne musí byť služba reštartovaná.
Príkazy spúštame z koreňového adresára.
\begin{lstlisting}[style=messages]
$ mv firmware/vibration-analyzer/config/mosquitto.conf 
     /etc/mosquitto/mosquitto.conf 
$ sudo systemctl restart mosquitto
$ sudo systemctl status mosquitto
\end{lstlisting}}

\item {Záznam na SD kartu umožníme umiestnením súboru nastavení OpenLog \verb|config.txt| 
z priečinku \verb|firmware/vibration-analyzer/conf| na pamäťovú kartu.}
\end{enumerate}


\section{Nahratie firmvéru}
\begin{enumerate}
\item {Prinesením IoT zariadenia do novej senzorovej siete v neznámom stave sa očakáva určenie 
prihlasovacích údajov WiFi prístupového bodu a URL lokácie pre MQTT broker v štruktúre
\verb|login| v súbore \verb|main.c| a priečinku \verb|firmware/vibration-analyzer/main/src|.
Odporúča sa, aby URL adresa pozostávala z doménového mena, ale prípustná je aj IP adresa servera
brokera (Zistená napr. cez \verb|ip addr|). Požadovanú úvodnú systémovú konfiguráciu je možné
ovplyvniť tam isto, v štruktúre \verb|conf|. Príklad sieťového pripojenia:
\begin{lstlisting}[style=messages, morekeywords = {"eduroam","12345","mqtt://192.168.1.10:1883"}]
static Provisioning login = {
    .wifi_ssid="eduroam",
    .wifi_pass="12345",
    .mqtt_url="mqtt://192.168.1.10:1883"
};
\end{lstlisting}}

\item V súbore \verb|main.c| musí na nahratie pravidiel konfigurácie, ako atribútu do asociatívnej časti nevolatilnej pamäte, 
dočasne \textbf{odkomentovaná} direktíva \verb|FACTORY_RESET|.

\item ESP32 pripojíme cez Micro USB na počítač. Pomocný napaľovač nastavení umiestníme do flash pamäte mikrokontroléra spustením nasledovných príkazov z priečinku \verb|firmware/vibration-analyzer|. Sériový port určíme podľa aktuálne priradeného názvu.
\begin{lstlisting}[style=messages]
$ . ../esp-idf/export.sh
$ idf.py build
$ idf.py -p /dev/ttyUSB0 flash
\end{lstlisting}

\item Samotný firmvér nahráme na ESP32 po opätovnom \textbf{zakomentovaní} \verb|FACTORY_RESET|:
\begin{lstlisting}[style=messages]
$ idf.py build
$ idf.py -p /dev/ttyUSB0 flash
\end{lstlisting}

\item Schopnosť prihlásiť sa na WiFi prístupový bod overíme:
\begin{lstlisting}[style=messages]
idf.py -p /dev/ttyUSB0 monitor
\end{lstlisting}
Posledný riadok výpisu pre úspešné prihlásenie má vyzerať podobne tomuto:
\begin{lstlisting}[style=messages]
W (858) wifi:<ba-add>idx:0 (ifx:0, 98:da:c4:79:6a:fa), tid:0, ssn:3, winSize:64
\end{lstlisting}

\item Senzorovú jednotku môžeme odpojiť od počítača a pripojiť na samostatný zdroj napájania.
Pred zapnutím ESP32 a po prihlásení klienta na MQTT topic ,,syslog'' by sme mali obdržať reťazec ,,imu started''.
Následne sa začne zber a vyhodnocovanie zrýchlenia zo snímača.
\begin{lstlisting}[style=messages]
$ mosquitto_sub -h localhost -t imu/1/syslog
imu started
\end{lstlisting}
\end{enumerate}

\section{Konfiguračný klient}
V priečinku \verb|firmware/vibration-analyzer/tests| sa nachádza nástroj
na vzdialenú interaktívnu konfiguráciu senzorovej jednotky:
\begin{lstlisting}[style=messages]
$ python config_tool.py
\end{lstlisting}

Nástroj podporuje uvedenú sadu príkazov. Predvolené dodatočne dopytovaných 
hodnôt sú v hranatých zátvorkách a na ich potvrdenie stačí stlačiť riadkovač.

\begin{itemize}[noitemsep, topsep=0pt]
	\item \textbf{end} - Ukončenie programu konfiguračného klienta
	\item \textbf{connect} - Pripojenie k serveru so službou MQTT broker. Ponúkané možnosti:
		\begin{itemize}
			\item ,,Device ID [1]'' - ID senzorovej jednotky na filtrovanie komunikácie. Naraz je umožnené spravovanie len 
			jedného kozového uzla
      		\item ,,Broker IP [192.168.1.103]'' - IP adresa alebo doménové meno MQTT brokera
        	\item ,,Broker Port [1883]'' - Číslo TCP portu MQTT brokera
		\end{itemize}
	\item \textbf{disconnect} - Odpojenie sa od nastavovania konkrétneho zariadenia
	\item \textbf{$\wedge$C} - (Ctrl+C) Zrušenie priebehu aktuálneho príkazu
	\item \textbf{set} - Zmena konfigurácie zadaná vo formáte JSON na výzvu: \verb|config>|
	\item \textbf{config} - Dopyt konfigurácie prítomnej na senzorovej jednotke
	\item \textbf{login} - Zmena sieťových nastavení zadaných vo formáte JSON na výzvu: \verb|login>|
	\item \textbf{credentials} - Dopyt nastavených údajov o sieťovom pripojení
	\item \textbf{topic} - Odoberanie MQTT témy po stanovení časový interval ohľadom zariadením produkovaných údajov. 
	Téma sa zadáva bez prefixu na výzvu: \verb|topic>| 
\end{itemize}

\section{Replikácia experimentov}
Výsledky experimentov na implementovanom firmvéri sa riadili podľa presne stanoveného poradia
uplatňovania pravidiel konfigurácie za stabilných podmienok. Najdôležitejšie
prepínače kompilátora, vložené do \emph{sdkconfig} nástrojom \verb|idf.py menuconfig|, ktoré 
sa použili plošne sú:
\begin{itemize}[noitemsep,topsep=0pt]
    \item Optimalizácia na veľkosť: -Os \\(\emph{CONFIG\_COMPILER\_OPTIMIZATION\_SIZE=y})
   	\item Taktovacia frekvencia: 160 MHz  \\ (\emph{CONFIG\_ESP32\_DEFAULT\_CPU\_FREQ\_MHZ=160})
    \item Interval plánovania operačného systému: 100 Hz\\  (\emph{CONFIG\_FREERTOS\_HZ=100})
\end{itemize}
    
Po odkomentovaní príslušnej direktívy na podmienenú kompiláciu v \emph{main.c} podľa typu experimentu
bol takto pozmenený firmvér nahratý na zariadenie. Predvolená konfigurácia bola
cez \emph{config\_tool.py} obnovená na začiatku pokusu vždy rovnaká, zachytená v technickej dokumentácii
k MQTT témam. Následne sa príkazom \verb|set| konfigurácia pozmeňovala a odčítavali
sa výpisy na konzolu aj do súboru. Realizovalo sa zakaždým 10 meraní, s ktorých bol vypočítaný aritmetický priemer.
\begin{lstlisting}[style=messages]
$ idf.py monitor | tee experiment.txt
\end{lstlisting}
\bigbreak

Prieskumná analýza datasetov a vyhodnotenie úspešnosti pipeline na syntetickom signále
sa nachádza v Jupyter notebookoch, ktoré sa otvárajú z priečinku \verb|measurements| príkazom:
\begin{lstlisting}[style=messages]
$ jupyter notebook
\end{lstlisting}


\subsection*{Experiment: Časová efektivita algoritmov na spracovanie signálu}
Direktíva podmienenej kompilácie: \verb|MEMORY_MEASUREMENT|

Konfigurácie:
\begin{lstlisting}[style=messages, numbers=left, numberstyle=\tiny]
{"sensor": {"n": 8}, "tsmooth": {"n": 8}, "fsmooth": {"n": 8}}
{"sensor": {"n": 16}, "tsmooth": {"n": 16}, "fsmooth": {"n": 16}}
{"sensor": {"n": 32}, "tsmooth": {"n": 32}, "fsmooth": {"n": 32}}
{"sensor": {"n": 64}, "tsmooth": {"n": 64}, "fsmooth": {"n": 64}}
{"sensor": {"n": 128}, "tsmooth": {"n": 128}, "fsmooth": {"n": 128}}
{"sensor": {"n": 256}, "tsmooth": {"n": 256}, "fsmooth": {"n": 256}}
{"sensor": {"n": 512}, "tsmooth": {"n": 512}, "fsmooth": {"n": 512}}
{"sensor": {"n": 1024}, "tsmooth": {"n": 1024}, "fsmooth": {"n": 1024}}
\end{lstlisting}

Filtrovanie relevantných riadkov:
\begin{lstlisting}[style=messages, language=sh]
$ sed -rn `s/^(.*)(MEM.*)$/\2/p' experiment.txt 
> memory_usage.csv
\end{lstlisting}

\subsection*{Experiment: Časová efektivita algoritmov na spracovanie signálu}
Direktíva podmienenej kompilácie: \verb|EXECUTION_TIME_ALGORITHMS|.

\noindent Zmena konfigurácie oproti predvolenej:
\begin{lstlisting}[style=experiments]
{"sensor": {"axis": [false, false, true]}}
\end{lstlisting}

\noindent Odskúšané algoritmy frekvenčnej transformácie a detekcie špičiek:
\begin{lstlisting}[style=experiments]
{"sensor": {"n": 32, "fs": 16}, "transform": {"f": "dft"}, "peak": {"strategy": "neighbours"}}
{"sensor": {"n": 32, "fs": 16}, "transform": {"f": "dft"}, "peak": {"strategy": "zero_crossing"}}
{"sensor": {"n": 32, "fs": 16}, "transform": {"f": "dft"}, "peak": {"strategy": "hill_walker"}}
{"sensor": {"n": 32, "fs": 16}, "transform": {"f": "dct"}, "peak": {"strategy": "neighbours"}}
\end{lstlisting}

\noindent Pre každú predošlú stratégiu sa odmerali postupne rozličné dĺžky okien:
\begin{lstlisting}[style=experiments]
{"sensor": {"n": 64, "fs": 32}}
{"sensor": {"n": 128, "fs": 64}}
{"sensor": {"n": 256, "fs": 128}}
{"sensor": {"n": 512, "fs": 256}}
{"sensor": {"n": 1024, "fs": 512}}
\end{lstlisting}


\subsection*{Experiment: Časová efektivita vyhladzovacieho filtra}
\noindent Direktíva podmienenej kompilácie: \verb|EXECUTION_TIME_SMOOTHING|.

\noindent Zmena konfigurácie oproti predvolenej:
\begin{lstlisting}[style=experiments]
{"sensor": {"fs": 256, "n": 512, "axis": [false, false, true]},  {"tsmooth": {"on": true}}
\end{lstlisting}

\noindent Konfigurácie:
\begin{lstlisting}[style=experiments]
{"tsmooth": {"n": 4, "repeat": 1}}
{"tsmooth": {"n": 16, "repeat": 1}}
{"tsmooth": {"n": 64, "repeat": 1}}
{"tsmooth": {"n": 4, "repeat": 4}}
{"tsmooth": {"n": 16, "repeat": 4}}
{"tsmooth": {"n": 64, "repeat": 4}}
{"tsmooth": {"n": 4, "repeat": 8}}
{"tsmooth": {"n": 16, "repeat": 8}}
{"tsmooth": {"n": 64, "repeat": 8}}
\end{lstlisting}


\subsection*{Experiment: Časová efektivita dátovej pipeline}
\noindent Direktíva podmienenej kompilácie: \verb|EXECUTION_TIME_PIPELINE|.

\noindent Zmena konfigurácie oproti predvolenej:
\begin{lstlisting}[style=messages]
{
 "sensor": {"fs": 16,"n": 32, "axis": [false, false, true]}, 
 "stats": {"min": false, "max": false, "rms": false, 
 		   "avg": false, "var": false, "std": false, 
 		   "skew": false, "kurt": false, "med": false, 
 		   "mad": false, "corr": false},
 "transform": {"log": true}
}
\end{lstlisting}

\noindent Nastavenie na začiatku série meraní. \textbf{Tabuľka A}:
\begin{lstlisting}[style=experiments]
{"sensor": {"axis": [false, false, true]}, "transform": {"f": "dft"}}
{"sensor": {"axis": [false, false, true]}, "transform": {"f": "dct"}}
\end{lstlisting}

\textbf{Tabuľka B}:
\begin{lstlisting}[style=experiments]
{"sensor": {"axis": [true, true, true]}, "transform": {"f": "dft"}}
{"sensor": {"axis": [true, true, true]}, "transform": {"f": "dct"}}
\end{lstlisting}

\textbf{Tabuľka C}:
\begin{lstlisting}[style=experiments]
{"sensor": {"axis": [false, false, true]}, "stats": {"min": true, "max": true, "rms": true, "avg": true, "var": true, "std": true, "skew": true, "kurt": true, "med": true, "mad": true, "corr": true},  "logger": {"samples": "f", "stats": true}, "transform": {"f": "dft"}}
\end{lstlisting}

\textbf{Tabuľka D}:
\begin{lstlisting}[style=experiments]
{"sensor": {"axis": [true, true, true]}, "transform": {"f": "dft"}}
\end{lstlisting}

\noindent Po každom riadku z predošlých úvodných nastavení nasleduje 9 obmien, kedy
sa postupne upravuje veľkosť okna a algoritmus detekcie špičiek.
\begin{lstlisting}[style=experiments]
{"sensor": {"fs": 16, "n": 32}, "peak": {"strategy": "neighbours"}}
{"peak": {"strategy": "zero_crossing"}}
{"peak": {"strategy": "hill_walker"}}
{"sensor": {"fs": 128, "n": 256}, "peak": {"strategy": "neighbours"}}
{"peak": {"strategy": "zero_crossing"}}
{"peak": {"strategy": "hill_walker"}}
{"sensor": {"fs": 512, "n": 1024}, "peak": {"strategy": "neighbours"}}
{"peak": {"strategy": "zero_crossing"}}
{"peak": {"strategy": "hill_walker"}}
\end{lstlisting}

\noindent Výpisy zo separátnych experimentov A, B, C, D sú prehľadané na výskyt riadkov s časmi a odtiaľ 
manuálne upravené do finálnej podoby
\begin{lstlisting}[style=messages, language=sh]
sed -rn `/^(.*)(main:.*)$/p' pipeline_time-X.txt 
> pipeline_time-X_filter.csv
\end{lstlisting}
