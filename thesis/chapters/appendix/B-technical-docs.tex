 \thispagestyle{empty}
\chapter{Technická dokumentácia}
\pagenumbering{arabic}
\renewcommand*{\thepage}{B-\arabic{page}}

\section{Doxygen dokumentácia}
Nástroj Doxygen zhotovil podľa komentárov v zdrojom kóde prehľadnú technickú dokumentáciu,
ktorá je po typografickej úprave súčasťou tejto prílohy.

\subsection{Moduly}
\begin{itemize}[noitemsep, topsep=0pt]
	\item \textbf{Udalosti} (\ref{modules:events}) - Binárne klasifikátory na označenie význačných úrovní v posuvnom okne vzoriek. 
	Zdrojový kód: \verb|events.h|, \verb|events.c|.
	\item \textbf{Akcelerometer} (\ref{modules:imu}) - Adaptér pre SPI rozhranie senzora LSM9DS1 lineárnej 3D akcelerácie (IMU).
	Zdrojový kód: \verb|inertial_unit.h|, \verb|inertial_unit.c|
	\item \textbf{Hardvérové adaptéry} (\ref{modules:hardware}) - Rozhrania na komunikáciu s perifériami.
	Zdrojový kód: \verb|peripheral.h|, \verb|peripheral.c|
	\item \textbf{Dátová pipeline} (\ref{modules:pipeline}) - Fázy spracovania zdrojového signálu. Zdrojový kód: \verb|pipeline.h|
	\begin{itemize}[noitemsep, topsep=0pt]
		\item \textbf{Oknové funkcie} - \verb|window.c|
		\item \textbf{Správa pamäti pipeline} - \verb|pipeline.c|
		\item \textbf{Fázy spracovania oknovaného signálu} - \verb|pipeline.c|
		\item \textbf{Message Pack serializácia} - Serializácia a parsovanie nameraných dát a konfigurácie. \verb|serialize.c|
	\end{itemize}
	\item \textbf{Deskriptívna štatistika} (\ref{modules:statistics}) - výpočet popisných štatistík. Zdrojový kód: \verb|statistics.h|, \verb|statistics.c|
\end{itemize}



\subsection{Udalosti} \label{modules:events}
Modul s binárnymi klasifikátormi na označenie význačných úrovní v posuvnom okne vzoriek.

\subsubsection*{Dátové štruktúry}
\textbf{struct SpectrumEvent} - Stav udalosti frekvenčného vedierka.
\begin{itemize}[noitemsep, topsep=0pt]
	\item \textbf{SpectrumEventAction action}: Značka vymedzenia udalosti: žiadna, začiatok, koniec
	\item \textbf{uint32\_t start}: Časová pečiatka alebo poradie posuvného okna, kedy začala aktuálne aktívna udalosť
	\item \textbf{uint32\_t duration}: Trvanie aktívnej udalosti v počte posuvných okien
	\item \textbf{int32\_t last\_seen}: Počet posuvných okien do minulosti, kedy bol detegovaný posledný výskyt špičky vo frekvencii
	\item \textbf{float amplitude}: riemerná amplitúda frekvenčného vedierka počas trvania udalosti
\end{itemize}

\subsubsection*{Enumerácie}
\textbf{enum SpectrumEventAction} - Časové vymedzenie udalosti frekvenčného spektra.
	\begin{itemize}[noitemsep, topsep=0pt, label=$\star$]
		\item \verb|SPECTRUM_EVENT_NONE|: Udalosť prebieha alebo žiadna nie je aktívna.
		\item \verb|SPECTRUM_EVENT_START|: Značka začiatku udalosti.
		\item \verb|SPECTRUM_EVENT_FINISH|: Značka ukončenia udalosti.
	\end{itemize}

\subsubsection*{Funkcie}
\begin{lstlisting}[style=docs]
void find_peaks_above_threshold (
    bool *peaks, const float *y, int n, float t
) 
\end{lstlisting}
    Hľadanie špičiek absolútnou prahovou úrovňou amplitúdy signálu. \\
\textbf{Parametre}:
\begin{itemize}[noitemsep, topsep=0pt, label=$\star$]
	\item \textbf{peaks} (out): Nájdené špičky v signále. Dĺžka poľa musí byť rovnaká počtu vzoriek
	\item \textbf{y} (in): Vzorky signálu 
	\item \textbf{n} (in): Počet vzoriek 
	\item \textbf{t} (in) Prahová úroveň amplitúdy. Odporúčaná hodnota je z prípustných hodnôt rozsahu pre vzorky 
\end{itemize}
\bigbreak
\hrule

\begin{lstlisting}[style=docs]
void find_peaks_neighbours (
    bool *peaks, const float *y, int n,  int k, 
    float e, float h_rel, float h
)
\end{lstlisting}
   Hľadanie špičiek s algoritmom význačnosti vrchola spomedzi susedov. $ f[t-i] < f[t] > f[t+i],\quad \forall i \in 1, 2, ..., k $ \\
\textbf{Parametre}:
\begin{itemize}[noitemsep, topsep=0pt, label=$\star$]
	\item \textbf{peaks} (out): Nájdené špičky v signále. Dĺžka poľa musí byť rovnaká počtu vzoriek
	\item \textbf{y} (in): Vzorky signálu 
	\item \textbf{n} (in): Počet vzoriek 
	\item \textbf{k} (in): Počet najbližších uvažovaných susedov na každú zo strán od kandidátnej špičky $[t-k; t+k]$;  Rozsah: $[1, n / 2]$
	\item \textbf{e} (in): Relatívna tolerancia pre vyššiu úroveň v susedstve od vrchola
	\item \textbf{h\_rel} (in): Minimálna relatívna výška špičky v susedstve
 	\item \textbf{h} (in): Absolútna prahová úroveň amplitúdy špičky
\end{itemize}
\bigbreak
\hrule

\begin{lstlisting}[style=docs]
void find_peaks_zero_crossing (
	bool *peaks, const float *y, int n, int k, float slope
)
\end{lstlisting}
   Hľadanie špičiek s algoritmom prechodu nulou do záporu. $Delta f[i] = 0$ \\
\textbf{Parametre}:
\begin{itemize}[noitemsep, topsep=0pt, label=$\star$]
	\item \textbf{peaks} (out): Nájdené špičky v signále. Dĺžka poľa musí byť rovnaká počtu vzoriek
	\item \textbf{y} (in): Vzorky signálu 
	\item \textbf{n} (in): Počet vzoriek 
	\item \textbf{k} (in): Dĺžka sečnice na každú stranu od kandidátnej špičky $[t-k; t+k]$; Rozsah: $[1, n / 2]$
	\item \textbf{slope} (in): Prahová úroveň strmosti kopca, čiže rozdielu medzi hladiny medzi koncami sečnice. Rozsah: $slope \geq 0 $
\end{itemize}
\bigbreak
\hrule

\begin{lstlisting}[style=docs]
void find_peaks_hill_walker (
	bool *peaks, const float *y, int n, 
	float tolerance, int hole, 
	float prominence, float isolation
)
\end{lstlisting}
   Hľadanie špičiek s modikovaným algoritmom horského turistu. \\
\textbf{Parametre}:
\begin{itemize}[noitemsep, topsep=0pt, label=$\star$]
	\item \textbf{peaks} (out): Nájdené špičky v signále. Dĺžka poľa musí byť rovnaká počtu vzoriek
	\item \textbf{y} (in): Vzorky signálu 
	\item \textbf{n} (in): Počet vzoriek 
	\item \textbf{tolerance} (in): Prahová úroveň vo vertikálnej osi. Rozsah: $[min(y), max(y)]$
 	\item \textbf{hole} (in): Prahová úroveň v horizontálnej osi. Rozsah: $[0, n]$
 	\item \textbf{prominence} (in):  Relatívna výška oproti predošlej navštívenej doline. Rozsah: $[0, \max(y) - \min(y)]$
 	\item \textbf{isolation} (in)  Vzdialenosť ku najbližšiemu predošlému vrcholu. Rozsah: $[0, n]$
\end{itemize}
\bigbreak
\hrule

\begin{lstlisting}[style=docs]
void event_init (SpectrumEvent *events, uint16_t bins)
\end{lstlisting}
   Nastavenie počiatočného stavu online detektora udalostí vo frekvenciách. \\
\textbf{Parametre}:
\begin{itemize}[noitemsep, topsep=0pt, label=$\star$]
	\item \textbf{events} (out): Pole udalostí frekvenčného spektra
	\item \textbf{bins} (in): Počet frekvenčných vedierok a zároveň dĺžka poľa udalostí
\end{itemize}
\bigbreak
\hrule

\begin{lstlisting}[style=docs]
size_t event_detection (
	size_t t, SpectrumEvent *events, const bool *peaks, 
	const float *spectrum, uint16_t bins,
	uint16_t min_duration, uint16_t time_proximity
)
\end{lstlisting}
   Online detekcia zmien v časovom priebehu frekvenčných spektier. \\
\textbf{Parametre}:
\begin{itemize}[noitemsep, topsep=0pt, label=$\star$]
\item \textbf{t}: Poradové číslo posuvného okna. S každým ďalším volaním funkcie musí byť navýšené o 1 
\item \textbf{events}: Pole udalostí frekvenčného spektra s dĺžkou počtu vedierok
\item \textbf{peaks} (in): Nájdené špičky vo aktuálnom frekvenčnom spektre jedným z klasifikátorov find\_peak\_*
\item \textbf{spectrum} (in): Frekvenčné spektrum aktuálneho posuvného okna vzoriek na zistenie priemernej amplitúdy udalostí
\item \textbf{bins} (in): Počet frekvenčných vedierok
\item \textbf{min\_duration} (in): Minimálne trvanie po koľkých oknách je vyhlásená špička za udalosť. Udáva oneskorenie vyhlásenia začiatku udalosti.
\item \textbf{time\_proximity} (in): Najväčšia vzdialenosť súvislej udalosti v počte okien. Najväčšia dĺžka časovej medzery medzi nájdenými špičkami. Udáva oneskorenie vyhlásenia ukončenia udalosti.
 
\item \textbf{Návratová hodnota}: Počet detegovaných zmien, čiže začiatočných a koncov udalostí v danom spektre posuvného okna
\end{itemize}
\bigbreak
\hrule



\subsection{Akcelerometer} \label{modules:imu}
Modul adaptéra pre SPI rozhranie senzora LSM9DS1 lineárnej 3D akcelerácie (IMU)

\subsubsection*{Dátové štruktúry}
\textbf{struct InertialUnit} - Inerciálna meracia jednotka.
\begin{itemize}[noitemsep, topsep=0pt]
	\item \textbf{gpio\_num\_t clk}: GPIO pin SPI hodinového signálu
	\item \textbf{gpio\_num\_t miso}: GPIO pin SPI Master In Slave Out
	\item \textbf{gpio\_num\_t xgcs}: GPIO pin SPI Chip select akcelometra a gyroskopu
	\item \textbf{gpio\_num\_t mcs}: GPIO pin SPI Chip select magnetometra
	\item \textbf{gpio\_num\_t int1}: GPIO vstup prerušenia č.1
	\item \textbf{gpio\_num\_t int2}: GPIO vstup prerušenia č.2
	\item \textbf{gpio\_num\_t en\_data}: GPIO vstup príznaku pripravených dát
	\item \textbf{gpio\_num\_t isr\_int1}: Podprogram prerušenia pre INT č.1
	\item \textbf{gpio\_num\_t isr\_int2}: Podprogram prerušenia pre INT č.2
	\item \textbf{spi\_host\_device\_t spi}: SPI zbernica
	\item \textbf{spi\_device\_handle\_t dev}: SPI zariadenie pre akcelerometer
	\item \textbf{AccelerationPrecision precision}: Zistená citlivosť akcelerometra v mg/LSB
\end{itemize}

\subsubsection*{Enumerácie}
\textbf{enum AccelerationRange} - Dynamický rozsah akcelerometra v g.
	\begin{itemize}[noitemsep, topsep=0pt, label=$\star$]
		\item \verb|IMU_2G|: Rozsah $\pm 2\;\mathrm{g}  = \pm 19.6133 \;\mathrm{m/s^2}$
		\item \verb|IMU_4G|: Rozsah $ \pm 4\;\mathrm{g} = \pm 39.2266 \;\mathrm{m/s^2}$
		\item \verb|IMU_8G|: Rozsah $ \pm 8\;\mathrm{g}  = \pm 78.4532 \;\mathrm{m/s^2}$
		\item \verb|IMU_16G|: Rozsah $ \pm 16\;\mathrm{g}  = \pm 156.9064 \;\mathrm{m/s^2}$
		\item \verb|IMU_RANGE_COUNT|: Počet možností nastavenia rozsahu na účely serializácie
	\end{itemize}
\bigbreak
\noindent\textbf{typedef float AccelerationPrecision} - Citlivosť akcelerometra podľa dynamického rozsahu v mg/LSB.
	

\subsubsection*{Funkcie}

\begin{lstlisting}[style=docs]
esp_err_t imu_setup(InertialUnit *imu)
\end{lstlisting}
   Inicializácia senzora lineárnej akcelerácie. \\
\textbf{Parametre}:
\begin{itemize}[noitemsep, topsep=0pt, label=$\star$]
	\item \textbf{imu}: Senzor
	\item \textbf{Návratová hodnota}: Úspešnosť inicializácie senzora
\end{itemize}
\bigbreak
\hrule

\begin{lstlisting}[style=docs]
void imu_acceleration_range(
	InertialUnit *imu, AccelerationRange range
)
\end{lstlisting}
   Nastavenie dynamického rozsah lineárneho 3D akcelerometra v g. \\
\textbf{Parametre}:
\begin{itemize}[noitemsep, topsep=0pt, label=$\star$]
	\item \textbf{imu}: Senzor
	\item \textbf{range}: Dynamický rozsah akcelerometra
\end{itemize}
\bigbreak
\hrule

\begin{lstlisting}[style=docs]
void imu_output_data_rate(InertialUnit *imu, uint16_t fs)
\end{lstlisting}
   Nastavenie výstupného dátového toku (ODR) akcelerometra podľa vzorkovanej frekvencie. \\
\textbf{Parametre}:
\begin{itemize}[noitemsep, topsep=0pt, label=$\star$]
	\item \textbf{imu}: Senzor
	\item \textbf{fs}: Vzorkovacia frekvencia v Hz. Hardvér povoľuje max. ODR 956 Hz
\end{itemize}
\bigbreak
\hrule

\begin{lstlisting}[style=docs]
void imu_output_data_rate(InertialUnit *imu, uint16_t fs)
\end{lstlisting}
   Meranie aktuálnej hodnoty 3D akcelerácie v $m/s^2$. \\
\textbf{Parametre}:
\begin{itemize}[noitemsep, topsep=0pt, label=$\star$]
	\item \textbf{imu}: Senzor
	\item \textbf{x} (out): Zrýchlenie v osi x. Rozsah je podľa nastavenia dynamického rozsahu.
 	\item \textbf{y} (out): Zrýchlenie v osi y
    \item \textbf{z} (out): Zrýchlenie v osi z
\end{itemize}


\subsection{Hardvérové adaptéry} \label{modules:hardware}

\subsubsection*{Dátové štruktúry}
\textbf{struct OpenLog} - SparkFun OpenLog - zaznenávač údajov na SD kartu cez sériovú linku.
\begin{itemize}[noitemsep, topsep=0pt]
	\item \textbf{gpio\_num\_t vcc}: GPIO pin na ovládanie napájania cez FET tranzistor
	\item \textbf{uint8\_t uart}: Číslo UART rozhrania
	\item \textbf{gpio\_num\_t rx}: GPIO pin UART RX
	\item \textbf{gpio\_num\_t tx}: GPIO pin UART TX
	\item \textbf{int baudrate}: Baudová rýchlosť komunikácie. Rovnaká rýchlosť musí byť nastavená v \verb|config.txt| na SD karte
	\item \textbf{int buffer}: Dĺžka vyrovnávacej pamäte pre UART
\end{itemize}
\bigbreak

\noindent\textbf{struct MqttAxisTopics} - MQTT témy pre os zrýchlenia.
\begin{itemize}[noitemsep, topsep=0pt]
	\item \textbf{char stats[TOPIC\_LENGTH]}: Názov témy pre štatistické údaje
	\item \textbf{char spectra[TOPIC\_LENGTH]}: Názov témy pre frekvenčné spektrum
	\item \textbf{char events[TOPIC\_LENGTH]}: Názov témy pre udalosti zmeny spektra
\end{itemize}


\subsubsection*{Funkcie}
\begin{lstlisting}[style=docs]
void axis_mqtt_topics (MqttAxisTopics *topics, int axis)
\end{lstlisting}
   Poskladanie názvu MQTT tém pre odosieanie dát o osi akcelerácie. \\
\textbf{Parametre}:
\begin{itemize}[noitemsep, topsep=0pt, label=$\star$]
	\item topics (out): MQTT témy zložené s označením osi x, y, z
	\item axis (in): Index osi vektora akcelerácie: 0, 1, 2 
\end{itemize}
\bigbreak
\hrule

\begin{lstlisting}[style=docs]
void clock_reconfigure (uint16_t frequency)
\end{lstlisting}
   Zmena frekvencie časovača. \\
\textbf{Parametre}:
\begin{itemize}[noitemsep, topsep=0pt, label=$\star$]
	\item \textbf{frequency} (in):	Vzorkovacia frekvencia v Hz 
\end{itemize}
\bigbreak
\hrule

\begin{lstlisting}[style=docs]
void clock_setup (uint16_t frequency, timer_isr_t action)
\end{lstlisting}
   Spustenie časovača na vzorkovanie signálu. \\
\textbf{Parametre}:
\begin{itemize}[noitemsep, topsep=0pt, label=$\star$]
	\item \textbf{frequency} (in): Vzorkovacia frekvencia v Hz
	\item \textbf{action} (in):	Obsluha prerušenia časovača s predpisom: \\ \verb|bool IRAM_ATTR f(void *args)|
\end{itemize}
\bigbreak
\hrule

\begin{lstlisting}[style=docs]
void mqtt_event_handler (
	void *handler_args, esp_event_base_t base, 
	int32_t event_id, void *event_data
)
\end{lstlisting}
Predbežná deklarácia spätného volania. Implementáciu musí poskytnúť hlavný program. Používa sa v \verb|mqtt_setup()|. \\
\bigbreak
\hrule

\begin{lstlisting}[style=docs]
esp_mqtt_client_handle_t mqtt_setup (
	const char *broker_url
)
\end{lstlisting}
   Pripojenie sa k MQTT broker a zaregistrovanie spätného volanie pre všetky udalosti. \\
\textbf{Parametre}:
\begin{itemize}[noitemsep, topsep=0pt, label=$\star$]
	\item \textbf{broker\_url}: URL MQTT broker 
\end{itemize}
\bigbreak
\hrule

\begin{lstlisting}[style=docs]
esp_err_t nvs_load (
	Configuration *conf, Provisioning *login
)
\end{lstlisting}
   Načítanie nastavení systému z nevolatilného úložiska. \\
\textbf{Parametre}:
\begin{itemize}[noitemsep, topsep=0pt, label=$\star$]
	\item \textbf{conf} (out): Globálne nastavenia spracovania dát
	\item \textbf{login} (out): Nastavenie sieťového pripojenia
\end{itemize}
\bigbreak
\hrule

\begin{lstlisting}[style=docs]
esp_err_t nvs_save_config (const Configuration *conf)
\end{lstlisting}
   Uloženie nastavení systému na nevolatilné úložisko. \\ Používa sa v \verb|mqtt_event_handler()|. \\
\textbf{Parametre}:
\begin{itemize}[noitemsep, topsep=0pt, label=$\star$]
	\item \textbf{conf} (in): Globálne nastavenia spracovania dát
\end{itemize}
\bigbreak
\hrule

\begin{lstlisting}[style=docs]
esp_err_t nvs_save_login (const Provisioning *login)
\end{lstlisting}
   Uloženie nastavení sieťového pripojenia na nevolatilné úložisko. \\ Používa sa v \verb|mqtt_event_handler()|. \\
\textbf{Parametre}:
\begin{itemize}[noitemsep, topsep=0pt, label=$\star$]
	\item \textbf{login} (in): Nastavenie sieťového pripojenia 
\end{itemize}
\bigbreak
\hrule

\begin{lstlisting}[style=docs]
void openlog_setup (OpenLog *logger)
\end{lstlisting}
Nastavenie UART rozhrania pre zariadenie OpenLog.
\bigbreak
\hrule

\begin{lstlisting}[style=docs]
void wifi_connect (wifi_config_t *wifi_config)
\end{lstlisting}
Pripojenie sa k Wifi AP blokujúce.


\subsection{Dátová pipeline} \label{modules:pipeline}

\subsubsection*{Dátové štruktúry}
\noindent\textbf{struct SamplingConfig} - Nastavenie vzorkovania signálu.
\begin{itemize}[noitemsep, topsep=0pt]
	\item \textbf{uint16\_t frequency}: Vzorkovacia frekvencia v Hz. Najviac \verb|MAX_FREQUENCY|
	\item \textbf{AccelerationRange range}: Dynamický rozsah akcelerometra
	\item \textbf{uint16\_t n}: Veľkosť posuvného okna. Musí byť mocninou dvojky a najviac \verb|MAX_BUFFER_SAMPLES|
	\item \textbf{float overlap}: Pomer prekryvu posuvných okien. Rozsah: 0 až \verb|MAX_OVERLAP|
	\item \textbf{bool axis[AXIS\_COUNT]}: Osi akcelerácie povolené na spracovanie
\end{itemize}
\bigbreak

\noindent\textbf{struct SmoothingConfig} - Nastavenie vyhladzovania časovo premenného signálu alebo frekvenčného spektra.
\begin{itemize}[noitemsep, topsep=0pt]
	\item \textbf{bool enable}: Vyhladzovanie signálu povolené
	\item \textbf{uint16\_t n}: Dĺžka konvolučnej masky
	\item \textbf{uint8\_t repeat}: Počet prechodov konvolučnej masky. Najviac \verb|MAX_SMOOTH_REPEAT|
\end{itemize}
\bigbreak

\noindent\textbf{struct StatisticsConfig} - Nastavenie zberu štatistík posuvného okna.
\begin{itemize}[noitemsep, topsep=0pt]
	\item \textbf{bool min}: Výpočet minimálnej hodnoty povolený
	\item \textbf{bool max}: Výpočet maximálnej hodnoty povolený
	\item \textbf{bool rms}: Výpočet strednej kvadratickej odchýlky povolený
	\item \textbf{bool mean}: Výpočet aritmetického priemeru povolený
	\item \textbf{bool variance}: Výpočet rozptylu povolený
	\item \textbf{bool std}: Výpočet smerodajnej odchýlky povolený
	\item \textbf{bool skewness}: Výpočet šikmosti povolený
	\item \textbf{bool kurtosis}: Výpočet špicatosti povolený
	\item \textbf{bool median}: Výpočet mediánu povolený
	\item \textbf{bool mad}: Výpočet mediánovej absolútnej odchýlky povolený
	\item \textbf{bool correlation}: Výpočet korelácie medzi osami povolený
\end{itemize}
\bigbreak

\noindent\textbf{struct FFTTransformConfig} - Nastavenie frekvenčnej transformácie.
\begin{itemize}[noitemsep, topsep=0pt]
	\item \textbf{WindowTypeConfig window}: Oknová funkcia
	\item \textbf{FrequencyTransform func}: Typ frekvenčnej transformácie
	\item \textbf{bool log}: Prevod magnitúdy frekvencie do dB
\end{itemize}
\bigbreak

\noindent\textbf{struct SaveFormatConfig} - Nastavenia ukladania a posielania spracovaných dát.
\begin{itemize}[noitemsep, topsep=0pt]
	\item \textbf{bool local}: Záznam vzoriek na SD kartu povolený. OpenLog bude zapnutý po spustení
	\item \textbf{bool mqtt}: Posielanie cez MQTT povolené. Wifi a MQTT klient bude zapnutý po spustení. Pozor: po deaktivácii
	sa zariadenie nedá vzdialene rekonfigurovať. Na znovu povolenie sa musí nahrať firmvér so touto možnosťou povolenou. 
	\item \textbf{bool mqtt\_stats}: Odosielanie štatistík cez MQTT na topic podľa \verb|MQTT_TOPIC_STATS|
	\item \textbf{bool mqtt\_events}:  Odosielanie zmien spektra cez MQTT na topic podľa \verb|MQTT_TOPIC_EVENT|
	\item \textbf{SendUnprocessed mqtt\_samples}:  Odosielanie nespracovaných vzoriek alebo frekvencií cez MQTT na topic podľa \verb|MQTT_TOPIC_STREAM|, \verb|MQTT_TOPIC_SPECTRUM|.
	\item \textbf{uint16\_t subsampling}: Podvzorkovanie pre záznam vzoriek bez ďalšieho spracovania. Preskočí sa každých \verb|subsampling| vzoriek
\end{itemize}
\bigbreak

\noindent\textbf{struct FFTTransformConfig} - Nastavenia algoritmov na detekciu udalostí a ich parametrov (popis s nachádza pri 
funkciách z modulu ,,Udalosti'' \ref{modules:events}.
\bigbreak

\noindent\textbf{struct Configuration} - Systémová konfigurácia pipeline spracovania vzoriek z akcelerometra.
\bigbreak

\noindent\textbf{struct Provisioning} - Sieťové nastavenia pre pripojenie na Wifi AP s WPA2 a MQTT broker.
\bigbreak

\noindent\textbf{struct Correlation} - Medzivýsledky korelácie zdieľanej všetkými osami spracovania s prístupom
cez zahrnutú synchronizačnú bariéru.
\bigbreak

\noindent\textbf{struct Statistics} - Výsledky všetkých dostupných štatistík.
\bigbreak

\noindent\textbf{struct BufferPipelineKernel} - Vyrovnávacie pamäte spoločné pre celú pipeline.
\bigbreak

\noindent\textbf{struct BufferPipelineAxis} - Vyrovnávacie pamäte samostatné pre každú os akcelerácie.
\bigbreak

\noindent\textbf{struct Sender} - Fronta pre záznam vzoriek bez ďalšiho spracovania.
\bigbreak

\subsubsection*{Konštanty}
V zátvorkách sú uvedené predvolené hodnoty
\begin{itemize}[noitemsep, topsep=0pt]
	\item \verb|AXIS_COUNT|: Celkový počet osí akcelerácie (3)
	\item \verb|MAX_MPACK_FIELDS_COUNT|: Maximálny počet dvojíc ,,kľúč - hodnota'' v Message Pack slovníku (20)
	\item \verb|SAMPLES_QUEUE_SLOTS|: Násobok veľkosti posuvného okna ako počet vzoriek čakajúcich na spracovanie vo fronte (3)
	\item \verb|MAX_CREDENTIALS_LENGTH|: Maximálna dĺžka prihlasovacieho údaju Wifi pripojenia (64)
	\item \verb|MAX_MQTT_URL|: Dĺžka URL adresy na MQTT broker (256)
	\item \verb|MAX_BUFFER_SAMPLES|: Najdlhšie povolené posuvné okno, vyššia mocnina 2 ako 1024 sa nezmestí do DRAM (1024)
	\item \verb|MAX_FREQUENCY|: Najvyššia vzorkovacia frekvencia daná fyzickým obmedzením akcelerometra (952)
	\item \verb|MAX_OVERLAP|:  Maximálny prekryv posuvných okien (0.8)
	\item \verb|MAX_SMOOTH_REPEAT|:  Maximálny počet prechodu konvolučnej masky vyhladzovacieho filtra (8)
	\item \verb|LARGEST_MESSAGE|: Najväčšia veľkosť vyrovnávacej pamäte pre serializáciu vzoriek (14000)
	\item \verb|LARGEST_CONFIG|: Najväčšia veľkosť serializovanej konfigurácie (480)
\end{itemize}


\subsubsection*{Enumerácie}

\noindent\textbf{enum WindowTypeConfig} - Oknové funkcie.
	\begin{itemize}[noitemsep, topsep=0pt, label=$\star$]
		\item \verb|BOXCAR_WINDOW|: Obdĺžníkové okno
		\item \verb|BARTLETT_WINDOW|: Bartlettovo okno
		\item \verb|HANN_WINDOW|: Hannovo okno
		\item \verb|HAMMING_WINDOW|: Hammingovo okno
		\item \verb|BLACKMAN_WINDOW|: Blackmanovo okno
		\item \verb|WINDOW_TYPE_COUNT|: Počet dostupných oknových funkcií. Potrebné pre serializáciu.
	\end{itemize}
\bigbreak

\noindent\textbf{enum PeakFindingStrategy} - Algoritmy na hľadanie špičiek.
	\begin{itemize}[noitemsep, topsep=0pt, label=$\star$]
		\item \verb|THRESHOLD|: Špičky nad prahovou úrovňou.
		\item \verb|NEIGHBOURS|: Špičky najvýznačnejšieho bodu spomedzi susedov.
		\item \verb|ZERO_CROSSING|: Špičky prechodou nulou do záporu.
		\item \verb|HILL_WALKER|: Špičky algoritmom horského turistu. 
		\item \verb|STRATEGY_COUNT|: Počet možností na účely serializácie
	\end{itemize}
\bigbreak

\noindent\textbf{enum FrequencyTransform} - Frekvenčné transformácie.
	\begin{itemize}[noitemsep, topsep=0pt, label=$\star$]
		\item \verb|DFT|: Rýchla Fourierová transformácia radix-2
		\item \verb|DCT|: Konsínusová transformácia DCT II
		\item \verb|TRANSFORM_COUNT|: Počet dostupných frekvenčných transformácií. Potrebné pre serializáciu
	\end{itemize}
\bigbreak

\noindent\textbf{enum SendUnprocessed} - Doména odosielaných nespracovaných vzoriek.
	\begin{itemize}[noitemsep, topsep=0pt, label=$\star$]
		\item \verb|RAW_NONE_SEND|: Žiadne nespracované vzorky 	
		\item \verb|RAW_TIME_SEND|: Nespracované vzorky v časovej oblasti 	
		\item \verb|RAW_FREQUENCY_SEND|: Nespracované vzorky vo frekvenčnej oblasti 
		\item \verb|SEND_UNPROCESSED_COUNT|: Počet možností na účely serializácie
	\end{itemize}
\bigbreak


\subsubsection*{Oknové funkcie}
\textbf{Parametre všetkých oknových funkcií}:
\begin{itemize}[noitemsep, topsep=0pt, label=$\star$]
	\item w (out): Váhy oknovej funkcie
	\item n (in): Dĺžka okna  
\end{itemize}

\hrule
\begin{lstlisting}[style=docs]
void bartlett_window (float *w, int n)
\end{lstlisting}
   Bartlettovo okno. $w(n) = \frac{2}{N - 1}\left(\frac{N - 1}{2} - \left|n - \frac{N - 1}{2} \right|\right)$
\bigbreak
\hrule

\begin{lstlisting}[style=docs]
void blackman_window (float *w, int n)
\end{lstlisting}
   Blackmanovo okno. $w(n) = 0.42 - 0.5\cos(2\pi n / N) + 0.08\cos(4\pi n / N)$
\bigbreak
\hrule

\begin{lstlisting}[style=docs]
void boxcar_window (float *w, int n)
\end{lstlisting}
   Obdĺžníkové okno.  $w(n) = 1 $
\bigbreak
\hrule

\begin{lstlisting}[style=docs]
void hamming_window (float *w, int n)
\end{lstlisting}
   Hammingovo okno.  $w(n) = 0.54 - 0.46\cos(2\pi n / N)$
\bigbreak
\hrule

\begin{lstlisting}[style=docs]
void hann_window(float *w, int n)
\end{lstlisting}
   Hannovo okno.  $w(n) = \sin^2(\pi n / N)$
\bigbreak
\hrule

\begin{lstlisting}[style=docs]
void mean_kernel (float *w, int n)
\end{lstlisting}
   Vyhladzovací filter kĺzavého priemeru.  $w(n) = \frac{1}{n},\, n = 0, 1, ..., N - 1$
\bigbreak
\hrule

\begin{lstlisting}[style=docs]
void window (WindowTypeConfig type, float *w, int n)
\end{lstlisting}
   Oknová funkcia podľa voľby. \\ 
\textbf{Parametre}
\begin{itemize}[noitemsep, topsep=0pt, label=$\star$]
	\item type (out): Oknová funkcia
	\item w (out): Váhy oknovej funkcie
	\item n (in): Dĺžka okna  
\end{itemize}
\bigbreak
\hrule

\subsubsection*{Správa pamäti pipeline}

\begin{lstlisting}[style=docs]
void axis_allocate (BufferPipelineAxis *p, const Configuration *conf)
\end{lstlisting}
   Alokácia dynamických vyrovnávacích pamätí pre jednu os akcelerácie. \\ 
\textbf{Parametre}:
\begin{itemize}[noitemsep, topsep=0pt, label=$\star$]
	\item \textbf{p} (out): Dynamické vyrovnávacie pamäte s dĺžkami podľa nastavení
	\item \textbf{conf} (in): Konfigurácia systému  
\end{itemize}
\bigbreak
\hrule

\begin{lstlisting}[style=docs]
void axis_release (BufferPipelineAxis *p)
\end{lstlisting}
   Uvoľnenie pamäte pre dynamické vyrovnávacie pamäte pre jednu os akcelerácie. \\ 
\textbf{Parametre}:
\begin{itemize}[noitemsep, topsep=0pt, label=$\star$]
	\item \textbf{p} (out): Dynamické vyrovnávacie pamäte
\end{itemize}
\bigbreak
\hrule

\begin{lstlisting}[style=docs]
void process_allocate (
	BufferPipelineKernel *p, const Configuration *conf
)
\end{lstlisting}
   Alokácia a inicializácia dynamických vyrovnávacích pamätí a synchronizačných primitív. \\ 
\textbf{Parametre}:
\begin{itemize}[noitemsep, topsep=0pt, label=$\star$]
	\item \textbf{p} (out): Dynamické vyrovnávacie pamäte s dĺžkami podľa nastavení
	\item \textbf{conf} (in): Konfigurácia systému  
\end{itemize}
\bigbreak
\hrule

\begin{lstlisting}[style=docs]
void process_release (BufferPipelineKernel *p)
\end{lstlisting}
   Uvoľnenie pamäte pre dynamické vyrovnávacie pamäte. \\ 
\textbf{Parametre}:
\begin{itemize}[noitemsep, topsep=0pt, label=$\star$]
	\item \textbf{p} (out): Dynamické vyrovnávacie pamäte
\end{itemize}
\bigbreak
\hrule

\begin{lstlisting}[style=docs]
void axis_release (BufferPipelineAxis *p)
\end{lstlisting}
   Uvoľnenie pamäte pre dynamické vyrovnávacie pamäte pre jednu os akcelerácie. \\ 
\textbf{Parametre}:
\begin{itemize}[noitemsep, topsep=0pt, label=$\star$]
	\item \textbf{p} (out): Dynamické vyrovnávacie pamäte
\end{itemize}
\bigbreak
\hrule

\begin{lstlisting}[style=docs]
void sender_release (Sender *sender)
\end{lstlisting}
   Odstránenie fronty na odosielanie nameraných vzoriek. \\ 
\textbf{Parametre}
\begin{itemize}[noitemsep, topsep=0pt, label=$\star$]
	\item \textbf{sender}: Fronta s vyhradenou kapacitou
\end{itemize}
\bigbreak
\hrule


\subsubsection*{Fázy spracovania oknovaného signálu}

\begin{lstlisting}[style=docs]
void buffer_shift_left(
	float *buffer, uint16_t n, uint16_t k
)
\end{lstlisting}
Posun vzoriek vo vyrovnávacej pamäti doľava, čím sa dosahuje prekryv okien. 
Nadbytočné hodnoty od začiatku poľa budú nahradené vzorkami o \verb|k| pozícii vpravo. \\ 
\textbf{Parametre}:
\begin{itemize}[noitemsep, topsep=0pt, label=$\star$]
	\item \textbf{buffer}: Vyrovnávacia pamäť, ktorej obsah bude posunutý
 	\item \textbf{n} (in): Dĺžka vyrovnávacej pamäte
	\item \textbf{k} (in): Počet pozícii o koľko sa majú posunúť hodnoty.
\end{itemize}
\bigbreak
\hrule

\begin{lstlisting}[style=docs]
void process_correlation(
	uint8_t axis, const float *buffer, 
	Statistics *stats, Correlation *corr, 
	const SamplingConfig *c
)
\end{lstlisting}
Korelácia medzi osami akcelerácie: XY, XZ, YZ. Dochádza k bariérovej synchronizácii. 
Úloha pre každú os zrýchlenia si nezávisle prepočíta rozdiely vzoriek od priemeru a smerodajné odchýlky.
Následne dochádza k bariérovej synchronizácii aktívnych osí. Každá úloha si dopočíta všetky korelácie
samostatne.  \\ 
\textbf{Parametre}:
\begin{itemize}[noitemsep, topsep=0pt, label=$\star$]
	\item \textbf{axis} (in): Os akcelerácie: 0, 1, 2
 	\item \textbf{buffer} (in): Posuvné okno vzoriek signálu          
 	\item \textbf{stats} (out): Zistené medzi-osové korelácie
 	\item \textbf{corr} (out): Pomocné polia pre výmenu predspracovaných údajov medzi úlohami (osami)
 	\item \textbf{corr} (in): Nastavenia vzorkovania. Využíva sa dĺžka posuvného okna a povolené osi.
\end{itemize}
\bigbreak
\hrule

\begin{lstlisting}[style=docs]
void process_smoothing(
	float *buffer, float *tmp, uint16_t n, 
	const float *kernel, const SmoothingConfig *c
)
\end{lstlisting}
Vyhladzovanie signálu. \\ 
\textbf{Parametre}:
\begin{itemize}[noitemsep, topsep=0pt, label=$\star$]
	\item \textbf{buffer}: Posuvné okno vzoriek signálu s dĺžkou \verb|n|, ktoré bude vyhladené
	\item \textbf{tmp}: Pomocné pole o dĺžke \verb|n + c.n - 1|
 	\item \textbf{window} (in): Váhy oknovej funkcie  s dĺžkou \verb|n|
 	\item \textbf{n} (in): Dĺžka posuvného okna
 	\item \textbf{kernel} (in): Konvolučná maska vyhladzovania
 	\item \textbf{c} (in): Nastavenia vyhladzovanie
\end{itemize}
\bigbreak
\hrule

\begin{lstlisting}[style=docs]
int process_spectrum(
	float *spectrum, const float *buffer, 
	const float *window, uint16_t n, 
	const FFTTransformConfig *c
)
\end{lstlisting}
Frekvenčné spektrum (FFT, FCT) posuvného okna vzoriek vynásobené váhami oknovej funkcie. \\ 
\textbf{Parametre}:
\begin{itemize}[noitemsep, topsep=0pt, label=$\star$]
	\item \textbf{spectrum} (out): Frekvenčné spektrum s dĺžkou $n / 2$  
	\item \textbf{buffer} (in): Posuvné okno vzoriek signálu s dĺžkou $n$
	\item \textbf{window} (in): Váhy oknovej funkcie  s dĺžkou $n$
	\item \textbf{n} (in): Dĺžka posuvného okna
	\item \textbf{c} (in): Nastavenia frekvenčnej transformácie
\end{itemize}
\bigbreak
\hrule

\begin{lstlisting}[style=docs]
void process_statistics(
	const float *buffer, uint16_t n, 
	Statistics *stats, const StatisticsConfig *c
)
\end{lstlisting}
Požadované štatistiky podľa nastavení. \\ 
\textbf{Parametre}:
\begin{itemize}[noitemsep, topsep=0pt, label=$\star$]
	\item \textbf{buffer} (in): Posuvné okno vzoriek signálu
 	\item \textbf{n} (in): Dĺžka posuvného okna
 	\item \textbf{stats} (out): Deskriptívne štatistiky zo vzoriek posuvného okna. Korektné hodnoty majú len tie povolené v nastaveniach `c`
	\item \textbf{c} (in): Povolenia pre zber vybraných štatistík
\end{itemize}
\bigbreak
\hrule

\begin{lstlisting}[style=docs]
void process_peak_finding(
	bool *peaks, const float *spectrum, 
	uint16_t bins, const EventDetectionConfig *c
)
\end{lstlisting}
Hľadanie špičiek vo frekvenčnom spektre podľa nastavení aktívneho algoritmu. \\ 
\textbf{Parametre}:
\begin{itemize}[noitemsep, topsep=0pt, label=$\star$]
	\item \textbf{peaks} (out): Váhy oknovej funkcie  s dĺžkou \verb|bins|
 	\item \textbf{spectrum} (in):  Frekvenčné spektrum s dĺžkou \verb|bins|  
 	\item \textbf{bins} (in): Počet frekvenčných vedierok
 	\item \textbf{c} (in): Nastavenia spracovania udalostí
\end{itemize}
\bigbreak
\hrule


\subsubsection*{Message Pack serializácia}

\begin{lstlisting}[style=docs]
size_t stream_serialize(
	char *msg, size_t size, 
	const float *stream, size_t n
)
\end{lstlisting}
Serializácia prúdu vzoriek v posuvnom okne do formátu Message Pack. \\ 
\textbf{Parametre}:
\begin{itemize}[noitemsep, topsep=0pt, label=$\star$]
	\item \textbf{msg} (out): Serializované vzorky signálu
	\item \textbf{size} (in): Vyhradená veľkosť pre správu do \verb|msg|
	\item \textbf{stream} (in): Vzorky signálu
	\item \textbf{n} (in): Počet vzoriek signálu
 	\item \textbf{Návratová hodnota}: Dĺžka serializovanej správy
\end{itemize}
\bigbreak
\hrule

\begin{lstlisting}[style=docs]
size_t stats_serialize(
	size_t timestamp, char *msg, size_t size, 
	const Statistics *stats, const StatisticsConfig *c
)
\end{lstlisting}
Serializácia štatistík signálu v posuvnom okne do formátu Message Pack. \\ 
\textbf{Parametre}:
\begin{itemize}[noitemsep, topsep=0pt, label=$\star$]
	\item \textbf{timestamp} (in): Poradové číslo posuvného okna
 	\item \textbf{msg} (out): Serializované štatistiky
 	\item \textbf{size} (in): Vyhradená veľkosť pre správu do \verb|msg|
 	\item \textbf{stats} (in): Štatistiky signálu
 	\item \textbf{n} (in): Nastavenia zberu štatistík. Do serializovanej správy sa zahrnú len aktívne štatistiky.
 	\item \textbf{Návratová hodnota}: Dĺžka serializovanej správy
\end{itemize}
\bigbreak
\hrule

\begin{lstlisting}[style=docs]
size_t spectra_serialize(
	size_t timestamp, char *msg, size_t size, 
	const float *spectrum, size_t n, uint16_t fs
)
\end{lstlisting}
Serializácia frekvenčného spektra posuvného okna do formátu Message Pack. \\ 
\textbf{Parametre}
\begin{itemize}[noitemsep, topsep=0pt, label=$\star$]
 	\item \textbf{timestamp} (in): Poradové číslo posuvného okna
 	\item \textbf{msg} (out): Serializované frekvenčné spektrum
 	\item \textbf{size} (in): Vyhradená veľkosť pre správu do \verb|msg|
 	\item \textbf{spectrum} (in): Frekvenčné spektrum
	\item \textbf{n} (in): Počet frekvenčných vedierok
	\item \textbf{fs} (in): Vzorkovacia frekvencia v Hz
 	\item \textbf{Návratová hodnota}: Dĺžka serializovanej správy
\end{itemize}
\bigbreak
\hrule

\begin{lstlisting}[style=docs]
size_t events_serialize(
	size_t timestamp, float bin_width, 
	char *msg, size_t size, 
	const SpectrumEvent *events, size_t n
)
\end{lstlisting}
Serializácia udalostí zmien frekvenčného spektra do formátu Message Pack. \\ 
\textbf{Parametre}:
\begin{itemize}[noitemsep, topsep=0pt, label=$\star$]
	\item \textbf{timestamp} (in): Poradové číslo posuvného okna
	\item \textbf{bin\_width} (in): Veľkosť frekvenčného vedierka v Hz: $fs / n$
	\item \textbf{msg} (in): Serializované udalosti
	\item \textbf{size} (in): Vyhradená veľkosť pre správu do  \verb|msg|
	\item \textbf{events} (in): Udalosti frekvenčného spektra s dĺžkou  \verb|n|. 
	Do správy budú pridané iba začiatočné a ukončujúce udalosti.
	\item \textbf{n} (in): Počet frekvenčných vedierok
	\item \textbf{Návratová hodnota}: Dĺžka serializovanej správy
\end{itemize}
\bigbreak
\hrule

\begin{lstlisting}[style=docs]
size_t config_serialize(
	char *msg, size_t size, 
	const Configuration *config
)
\end{lstlisting}
Serializácia systémovej konfigurácie do formátu Message Pack. \\ 
\textbf{Parametre}:
\begin{itemize}[noitemsep, topsep=0pt, label=$\star$]
	\item \textbf{msg} (out): Serializovaná konfigurácia
 	\item \textbf{size} (in): Vyhradená veľkosť pre správu do \verb|msg|
	\item \textbf{config} (in): Systémová konfigurácia
	\item \textbf{Návratová hodnota}: Dĺžka serializovanej správy
\end{itemize}
\bigbreak
\hrule

\begin{lstlisting}[style=docs]
bool config_parse(
	const char *msg, int size, 
	Configuration *conf, bool *error
)
\end{lstlisting}
Parsovanie systémovej konfigurácie z formátu Message Pack. \\ 
\textbf{Parametre}:
\begin{itemize}[noitemsep, topsep=0pt, label=$\star$]
	\item \textbf{msg} (in): Serializovaná konfigurácia
	\item \textbf{size} (in): Dĺžka konfigurácie v Message Pack
 	\item \textbf{conf} (out): Systémová konfigurácia
	\item \textbf{error} (out): Chyba pri parsovaní
	\item \textbf{Návratová hodnota}: Zmena konfigurácie oproti pôvodnému obsahu \verb|conf|
\end{itemize}
\bigbreak
\hrule

\begin{lstlisting}[style=docs]
size_t login_serialize(
	char *msg, size_t size,
	const Provisioning *conf
)
\end{lstlisting}
Serializácia údajov o sieťovom pripojení. \\ 
\textbf{Parametre}:
\begin{itemize}[noitemsep, topsep=0pt, label=$\star$]
	\item \textbf{msg} (out): Serializované nastavenia pripojenia
	\item \textbf{size} (in): Vyhradená veľkosť pre správu do \verb|msg|
	\item \textbf{config} (in): Nastavenia pripojenia
	\item \textbf{Návratová hodnota}: Dĺžka serializovanej správy
\end{itemize}
\bigbreak
\hrule

\begin{lstlisting}[style=docs]
bool login_parse(
	const char *msg, size_t size, 
	Provisioning *conf
)
\end{lstlisting}
Parsovanie nastavení sieťového pripojenia z formátu Message Pack. \\ 
\textbf{Parametre}:
\begin{itemize}[noitemsep, topsep=0pt, label=$\star$]
	\item \textbf{msg} (in): Serializované konfigurácia
 	\item \textbf{size} (in): Vyhradená veľkosť pre správu do \verb|msg|
	\item \textbf{conf} (out): Nastavenia pripojenia
	\item \textbf{Návratová hodnota}: Chyba pri parsovaní
\end{itemize}
\bigbreak
\hrule


\subsection{Deskriptívna štatistika} \label{modules:statistics}

\subsubsection*{Funkcie}
\hrule
\begin{lstlisting}[style=docs]
float minimum(const float *x, int n)
\end{lstlisting}
Najnižšia hodnota. \\ 
\textbf{Parametre}:
\begin{itemize}[noitemsep, topsep=0pt, label=$\star$]
	\item \textbf{x} (in): Vzorky signálu
	\item \textbf{n} (in):  Počet vzoriek signálu
	\item \textbf{Návratová hodnota}: Mimimum z hodnôt signálu
\end{itemize}
\bigbreak
\hrule

\begin{lstlisting}[style=docs]
float maximum(const float *x, int n)
\end{lstlisting}
Najvyššia hodnota. \\ 
\textbf{Parametre}:
\begin{itemize}[noitemsep, topsep=0pt, label=$\star$]
	\item \textbf{x} (in): Vzorky signálu
	\item \textbf{n} (in):  Počet vzoriek signálu
	\item \textbf{Návratová hodnota}: Maximum z hodnôt signálu
\end{itemize}
\bigbreak
\hrule

\begin{lstlisting}[style=docs]
float root_mean_square(const float *x, int n)
\end{lstlisting}
Stredná kvadratická odchýlka. \\ 
\textbf{Parametre}:
\begin{itemize}[noitemsep, topsep=0pt, label=$\star$]
	\item \textbf{x} (in): Vzorky signálu
	\item \textbf{n} (in):  Počet vzoriek signálu
	\item \textbf{Návratová hodnota}:  RMS z hodnôt signálu
\end{itemize}
\bigbreak
\hrule

\begin{lstlisting}[style=docs]
float mean(const float *x, int n)
\end{lstlisting}
Aritmetický výberový priemer. \\ 
\textbf{Parametre}:
\begin{itemize}[noitemsep, topsep=0pt, label=$\star$]
	\item \textbf{x} (in): Vzorky signálu
	\item \textbf{n} (in):  Počet vzoriek signálu
	\item \textbf{Návratová hodnota}:  Priemer hodnôt signálu
\end{itemize}
\bigbreak
\hrule

\begin{lstlisting}[style=docs]
float variance(const float *x, int n, float mean)
\end{lstlisting}
Rozptyl populácie (vychýlená štatistika). \\ 
\textbf{Parametre}:
\begin{itemize}[noitemsep, topsep=0pt, label=$\star$]
	\item \textbf{x} (in): Vzorky signálu
	\item \textbf{n} (in):  Počet vzoriek signálu
	\item \textbf{Návratová hodnota}:  Rozptyl hodnôt signálu
\end{itemize}
\bigbreak
\hrule

\begin{lstlisting}[style=docs]
float standard_deviation(float variance)
\end{lstlisting}
Smerodajná odchýlka. \\ 
\textbf{Parametre}:
\begin{itemize}[noitemsep, topsep=0pt, label=$\star$]
	\item \textbf{variiance} (in): Rozptyl signálu
	\item \textbf{Návratová hodnota}:  Smerodajná odchýlka hodnôt signálu
\end{itemize}
\bigbreak
\hrule

\begin{lstlisting}[style=docs]
float moment(const float *x, int n, int m, float mean)
\end{lstlisting}
Centrálny moment rádu \verb|m|. \\ 
\textbf{Parametre}:
\begin{itemize}[noitemsep, topsep=0pt, label=$\star$]
	\item \textbf{x} (in): Vzorky signálu
	\item \textbf{n} (in): Počet vzoriek signálu
	\item \textbf{m} (in): Rád centrálneho momentu. Kladné číslo väčšie ako 1.
	\item \textbf{mean} (in): Aritmetický priemer signálu
	\item \textbf{Návratová hodnota}:  Centrálny moment
\end{itemize}
\bigbreak
\hrule

\begin{lstlisting}[style=docs]
float skewness(const float *x, int n, float mean)
\end{lstlisting}
Šikmosť. \\ 
\textbf{Parametre}:
\begin{itemize}[noitemsep, topsep=0pt, label=$\star$]
	\item \textbf{x} (in): Vzorky signálu
	\item \textbf{n} (in): Počet vzoriek signálu
	\item \textbf{mean} (in): Aritmetický priemer signálu
	\item \textbf{Návratová hodnota}:  Šikmosť
\end{itemize}
\bigbreak
\hrule

\begin{lstlisting}[style=docs]
float kurtosis(const float *x, int n, float mean)
\end{lstlisting}
Špicatosť. \\ 
\textbf{Parametre}:
\begin{itemize}[noitemsep, topsep=0pt, label=$\star$]
	\item \textbf{x} (in): Vzorky signálu
	\item \textbf{n} (in): Počet vzoriek signálu
	\item \textbf{mean} (in): Aritmetický priemer signálu
	\item \textbf{Návratová hodnota}:  Špicatosť
\end{itemize}
\bigbreak
\hrule

\begin{lstlisting}[style=docs]
float correlation(
	const float *x_diff, const float *y_diff, int n, 
	float x_std, float y_std
)
\end{lstlisting}
Korelácia z medzivýsledkov. \\ 
\textbf{Parametre}:
\begin{itemize}[noitemsep, topsep=0pt, label=$\star$]
	\item \textbf{x\_diff} (in):  Predspracované vzorky prvého signálu odčítané od aritmetického priemeru: $(x_i - \bar{x})$
	\item \textbf{y\_diff} (in):  Predspracované vzorky druhého signálu odčítané od aritmetického priemeru: $(y_i - \bar{y})$
	\item \textbf{n} (in): Počet vzoriek signálu. Dĺžky oboch polí musia byť rovnaké.
	\item \textbf{x\_std} (in): Smerodajná odchýlka prvého signálu
	\item \textbf{y\_std} (in): Smerodajná odchýlka druhého signálu
	\item \textbf{Návratová hodnota}:  Pearsonov korelačný koeficient
\end{itemize}
\bigbreak
\hrule

\begin{lstlisting}[style=docs]
float quickselect(const float *x, int n, int k)
\end{lstlisting}
Quickselect. Algoritmus na nájdenie k-teho najmenšieho prvku v nezoradenom poli. Aby nedochádzalo
k modifikácii poradia pôvodného poľa kopíruje prvky do poľa z premenlivou dĺžkou (Variable-length array) 
 podľa \verb|n|, na zásobníku. \\ 
\textbf{Parametre}:
\begin{itemize}[noitemsep, topsep=0pt, label=$\star$]
	\item \textbf{x}: Vzorky signálu
	\item \textbf{n} (in): Počet vzoriek signálu
	\item \textbf{k} (in): Rád k-teho najmenšieho prvku
	\item \textbf{Návratová hodnota}: k-ty najmenší prvok
\end{itemize}
\bigbreak
\hrule

\begin{lstlisting}[style=docs]
float median(const float *x, int n)
\end{lstlisting}
Medián cez Quickselect. \\ 
\textbf{Parametre}:
\begin{itemize}[noitemsep, topsep=0pt, label=$\star$]
	\item \textbf{x} (in): Vzorky signálu
	\item \textbf{n} (in): Počet vzoriek signálu
	\item \textbf{Návratová hodnota}: Medián
\end{itemize}
\bigbreak
\hrule

\begin{lstlisting}[style=docs]
float median_abs_deviation(
	const float *x, int n, float med
)
\end{lstlisting}
Mediánová absolútna odchýlka (MAD).  Medzi-výsledky odchýlok na nájdenie mediánu ukladá
do poľa z premenlivou dĺžkou (Variable-length array) podľa \verb|n|, na zásobníku \\
\textbf{Parametre}:
\begin{itemize}[noitemsep, topsep=0pt, label=$\star$]
	\item \textbf{x} (in): Vzorky signálu
	\item \textbf{n} (in): Počet vzoriek signálu
	\item \textbf{med} (in):  Medián signálu
	\item \textbf{Návratová hodnota}: MAD
\end{itemize}
\bigbreak
\hrule

\begin{lstlisting}[style=docs]
float average_abs_deviation(
	const float *x, int n, float mean
)
\end{lstlisting}
Priemerná absolútna odchýlka (AAD). \\ 
\textbf{Parametre}:
\begin{itemize}[noitemsep, topsep=0pt, label=$\star$]
	\item \textbf{x} (in): Vzorky signálu
	\item \textbf{n} (in): Počet vzoriek signálu
	\item \textbf{mean} (in):  Aritmetický priemer signálu
	\item \textbf{Návratová hodnota}: AAD
\end{itemize}
\bigbreak
\hrule


\section{Prehľad MQTT topics} 
imu/[id]/
"imu started"
	"config received"
	"config malformed"
	"config applied"
	"login malformed"
	"login saved"

\begin{itemize}[noitemsep,topsep=0pt]
	\item \textbf{syslog}
	\item \textbf{config/request}
	\item \textbf{config/response}
	\item \textbf{config/set}
	\item \textbf{login/request}
	\item \textbf{login/response}
	\item \textbf{login/set}
	\item \textbf{samples}
	\item \textbf{stats/x}
	\item \textbf{spectrum/x}
	\item \textbf{events/x}
\end{itemize}
	
\section{Štruktúra Message Pack správ}
- samples
\begin{verbatim}
[-0.07836493849754333, -0.9104690551757812, -9.964909553527832, -1.7736798524856567, -16.43988800048828, -0.4007977843284607, 1.4985052347183228, 5.201996326446533, 1.499103307723999]
\end{verbatim}

- stats/x, stats/y, stats/z
\begin{verbatim}
{"t": 386, "min": -9.964909553527832, "max": 11.8468656539917, "rms": 8.343297004699707, "avg": 4.126497745513916, "std": 7.251387119293213, "skew": -0.6144028306007385, "kurt": -0.8219752311706543, "med": 5.772983551025391, "mad": 5.370391845703125}
\end{verbatim}

- spectrum/x spectrum/y  spectrum/z
\begin{verbatim}
{"t": 25, "fs": 8, "bins": [0.0, -0.026730040088295937, -38.7723274230957, -38.19501495361328]}
\end{verbatim}

- events/x events/y events/z
\begin{verbatim}
{
"t": 386, "df": 2.0,
"A": [{"i": 2, "t": 382, "d": 5, "h": -5.620919704437256}, {"i": 3, "t": 382, "d": 5, "h": -3.549427032470703}],
"Z": []
}
\end{verbatim}

\begin{verbatim}
{"sensor": {"fs": 8, "range": "2g", "n": 8, "overlap": 0.5, "axis": [false, false, true]},
"tsmooth": {"on": false, "n": 8, "repeat": 1},
"stats": {"min": true, "max": true, "rms": true, "avg": true, "var": true, "std": true, "skew": true, "kurtosis": true, "med": true, "mad": true, "corr": false},
"transform": {"w": "hann", "f": "dft", "log": true},
"fsmooth": {"on": false, "n": 8, "repeat": 1},
"peak": {"tmin": 4, "tprox": 5, "strategy": "threshold", "threshold": {"t": -15.0},
"neighbours": {"k": 9, "e": 0.0, "h": -100.0, "h_rel": 10.0},
"zero_crossing": {"k": 4, "slope": 3.0},
"hill_walker": {"t": 0.0, "h": 0, "p": 10.0, "i": 3.0}},
"logger": {"local": false, "mqtt": true, "samples": "t", "stats": true, "events": true, "subsamp": 1}}
\end{verbatim}

\section{Datasety z premávky}
Datasety z autobusov a električiek boli zaznamenané v dátumoch 1.11. a 3.11.2021 so vzorkovacou frekvenciou 500 Hz 
a rozlíšením $\pm 2$ g so zariadením opísaním v hlavnej časti. Vozidlá boli súčasťou bežnej výpravy metskej 
hromadnej dopravy prepravcu Dopravný podnik Bratislava, a.s. Mierne diskrepancie na úrovni vzoriek a nezmyselné 
presahy v nahrávaní boli odstránené. Asfaltové povrchy ciest boli pomerne nové a suché.

\paragraph{L3\_StnVinohrady\_Riazanska.csv}
\begin{itemize}[noitemsep, topsep=0pt]
  	\item \textbf{Linka:} 3
  	\item \textbf{Trvanie:} 120033 vzoriek (240,07 s)
  	\item \textbf{Zastávky:} Stn. Vinohrady (v pokoji pred semafórom), Nám. Biely Kríž, Mladá Garda, Riazanská
  	\item \textbf{Vozidlo:} električka Škoda 30 T
	\item \textbf{Umiestnenie:} pravá časť nápravy, vyvýšené sedenie v štvorke
\end{itemize}

\paragraph{L3\_Pionierska\_RacianskeMyto.csv}
\begin{itemize}[noitemsep, topsep=0pt]
  	\item \textbf{Linka:} 3
  	\item \textbf{Trvanie:} 70437 vzoriek (140,87 s)
  	\item \textbf{Zastávky:} Pionierska, Ursínyho, Račianske mýto
  	\item \textbf{Vozidlo:} električka Škoda 30 T
  	\item \textbf{Umiestnenie:} pravá časť nápravy, vyvýšené sedenie v štvorke
\end{itemize}

\paragraph{L9\_Postova\_KralovskeUdolie.csv}
\begin{itemize}[noitemsep, topsep=0pt]
  	\item \textbf{Linka:} 9
  	\item \textbf{Trvanie:} 149150 vzoriek (298,3 s)
  	\item \textbf{Zastávky:} Poštová, Kapucínska, (Tunel), Kráľovské údolie
  	\item \textbf{Vozidlo:} električka Škoda 29 T
  	\item \textbf{Umiestnenie:} ľavá časť v zníženom sedení medzi harmonikou a vyvýšenou zadnou plošinou
\end{itemize}
  		
\paragraph{L9\_Lanfranconi\_Riviera.csv}
\begin{itemize}[noitemsep, topsep=0pt]
  	\item \textbf{Linka:} 9
  	\item \textbf{Trvanie:} 79010 vzoriek (158,02 s)
  	\item \textbf{Zastávky:} Lanfranconi, Botanická záhrada, Riviéra
  	\item \textbf{Vozidlo:} električka Škoda 29 T
  	\item \textbf{Umiestnenie:} ľavá časť v zníženom sedení medzi harmonikou a vyvýšenou zadnou plošinou
\end{itemize}
  	
\paragraph{L35\_1915\_Most\_Kutiky\_neutral.csv}
	\begin{itemize}[noitemsep, topsep=0pt]
  	\item \textbf{Linka:} 35
  	\item \textbf{Trvanie:} 8315 vzoriek (16,63 s)
  	\item \textbf{Zastávky:} žiadne - zastavený na moste Kútiky kvôli prekážke na ceste
  	\item \textbf{Vozidlo:} midibus Solaris Urbino 8,6 \#1915
  	\item \textbf{Umiestnenie:} pravá zadná náprava, predposledné zadné sedenie proti smeru jazdy
  	\end{itemize}
\paragraph{L35\_1915\_Borska\_Zaluhy.csv}
	\begin{itemize}[noitemsep, topsep=0pt]
  	\item \textbf{Linka:} 35
  	\item \textbf{Trvanie:} 109502 vzoriek (219 s)
  	\item \textbf{Zastávky:} koniec Púpavovej ulice (jazda), Borská (zastávka), Záluhy (jazda, hneď za pravotočivou zákrutou križovatky na smer Lamač)
  	\item \textbf{Vozidlo:} midibus Solaris Urbino 8,6 \#1915
  	\item \textbf{Umiestnenie:} pravá zadná náprava, predposledné zadné sedenie proti smeru jazdy
  	\end{itemize}
 
\paragraph{L20\_3014\_Zaluhy\_Drobneho.csv}
	\begin{itemize}[noitemsep, topsep=0pt]
  	\item \textbf{Linka:} 20
  	\item \textbf{Trvanie:} 113001 vzoriek (226 s)
  	\item \textbf{Zastávky:} Záluhy (jazda, pred križovatkou smer Dúbravka od Lamača), Záluhy (zastávka), Švantnerova, Alexyho, Drobného, Podvornice (na polceste, jazda, križovatka)
  	\item \textbf{Vozidlo:} elektrobus SOR NS 12 Electric \#3014
  	\item \textbf{Umiestnenie:} ľavá zadná náprava, predposledné zadné sedenie, pod pravým sedadlom v dvojke
  	\end{itemize}
  
\paragraph{L83\_4940\_PriKrizi\_Alexyho.csv}
	\begin{itemize}[noitemsep, topsep=0pt]
  	\item \textbf{Linka:} 83
  	\item \textbf{Trvanie:} 208089 vzoriek (416,18 s)
  	\item \textbf{Zastávky:} Pri Kríži (tesne po rozbehnutí), Homolova, Štepná, Žatevná, Pekníková, Drobného, Alexyho (križovatka, zastavenie na semafóre)
  	\item \textbf{Vozidlo:} kĺbový autobus Mercedes-Benz O 530 GL CapaCity \#4940
  	\item \textbf{Umiestnenie:} nad motorom vpravo vzadu, posledné zadné priečne sedadlo pred plošinou na batožinu
  	\end{itemize}
  
\paragraph{L83\_4940\_Alexyho\_Svantnerova.csv}
	\begin{itemize}[noitemsep, topsep=0pt]
  	\item \textbf{Linka:} 83
  	\item \textbf{Trvanie:} 44182 vzoriek (88,36 s)
  	\item \textbf{Zastávky:} Alexyho (zastávka), Švantnerova (zastávka, na zvažujúcom kopci)
  	\item \textbf{Vozidlo:} kĺbový autobus Mercedes-Benz O 530 GL CapaCity \#4940
  	\item \textbf{Umiestnenie:} nad motorom vpravo vzadu, posledné zadné priečne sedadlo pred plošinou na batožinu
  	\end{itemize}

\paragraph{L4\_7954\_ZaluhyKrizovatka\_KutikyObratisko.csv}
	\begin{itemize}[noitemsep, topsep=0pt]
  	\item \textbf{Linka:} 4
  	\item \textbf{Trvanie:} 97362 vzoriek (194,72 s)
  	\item \textbf{Zastávky:} Záluhy (semafór, pokoj),  Horné Krčace, Dolné Krčace, Kútiky obratisko za druhou výhybkou (pohyb)
  	\item \textbf{Vozidlo:} električka ČKD Tatra T6A5 \#7954
  	\item \textbf{Umiestnenie:} zadný vozeň v okolí nad ľavou časťou prednej nápravy
  	\end{itemize}

\clearpage