\chapter{Analýza}

\section{Senzorová sieť}
Nízko-energetické zariadenia komunikujúce cez odľahčené sieťové protokoly so snahou spracovania v reálnom čase a ponechaním najdôležitejších informácii dolovaním z veľkého množstva zdrojových dát. Cloud / Fog comuting. Sink a Edge nodes

Vlastnosti senzorovej siete
\begin{itemize}
\itemsep0em 
\item Autokonfigurácia senzora - reakcia na zmeny v sieti a prostredia pôsobenia
\item Škálovateľnosť - veľké množstvo senzorov so spoločným účelom a schopnosťou vzájomnej kooperácie a interoperability.
\item Odolnosť voči chybám - v prípade pridania alebo odobratia uzla budú spojenia bez prerušenia.
\item Energeticky efektívna komunikácia uzlov - s upravenými protokolmi štandardného sieťového zásobníka
\begin{itemize}
\itemsep0em 
\item Event-driven - stály zber dát a reakcia na náhle zmeny. posielajú údaje až po prekročený kritického prahu
\item Query-driven - zbierajú údaje iba po prijatí dopytu od používateľa
\item Time-driven - pravidelne odosielajú údaje sinku. vzorkovaciu frekvenciu volí sink
\end{itemize}
\end{itemize}
\cite{wsn-overview}

Spracovanie toku informácií (IFP - Information flow processing) - nástroj na včasné spracovanie dát ako tečie z periférií do centra systému. Snahou je ukladanie agregovaných štatistík, napr. detektor požiaru za použitia čidiel teploty a dymu nepotrebuje ukladať jednotlivé merania, lebo sú samo o sebe nepodstatné. Keď nastane varovná situácia, je potrebné aby tá obsahoval všetky údaje na lokalizáciu ohniska.

CEP - Complex event processing - spracúva toky udalosti zo zdrojov reálneho sveta na základe aplikovania aktívnych pravidiel stanovených správcami systému a poupraví toky do komplexnejšieho výstupu. Pravidlá sú v tvare Event-Condition-Action (ECA).
\begin{itemize}
\itemsep0em
\item Udalosť - definuje zdroje ako generátory udalostí
\item Podmienka - uvažuje ktorá časť udalosti bude braná do úvahy pri spracovaní, napr. môže ísť o prekročenie prahu
\item Akcia - aká sada úloh má byť vykonaná pri detekcii udalosti
\end{itemize}

Behové pravidlá sú spracúvané vo viacerých fázach
\begin{itemize}
\item Signalizácia - detekcia udalosti
\item Spustenie - asociácia udalosti so sadou pravidiel
\item Vyhodnotenie - vyhodnotenie podmienky
\item Plánovanie - stanovenie poradia vykonania
\item Vykonanie - vykonanie pravidlá
\end{itemize}
\cite{processing-information-flows}

\subsection{Senzorová jednotka}

Súčasti senzorovej jednotky:
\begin{itemize}
\item Zberná jednotka
\item Výpočtová jednotka
\item Komunikačná jednotka
\item Napájacia jednotka
\end{itemize}

Obmedzenia na senzorové uzly 
\begin{itemize}
\item Spotreba energie - energetická autonómia uzlov vo WSN umožňuje nasadzovanie zariadení do odľahlých miest pre využitie v inteligentných mestách alebo na účely ochrany prírody. životnosť senzorovej jednotky je ohraničená kapacitou batérie.
\item Dosah komunikácie - Senzory disponujú obmedzenou energiou na vysielanie a dosah je negatívne ovplyvnený silou signálu na anténe. Z toho vyplývajú aj nižšie prenosové rýchlosti.
\item Výpočtový výkon a úložisko - Nízka taktovacia frekvencia procesora v megaherzoch a veľkosti pracovných pamätí v stovkách kilobajoch alebo megabajtoch.
\end{itemize}
\cite{big-data-collection-wsn}


Za ideálnych okolností by sa mal online algoritmus učiť kontinuálne bez ukladania predošlých bodov a detekcií.
V rozhodnutiach algoritmu sú zahrnuté informácie o všetkých predošlých bodoch do terajšieho rozhodnutia. Mal by mať schopnosť sa adaptovať dynamickému prostrediu, v ktorom pôsobí. Bez nutnosti manuálnych úprav parametrov modelu. Zároveň je žiaduce minimalizovať falošné pozitíva a negatíva pri detekcii udalostí.

Kontinuálne spracovanie častokrát vyžaduje, aby boli dáta rozdelenie do rovnako dlhých blokov zvaných okná. Okno $\mathcal{W}_{\sigma, \pi}$ je funkciou morfujúca maticu vstupných dát $X$ do vektora $w = (W_1, ... , W_q)$  \cite{online-anomaly-detection}

\section{Monitorovanie vibrácií a šoku}
Vibrácie sú pohyb hmoty okolo rovnovážneho stavu, ktorý býva vzniká zväčša ako vedľajší nežiaduci produkt
inak užitočných javov. Dlhodobé vystavenie súčiastok na vibrácie môže napr. poškodzovať ložiská motorov.
 
Fyzikálny model vibrácii (s jedným stupňom voľnosti) ako oscilatórny systém
Zjednodušený mechanický model s pružinou ($kx$), tlmič ($c\frac{dx}{dt}$) a hmota ($m\frac{d^2x}{dt^2}$)
Odozva na mechanické vibrácie (voľné a nútené) a šok
\cite{vibrations-shock}
 
\subsection{Meranie fyzikálnej veličiny akcelerácie}
Data pipeline - Technológia MEMS 3DOF akcelerometrov, Vzorkovanie, kvantovanie, kódovanie analógovo-digitálnym prevodíkom -

\subsection{MEMS kapacitný akcelerometer}
\cite{mdof-mems-accelerometers} 

\subsection{Analógovo-digitálny prevodník}
vzorkovacia frekvencia, aliasing, prepočet z A/D hodnoty na akceleráciu. Kvantizačný šum

Adaptívne vzorkovanie - preskočenie vzorky, keď je predpoklad, že na základe existujúcich vzoriek vieme dostatočne dobre odhadnúť budúce čítania. \cite{adaptive-sampling}. Efekt zberu až po prekročení prahu intenzity akcelerácie na signál.

\subsection{Odvodzovanie rýchlosti a polohy zo zrýchlenia}
fyzikálne vzťahy pre polohu, rýchlosť a zrýchlenie, numerická derivácia a integrácia (obdĺžníkové, lichobežníkové, Simposonovo pravidlo) \cite{integration-acceleration-envelopes}

\begin{equation}
a = x_{ADC} \cdot \frac{(2^{b_{ADC}} / 2)}{(G_{res} \cdot g)};\; g = 9.81\,m/s^2;\; b_{ADC} = 12\,\mathrm{bit};\; G_{res} = 2\,g
\end{equation}

\section{Metódy analýzy signálu v časovej doméne}
časový rad, okná
\cite{time-series-analysis} \cite{practical-time-series} \cite{generalized-esd} \cite{twitter-esd}, 
online algoritmus \cite{online-anomaly-detection}, 
požiadavky na efektívne algoritmy
	

\subsection{Číselné charakteristiky štatistického rozdelenia}
bodové odhady momentov. výberový priemer = stredná hodnota, výberový rozptyl (Welfordov online algoritmus), šikmosť, špicatosť, kvantily, normálne rozdelenie pravdepodobnosti

\subsection{Prúdové algoritmy}
stream algorithm models, Data Stream Model - cash register, turnstile, time series, window model
Odhad momentov, Rátanie frekvencií, Zistenie či sa dopyt už vyskytol
Count–min sketch (\url{https://florian.github.io/count-min-sketch/}) - probabilistická dátová štruktúra

Shewart Control Chart, CUSUM


\subsection{Online detekcia anomálií a odchýliek pozorovaní}
Outlier je pozorovanie, ktoré sa odchyluje, tak významne od ostatných pozorovaní, že vzbudzuje
podozrenie, že bolo vytvorené odlišným mechanizmom. Normálne dáta, Šum, Anomálie (slabé a silné odchýlky)
Výstupy algoritmov na detekciu výchyliek: 
\begin{itemize}
\itemsep0pt
\item Outlier skóre - miera vychýlenosti bodu
\item Binárne štítky - binárne štítky, označenie, či je bod anomália alebo nie.
\end{itemize}
Algoritmy na anomálie vytvárajú model normálnych vzorov v dátach a skóre vychýlenosti je dané deviáciou od týchto vzorov.
\cite{outlier-analysis} 
Hampel filter

\cite{survey-univariate-time-series} 
Množstvo prenášaných a spracovaných dát prevyšuje ľudskú schopnosť manuálneho prieskumu. Anomália je pozorovanie alebo postupnosť pozorovaní, ktoré sa významne odchyluje od distribúcie zvyšku dát. 
\begin{itemize}
	\item Bodové anomálie - $x_t$ je bodová anomália ak sa jeho hodnota významne odlišuje od všetkých 		
		bodov v intervale $ [x_{t-k}; x_{t+k}] $
	\item Kolektívne (disonanancie) - jednotlivé body nepredstavujú anomálne správanie, až ak vezmeme dlhšiu postupnosť môže byť označená za anomáliu.
	\item Kontextové výchylky / anomálie - body sú normálne v určitom kontexte, ale v inom anomálne.
\end{itemize}		

\cite{review-outlier-datection} \cite{anomaly-detection-algorithms}
\begin{equation}
|x - \hat{x}| \geq \theta 
\end{equation}

Bežiaci filter  
\cite{anomaly-detection-models}

Skóre anomálnosti, 	binárny klasifikátor, 
Falošne pozívny - proces je normálny, ale registrujeme neočakávané správanie, 
Falošne negatívny - proces je abnormálny, ale správanie prechádza bez povšimnutia
Matica zámen, Precision, Recall, 
ROC - kvalita binárneho klasifikátora v závislosi od prahu, AUC
\cite{wsn-outlier-detection-survey}

Z-score a Z-test: 
\begin{equation}
z = \frac{|x - \mu|}{\sigma} \approx \frac{|x - \bar{x}|}{S}
\end{equation}
Bežiaci priemerovací filter \cite{anomaly-detection-models}

Zisťovanie výchyliek testovaním štatistických hypotéz: 
generalized extreme Studentized deviate test \cite{generalized-esd} 

Median Absolute Deviate (robustná štatistika na určenie vychýlenosti od normálu)
\begin{equation}
MAD = \mathrm{median}(|X_i - \bar{X}|)
\end{equation}

Klastrovanie s DBSCAN / LOF
\cite{change-theory}

\subsection{Algoritmy na rozpoznávanie špičiek}
lokálne minimá a maximá extrémy - cez prvú a druhú deriváciu,
topografická prominencia a izolácia - relatívna výška vrchola / extrému, 

\cite{survey-peaks-valleys}
\cite{peek-mountaineer-method}
\cite{ecg-r-peak-detection}
\cite{ampd-algorithm}

\section{Frekvenčná a časovo-frekvenčná analýza signálu}
vlastnosti frekvenčného spektra, decibele, spektrogram, odstup od šumu (SNR), power spectrum density (dbFS), spektrálny analyzátor 

\subsection{Fourierová transformácia}
Diskrétna fourierová transformácia mapuje sinál dĺžky $N$ do množiny $N$ diskrétnych frekvenčných komponentov. \cite{signal-processing}
\begin{equation}
X = \mathbf{W}x; W_{nk} = e^{-i\frac{2\pi}{N}nk} = W_N^{nk}
\end{equation}
Inverzná transformácia
\begin{equation}
x = \frac{1}{N}\mathbf{W}^H X
\end{equation}

Integrálne transformácie: Fourierová transformácia (CFT, DFT), Kosínusová transfomácia (MDCT), Vlnková transformácia (CWT) \cite{dct} \cite{casove-frekvencia-analyza-signalu}

\begin{equation}
T(n) = \int{f(t) K(t,x) \mathrm{dt}}
\end{equation}

\begin{equation}
\mathcal{F}: X(\omega) = \int_{-\infty}^{+\infty}{x(t) \cdot e^{-i\omega t} \mathrm{dt}}
\end{equation}

\subsection{Algoritmus FFT pre DFT a DCT}
Opis DIT radix-2 FFT algoritmus komplexných, reálny, pre MDCT \cite{fft-blackbox}

\begin{equation}
X(m) = \sum_{n = 0}^{N-1}{x(n) \cdot e^{-i2\pi n m / N}}
\end{equation}

Frekvenčné rozlíšenie
\begin{equation}
\Delta f = \frac{f_s}{N}
\end{equation}

\subsection{Oknové funkcie}
Gaborová transformácia, Prehľad okien a ich transformácií (sinc), Efekt oknových funkcií na spectral leakage, výhodné percentá prekryvu FT 	\cite{understanding-dsp} \cite{spectral-density-estimation}
Priemerovanie a prekryv - Amplitude Flatness (AF), Power Flatness (PF), Overlap Correlation (OC)

\subsection{Filtre s konečnou impulznou odozvou}
Roziel medzi FIR a IIR, Dolná pripusť, pásmová priepusť, horná pripusť,
 Konvolúcia a konvolučné jadro, konvolučná veta, účel: identifikácia prítomnosti známej frekvencie v signále akcelerácie
\begin{equation}
y(n) = \sum_{k=0}^{D_y}{x(k) \cdot h(n-k)} = x(n) * h(n)
\end{equation}

Prenosová funkcia
\begin{equation}
H(\Omega) = \frac{Y(\Omega)}{X(\Omega)}
\end{equation}	

Detektor obálok	\footnote{\url{https://www.mathworks.com/help/dsp/ug/envelope-detection.html}} \footnote{\url{https://www.dsprelated.com/showarticle/938.php}}

\subsection{Algoritmus separácie diskrétnej energie}
Odhad okamžitej frekvencie, DESA, TKEO (Teager-Kaiserov energetický operátor) \cite{eeg-spanok} 
\begin{equation}
\psi(x(n)) = x^2(n) - x(n-1) \cdot x(n+1) \approx A^2\omega^2
\end{equation}
	
\emptypage 
