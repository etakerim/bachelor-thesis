\chapter{Analýza}

\section{Senzorová sieť}
Opis prostredia a konštrukčné obmedzenia - nízko-energetické zariadenia komunikujúce cez odľahčené sieťové
protokoly so snahou spracovania v reálnom čase a ponechaním najdôležitejších informácii dolovaním z veľkého množstva zdrojových dát. Metódy prístupy: dopytové, periodický zber.
\cite{wsn-overview}

\subsection{Senzorová jednotka}
Acquistion unit, Processing unit, Communication unit, ARM Cortex M4
\cite{big-data-collection-wsn}

\section{Monitorovanie vibrácií a šoku}
Fyzikálny model vibrácii (s jedným stupňom voľnosti) ako oscilatórny systém, fyzikálne jednotky
\cite{vibrations-shock}
 
\subsection{Meranie fyzikálnej veličiny akcelerácie}
Data pipeline - Technológia MEMS 3DOF akcelerometrov, Vzorkovanie, kvantovanie, kódovanie analógovo-digitálnym prevodíkom - vzorkovacia frekvencia, aliasing, prepočet z A/D hodnoty na akceleráciu. \cite{mdof-mems-accelerometers} 

Adaptívne vzorkovanie - preskočenie vzorky, keď je predpoklad, že na základe existujúcich vzoriek vieme dostatočne dobre odhadnúť budúce čítania. \cite{adaptive-sampling}. Efekt zberu až po prekročení prahu intenzity akcelerácie na signál.

\subsection{Odvodzovanie rýchlosti a polohy zo zrýchlenia}
fyzikálne vzťahy pre polohu, rýchlosť a zrýchlenie, numerická derivácia a integrácia (obdĺžníkové, lichobežníkové, Simposonovo pravidlo) \cite{integration-acceleration-envelopes}

\begin{equation}
a = x_{ADC} \cdot \frac{(2^{b_{ADC}} / 2)}{(G_{res} \cdot g)};\; g = 9.81\,m/s^2;\; b_{ADC} = 12\,\mathrm{bit};\; G_{res} = 2\,g
\end{equation}

\section{Metódy analýzy signálu v časovej doméne}
časový rad, okná
\cite{time-series-analysis} \cite{practical-time-series} \cite{generalized-esd} \cite{twitter-esd}, 
online algoritmus \cite{online-anomaly-detection}, 
požiadavky na efektívne algoritmy
	

\subsection{Číselné charakteristiky štatistického rozdelenia}
bodové odhady momentov. výberový priemer = stredná hodnota, výberový rozptyl (Welfordov online algoritmus), šikmosť, špicatosť, kvantily, normálne rozdelenie pravdepodobnosti

\subsection{Algoritmy na rozpoznávanie špičiek}
lokálne minimá a maximá extrémy - cez prvú a druhú deriváciu,
topografická prominencia a izolácia - relatívna výška vrchola / extrému, 
Z-score a Z-test: 
\begin{equation}
z = \frac{|x - \mu|}{\sigma} \approx \frac{|x - \bar{x}|}{S}
\end{equation}
Zisťovanie výchyliek testovaním štatistických hypotéz: generalized extreme Studentized deviate test, 
Median Absolute Deviate
\begin{equation}
MAD = \mathrm{median}(|X_i - \bar{X}|)
\end{equation}
Shewart Control Chart, CUSUM \cite{change-theory}
	 
\subsection{Online detekcia anomálií a odchýliek pozorovaní}
\cite{survey-univariate-time-series} 
Množstvo prenášaných a spracovaných dát prevyšuje ľudskú schopnosť manuálneho prieskumu. Anomália je pozorovanie alebo postupnosť pozorovaní, ktoré sa významne odchyluje od distribúcie zvyšku dát. 
\begin{itemize}
	\item Bodové anomálie - $x_t$ je bodová anomália ak sa jeho hodnota významne odlišuje od všetkých 		
		bodov v intervale $ [x_{t-k}; x_{t+k}] $
	\item Kolektívne (disonanancie) - jednotlivé body nepredstavujú anomálne správanie, až ak vezmeme dlhšiu postupnosť môže byť označená za anomáliu.
	\item Kontextové výchylky / anomálie - body sú normálne v určitom kontexte, ale v inom anomálne.
\end{itemize}		

\cite{review-outlier-datection} \cite{anomaly-detection-algorithms} \cite{outlier-analysis}
\begin{equation}
|x - \hat{x}| \geq \theta 
\end{equation}

Skóre anomálnosti, 	binárny klasifikátor, Falošne pozívny/negatívny, Matica zámen, Precision, Recall, 
ROC - kvalita binárneho klasifikátora v závislosi od prahu, AUC

\section{Frekvenčná a časovo-frekvenčná analýza signálu}
vlastnosti frekvenčného spektra, decibele, spektrogram, odstup od šumu (SNR), power spectrum density (dbFS), spektrálny analyzátor 

\subsection{Fourierová transformácia}
Integrálne transformácie: Fourierová transformácia (CFT, DFT), Kosínusová transfomácia (MDCT), Vlnková transformácia (CWT) \cite{dct} \cite{casove-frekvencia-analyza-signalu}

\begin{equation}
T(n) = \int{f(t) K(t,x) \mathrm{dt}}
\end{equation}

\begin{equation}
\mathcal{F}: X(\omega) = \int_{-\infty}^{+\infty}{x(t) \cdot e^{-i\omega t} \mathrm{dt}}
\end{equation}

\subsection{Algoritmus FFT pre DFT a DCT}
Opis DIT radix-2 FFT algoritmus komplexných, reálny, pre MDCT \cite{fft-blackbox}

\begin{equation}
X(m) = \sum_{n = 0}^{N-1}{x(n) \cdot e^{-i2\pi n m / N}}
\end{equation}

Frekvenčné rozlíšenie
\begin{equation}
\Delta f = \frac{f_s}{N}
\end{equation}

\subsection{Oknové funkcie}
Gaborová transformácia, Prehľad okien a ich transformácií (sinc), Efekt oknových funkcií na spectral leakage, výhodné percentá prekryvu FT 	\cite{understanding-dsp} \cite{signal-processing} \cite{spectral-density-estimation}

\subsection{Filtre s konečnou impulznou odozvou}
Roziel medzi FIR a IIR, Dolná pripusť, pásmová priepusť, horná pripusť,
 Konvolúcia a konvolučné jadro, konvolučná veta, účel: identifikácia prítomnosti známej frekvencie v signále akcelerácie
\begin{equation}
y(n) = \sum_{k=0}^{D_y}{x(k) \cdot h(n-k)} = x(n) * h(n)
\end{equation}

Prenosová funkcia
\begin{equation}
H(\Omega) = \frac{Y(\Omega)}{X(\Omega)}
\end{equation}	

Detektor obálok	\footnote{\url{https://www.mathworks.com/help/dsp/ug/envelope-detection.html}} \footnote{\url{https://www.dsprelated.com/showarticle/938.php}}

\subsection{Algoritmus separácie diskrétnej energie}
Odhad okamžitej frekvencie, DESA, TKEO (Teager-Kaiserov energetický operátor) \cite{eeg-spanok} 
\begin{equation}
\psi(x(n)) = x^2(n) - x(n-1) \cdot x(n+1) \approx A^2\omega^2
\end{equation}
	
\section{Kompresia na redukciu objemu dát pri prenose}
Kompresný pomer
\subsection{Bezstrátová kompresia}
LZ77, Huffman 
\subsection{Stratová kompresia}
DCT, DWT, Stratová (Kvantizačné tabuľky)
	
\emptypage 
