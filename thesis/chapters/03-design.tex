\chapter{Návrh riešenia}
Požiadavky
\begin{itemize}
\item Najvhodnejší spôsob identifikácie špičiek vo frekvenčnej doméne
\item Možnosti redukcie zberu dát za cenu redukcie energetické nároky a prenosu významných čŕt signálu
\item Pravidlový systém na definíciu udalostí záujmu a ich spoľahlivá identifikácia
\end{itemize}

Riešenia:
Pravidlá
	- Zbieranie vzoriek nad určitý prah - redukcia formátu, agregácia za okná,
	- Detekcia zmien (Zhromažďovanie oknových štatistík)
	- Pipeline detekcie frekvenčných zložiek:
		- Nastavenie vzorkovania podľa rozlíšenia
		- Výber osí alebo získanie magnitúdy
		- Odstránenie trendu (obálky) a jednosmernej zložky (odstránenie priemeru)
		- Oknovanie (porovnánie oknových funkcií, dĺžka okna)
		- Prevod do frekvenčnej domény (porovnanie algoritmov)
		- Priemerovanie spektrogramov
		- Vyhladzovanie cez Mean filter
		- Priemerovanie spektrogramov Welchov metódou pre stálejší obraz spektra
		- Detekcia špičiek
		- Vygenerovanie udalostí podľa frekvenčných binov (frekvencia, čas trvania) podľa parametrov
	- Pozorovanie prítomnosti konkrétnych zložiek cez FIR filter
	- Extrahovanie rýchlosti a zrýchlenia
Sumarizácia:
Proces FT - Preprocessing: DC and trend removal:
Oknovanie (predpočítané koeficienty okna pre N) a
FFT motýliky (predpočítané twiddle factors pre N)
\emptypage
