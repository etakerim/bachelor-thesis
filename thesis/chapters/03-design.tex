\chapter{Opis riešenia}
Požiadavky
\begin{itemize}
\item Najvhodnejší spôsob identifikácie lokálnych extrémov - špičiek
\item Možnosti redukcie zberu dát za cenu redukcie energetické nároky a prenosu významných čŕt signálu
\item Pravidlový systém na definíciu udalostí záujmu operátorov a ich spoľahlivá identifikácia
\item Upozornenie na nezvyčajnosti pri prevoze (detekciou anomálií) - veľká nerovnosť, neočakávaný pohyb
\end{itemize}

Senzorová jednotka s akcelerometrom na meranie vibrácii a nárazov pri prevoze krehkých látok/materiálov upozorňujúca na základe konfigurovateľných pravidiel alebo nezvyčajných vzorov (pozn.:lepšie dodefinovať). Preskúmanie možností redukcie zberu alebo nasnímaných údajov - nastavenie vzorkovacej frekvencie / akceptovanie nad stanovenú amplitúdu / kompresia. Rozšírenie: /distribuovaná/ korelácia údajov z viacerých akcelometrov z jedného balíka / vozidla.

%\section{Hardvér senzorovej jednotky}
%\section{Vývojové prostredie a knižnice}
% CMSIS, MDK5, OPENOCD, GCC
%\section{Koordinácia subsystémov}
%\section{Konzervácia energie znížením vzorkovania}
%\section{Pravidlový systém na extrahovanie čŕt záujmu}


%Táto časť bakalárskeho projektu obsahuje opis výsledkov riešenia jednotlivých etáp projektu. V prípade, že záverečný projekt nerieši 
%všetky etapy, malo by byť v príslušnej časti uvedené kto, resp. kde sa príslušná etapa rieši/riešila/bude riešiť.

% Typické etapy riešenia pri tvorbe softvérového systému: 
% špecifikácia požiadaviek, návrh, implementácia (ak to zadanie požaduje), overenie riešenia

%Podľa možností treba vychádzať zo známych prístupov (napr. pri softvérových projektoch štruktúrovaný alebo objektovo orientovaný %prístup) a techník (napr. blokové schémy, vývojové diagramy, UML, entito-relačné diagramy atď.). Táto časť práce závisí od konkrétneho %zadania.
%Je dôležité prezentovať návrhové rozhodnutia, alternatívy, ktoré sa zvažovali pri riešení a samotný návrh riešenia zadaného problému. %Štruktúrovanie textu tejto časti BP by malo vychádzať zo zadanej úlohy, ktorá sa rieši. Najmä v tejto časti študent preukazuje tvorivý %prístup k riešeniu problémov a kritické myslenie.
\emptypage 
